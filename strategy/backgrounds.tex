\label{sec:backgrounds}
%% An accurate description of the background process is an essential aspect of any analysis since it affects the extraction of signal yields.
The main background sources differ between the three channels, but can be divided into two categories.

In the first one there are processes which have the same final state as the signal and survive all signal region cuts (\textit{irreducible backround}).
These are processes that generate the same stable particles as the signal,
although the kinematic distributions may be different.
Usually processes in this class are estimated with simulation,
but in some cases it is possible to constrain their normalisation in a control region.

In the second one fall processes that have a different final state than the signal, but enter in the signal region nonetheless (\textit{reducible background}).
Although the final states are different,
either due to additional particles produced in the same hard scattering
that are reconstructed in the detector or the identification algorithms,
or coming from other collisions in the same bunch crossing (\pileup),
they generate events which pass the selection.
These processes have cross sections orders of magnitude larger than the signal.
Often these backgrounds prove difficult to model in simulation,
and it becomes advisable to use a data-driven method for their estimation.

\paragraph{Four lepton channel\\}
For the 4\Pl channel the predominant background component is the production of two on-shell \PZ bosons
and a photon that is either radiated as FSR from one of the leptons
or from a misreconstructed jet.
%% As described in Section \ref{sec:simulation}
It can either be produced from two quarks,
or proceed from a gluon initiated loop.
Additional backgrounds such as $\PQt\PAQt\PZ$ and VVV (V = \PZ, \PW) result in very small contributions.

\todo{Expand, e.g. ZZ from qq/gg}

\paragraph{Three lepton channel}
In the 3\Pl channel the main background contributions are Drell-Yan + jets, $\PZ\PGg$ + jets
and $\PW\PZ$ plus a photon that is either radiated as FSR from one of the leptons or from a misreconstructed jet.
Another significant contribution comes from $\PZ\PZ \to 4\Pl$ where one of the leptons is lost or misreconstructed as a photon.
There is also a small fraction of events from $\PZ\PZ\PGg$ in which one of the leptons is outside the detector acceptance. \todo{should check in MC}
Additional backgrounds from rare processes such as $\PQt\PAQt\PZ$ and VVV (V = \PZ, \PW) result in very small contributions and are estimated with simulation.

\paragraph{Two lepton channel\\}
\todo{Describe backgrounds in 2L}

\paragraph{Reducible background\\}
This group of backgrounds arises from processes in which some of the reconstructed leptons or photons do not correspond to real, prompt leptons from the vector bosons decay or photons from the hard process.
This particles can either be \nonprompt leptons or photons, or misidentified light-flavour jets.

\Nonprompt leptons come mainly from decays of heavy flavour mesons and electrons from asymmetric photons conversions,
while \nonprompt photons originate primarily from decays of light neutral mesons like \PGpz or \PGh.
Both leptons and photons in this category tend to be non-isolated from the nearby jet activity.

The other class is comprised of misidentified jets, mostly from light-flavour quarks, which can erroneously be reconstructed as either leptons,
if a track is associated to the main energy deposits, or as photons otherwise.
These misidentified photons tend to have a different energy distribution in the ECAL with respect to real photons,
which makes shower shape variables effective in separating this background from real photons.

In the following the terms \textit{fake leptons} and \textit{fake photons} are used to refer to both \nonprompt and misidentified objects.
