\label{sec:lepCR3l}
%% \note{SMP-16-002~\cite{SMP-16-002} (2015 data) uses a dijet region to measure the fake rate:
%% \textit{``The misidentification probability is measured from a sample of
%% dijet events enriched in nonprompt leptons. The sample is selected
%% with one jet passing the relaxed lepton identification requirements
%% matched to a single lepton trigger, defined as the probe lepton.''}}

%% \note{SMP-20-014~\cite{SMP-20-014} (full Run2) uses a L+j region:
%% \textit{``This CR is defined by requiring a single lepton with pT greater than 10 GeV, and at least
%% a reconstructed jet that is well separated from the lepton at $\DR(\PGg, j) > 0.7$. Contributions from
%% EWK processes are subtracted to obtain a pure nonprompt measurement region.''}}

Seven leptonic control regions are defined for the three lepton channel, following the strategy used in Reference~\cite{SMP-20-014},
using the same selections as the signal region except for the lepton identification.
The total number of events in each of these categories are denoted
$N_{LLL}$, $N_{LLT}$, $N_{LTL}$, $N_{TLL}$, $N_{LTT}$, $N_{TLT}$, and $N_{TTL}$, where the subindex denotes whether the lepton is tight ($T$) or loose but not tight ($L$).

This classification produces three regions where only one lepton fails the selection, three where two leptons fail and one where all three leptons are \nonprompt.
The agreement between the simulation and the data decreases accordingly, given the instrumental nature of this background, which makes it difficult to model.

\paragraph{Contribution to signal region\\}
The yield in each category can be expressed as a function of the number of events
with a given composition of prompt ($P$) or \nonprompt ($F$) leptons:
$N_{FFF}$, $N_{FFP}$, $N_{FPF}$, $N_{PFF}$, $N_{FPP}$, $N_{PFP}$, and $N_{PPF}$.
Assuming that the prompt leptons always pass the tight selection,
the number of \nonprompt events in the signal region $N_{TTT}$ can be related to the quantities defined previously:
\begin{equation}
  \resizebox{\textwidth}{!}{
    $\begin{aligned}
      N_{TTT} &= f_1 N_{FPP} + f_2 N_{PFP}+f_3 N_{PPF}+f_1 f_2 N_{FFP}+f_2 f_3 N_{PFF}+f_1 f_3 N_{FPF}+f_1 f_2 f_3 N_{FFF}\\
      N_{TTL} &= (1-f_3) N_{PPF} + f_2(1-f_3) N_{PFF} + f_1(1-f_3) N_{FPF} + f_1 f_2(1-f_3) N_{FFF}\\
      N_{TLT} &= (1-f_2) N_{PFP} + f_3(1-f_2) N_{PFF} + f_1(1-f_2) N_{FFP} + f_1 f_3(1-f_2) N_{FFF}\\
      N_{LTT} &= (1-f_1) N_{FPP} + f_2(1-f_1) N_{FFP} + f_3(1-f_1) N_{FPF} + f_3 f_2(1-f_1) N_{FFF}\\
      N_{TLL} &= (1-f_2)(1-f_3) N_{PFF} + f_1(1-f_2)(1-f_3) N_{FFF}\\
      N_{LTL} &= (1-f_1)(1-f_3) N_{FPF} + f_2(1-f_1)(1-f_3) N_{FFF}\\
      N_{LLT} &= (1-f_2)(1-f_1) N_{FFP} + f_3(1-f_2)(1-f_1) N_{FFF}\\
      N_{LLL} &= (1-f_1)(1-f_2)(1-f_3) N_{FFF}
    \end{aligned}$
  }
\end{equation}
where $f_i = f(\pt^{i},\, \eta^{i})$ is the probability of the $i$-th loose \nonprompt lepton to pass the tight criteria (i ranges from 1 to 3).
Then solving this yields $N_{TTT}$ as a function of the yields in the application regions:
\begin{equation}
\begin{split}
N_{TTT} =
  \frac{f_1}{1-f_1} N_{LTT}
+ \frac{f_2}{1-f_2} N_{TLT}
+ \frac{f_3}{1-f_3} N_{TTL}
+ \frac{f_1 f_2 f_3}{(1-f_1)(1-f_2)(1-f_3)} N_{LLL}\\
- \frac{f_1 f_2}{(1-f_1)(1-f_2)} N_{LLT}
- \frac{f_1 f_3}{(1-f_1)(1-f_3)} N_{LTL}
- \frac{f_2 f_3}{(1-f_2)(1-f_3)} N_{TLL}
\end{split}
\end{equation}
