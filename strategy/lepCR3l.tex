\label{sec:lepCR3l}
\note{At the moment, the data-driven fake estimate does NOT work in SR3P. Maybe we shouldn't discuss it if we don't use it.}

\note{SMP-16-002~\cite{SMP-16-002} (2015 data) uses a dijet region to measure the fake rate:
\textit{``The misidentification probability is measured from a sample of
dijet events enriched in nonprompt leptons. The sample is selected
with one jet passing the relaxed lepton identification requirements
matched to a single lepton trigger, defined as the probe lepton.''}}

\note{SMP-20-014~\cite{SMP-20-014} (full Run2) uses a L+j region:
\textit{``This CR is defined by requiring a single lepton with pT greater than 10 GeV, and at least
a reconstructed jet that is well separated from the lepton at $\DR(\PGg, j) > 0.7$. Contributions from
EWK processes are subtracted to obtain a pure nonprompt measurement region.''}}

Seven leptonic control regions are defined for the three lepton channel, following the strategy used in Reference~\cite{SMP-20-014},
using the same selections as the signal region except for the lepton identification.
The leptons are ordered: $\Pl^\PZ_1$, $\Pl^\PZ_2$ and $\Pl^\PW$
and the regions are defined based on which lepton fails the tight selection.
Unlike the four lepton channel the order is important,
so for example an event where only the leading lepton from the \PZ fails will be in a different control region
from an event where only the subleading lepton does not pass the tight selection.

This classification produces three regions where only one lepton fails the selection, three where two leptons fail and one where all three leptons are \nonprompt.
The agreement between the simulation and the data decreases accordingly, given the instrumental nature of this background, which makes it difficult to model.

\begin{figure}
  \subfigure [$\Pl^\PW$ fails  ] {\includegraphics[width=.333333333\textwidth]{Figures/dataMC_noLFR/Run2/fullMC/CR110/lll_mass_pow.pdf}}%
  \subfigure [$\Pl^\PZ_1$ fails] {\includegraphics[width=.333333333\textwidth]{Figures/dataMC_noLFR/Run2/fullMC/CR011/lll_mass_pow.pdf}}%
  \subfigure [$\Pl^\PZ_2$ fails] {\includegraphics[width=.333333333\textwidth]{Figures/dataMC_noLFR/Run2/fullMC/CR101/lll_mass_pow.pdf}}
  \caption{Invariant mass of the three charged leptons in the three lepton control regions for the 3L channel where only one lepton fails the selection.}
  \label{fig:CR3L_1_Run2}
\end{figure}

\begin{figure}
  \subfigure [$\Pl^\PZ_1$ and $\Pl^\PZ_2$ fail] {\includegraphics[width=.333333333\textwidth]{Figures/dataMC_noLFR/Run2/fullMC/CR001/lll_mass_pow.pdf}}%
  \subfigure [$\Pl^\PZ_1$ and $\Pl^\PW$ fail  ] {\includegraphics[width=.333333333\textwidth]{Figures/dataMC_noLFR/Run2/fullMC/CR010/lll_mass_pow.pdf}}%
  \subfigure [$\Pl^\PZ_2$ and $\Pl^\PW$ fail  ] {\includegraphics[width=.333333333\textwidth]{Figures/dataMC_noLFR/Run2/fullMC/CR100/lll_mass_pow.pdf}}
  \caption{Invariant mass of the three charged leptons in the three lepton control regions for the 3L channel where two leptons fail the selection.}
  \label{fig:CR3L_2_Run2}
\end{figure}
