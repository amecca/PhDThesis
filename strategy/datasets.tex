\section{Dataset and samples}
\label{sec:datasets}

\subsection{Trigger selection}
\label{sec:triggers}
Several HLT paths are used to select events which present a certain number of electrons or muons in the final state.
Due to the evolution of trigger requirements with instantaneous LHC luminosity,
the collection of HLT paths used in this analysis is different for each data taking year.

The HLT paths used for collision data are listed in Tables~\ref{tab:triggerpaths2016}, \ref{tab:triggerpaths2017} and \ref{tab:triggerpaths2018},
together with
%% their L1 seed, prescale value and
the associated primary dataset.
The trigger paths and the disambiguation scheme are the same used in Reference~\cite{CMS-PAS-HIG-19-001},
and they were optimised for the phase space of the $\PH \to \PZ \PZst \to 4\Pl$ analysis.

\begin{table*}
  \caption{Trigger paths used in 2016 collision data. All triggers have prescale = 1.}
  \label{tab:triggerpaths2016}
  \centering
  \small
  \begin{tabular}{ l l l }
    \toprule %--------------------------------------------------------------------------------------------------------------------------
    Trigger description & Requirements & Primary Dataset \\
    \midrule %--------------------------------------------------------------------------------------------------------------------------
    Double isolated electron     & $\pt^{\Pe_1}, \pt^{\Pe_2} > 17, 12 \GeVc$                 & \multirow{4}{*}{DoubleEG} \\
    Double isolated electron     & $\pt^{\Pe_1}, \pt^{\Pe_2} > 23, 12 \GeVc$                 & \\
    Double isolated GSF electron & $\pt^{\Pe_1}, \pt^{\Pe_2} > 33, 33 \GeVc$                 & \\
    Triple isolated electron     & $\pt^{\Pe_1}, \pt^{\Pe_2}, \pt^{\Pe_3} > 16, 12, 8 \GeVc$ & \\
    \hline
    Double isolated muon & $\pt^{\PGm_1}, \pt^{\PGm_2} > 17, 8 \GeVc$                   & \multirow{3}{*}{DoubleMuon} \\
    Double isolated muon\hyperlink{tab:triggerpaths2016:fn1}{${}^1$}
                         & $\pt^{\PGm_1}, \pt^{\PGm_2} > 17, 8 \GeVc$                   & \\
    Triple isolated muon & $\pt^{\PGm_1}, \pt^{\PGm_2}, \pt^{\PGm_3} > 12, 10, 5 \GeVc$ & \\
    \hline
    Tracker muon \!+\! isolated electron & $\pt^{\PGm}, \pt^{\Pe} > 8 , 17$                   & \multirow{7}{*}{MuonEG} \\
    Tracker muon \!+\! isolated electron & $\pt^{\PGm}, \pt^{\Pe} > 8 , 23$                   & \\
    Tracker muon \!+\! isolated electron & $\pt^{\PGm}, \pt^{\Pe} > 17, 12$                   & \\
    Tracker muon \!+\! isolated electron & $\pt^{\PGm}, \pt^{\Pe} > 23, 12$                   & \\
    Tracker muon \!+\! isolated electron & $\pt^{\PGm}, \pt^{\Pe} > 23, 8 $                   & \\
    Tracker muon \!+\! double electron   & $\pt^{\PGm},\pt^{\Pe_1},\pt^{\Pe_2} >8,12,12\GeVc$ & \\
    Double tracker muon + electron   & $\pt^{\PGm_1},\pt^{\PGm_2},\pt^{\Pe}>9,9,9\GeVc$   & \\
    \hline
    Single tight ID central electron & $\pt^{\Pe}>25$, $|\eta^\Pe| < 2.1$ & \multirow{3}{*}{SingleElectron} \\
    Single tight ID electron         & $\pt^{\Pe}>27$                     & \\
    Single tight ID GSF electron     & $\pt^{\Pe}>27$, $|\eta^\Pe| < 2.1$ & \\
    \hline
    Single isolated muon & $\pt^{\PGm}>20$ & \multirow{2}{*}{SingleMuon} \\
    Single isolated muon & $\pt^{\PGm}>22$ & \\
    \bottomrule %-----------------------------------------------------------------------------------------------------------------------
    \noalign{\vspace{.5ex}} % small vertical space
    \multicolumn{3}{l}{\quad\hypertarget{tab:triggerpaths2016:fn1}{1}: $\mu_1$ is global, $\mu_2$ is tracker muon} \\
  \end{tabular}
\end{table*}

\begin{table*}
  \caption{Trigger paths used in 2017 collision data. All triggers have prescale = 1.}
  \label{tab:triggerpaths2017}
  \centering
  \small
  \begin{tabular}{ l l l }
    \toprule %--------------------------------------------------------------------------------------------------------------------------
    Trigger description & Requirements & Primary Dataset \\
    \midrule %--------------------------------------------------------------------------------------------------------------------------
    Double isolated electron & $\pt^{\Pe_1}, \pt^{\Pe_2} > 23, 12 \GeVc$                 & \multirow{3}{*}{DoubleEG} \\
    Double GSF electron      & $\pt^{\Pe_1}, \pt^{\Pe_2} > 33, 33 \GeVc$                 & \\
    Triple electron          & $\pt^{\Pe_1}, \pt^{\Pe_2}, \pt^{\Pe_3} > 16, 12, 8 \GeVc$ & \\
    \hline
    Double isolated muon & $\pt^{\PGm_1}, \pt^{\PGm_2} > 17, 8 \GeVc,\ m_{\PGm\PGm} > 3.8\GeVcc$ & \multirow{4}{*}{DoubleMuon} \\
    Double isolated muon & $\pt^{\PGm_1}, \pt^{\PGm_2} > 17, 8 \GeVc,\ m_{\PGm\PGm} > 8  \GeVcc$ & \\
    Triple muon          & $\pt^{\PGm_1}, \pt^{\PGm_2}, \pt^{\PGm_3} > 12, 10, 5\GeV$            & \\
    Triple muon          & $\pt^{\PGm_1}, \pt^{\PGm_2}, \pt^{\PGm_3} > 10,  5, 5\GeV$            & \\
    \hline
    Iso muon \!+\! iso electron & $\pt^\PGm, \pt^\Pe > 23, 12 \GeVc$                            & \multirow{7}{*}{MuonEG} \\
    Iso muon \!+\! iso electron & $\pt^\PGm, \pt^\Pe > 8 , 23 \GeVc$                            & \\
    Iso muon \!+\! iso electron & $\pt^\PGm, \pt^\Pe > 12, 23 \GeVc$                            & \\
    Iso muon \!+\! iso electron & $\pt^\PGm, \pt^\Pe > 23, 12 \GeVc$                            & \\
    Double muon \!+\! electron            & $\pt^{\PGm_1}, \pt^{\PGm_2}, \pt^\Pe > 9,9,9 \GeVc$ & \\
    Muon \!+\! double electron            & $\pt^\PGm, \pt^{\Pe_1}, \pt^{\Pe_2} > 8,12,12\GeVc$ & \\
    Muon \!+\! double electron            & $\pt^\PGm, \pt^{\Pe_1}, \pt^{\Pe_2} > 8,12,12\GeVc$ & \\
    \hline
    Single tight ID electron & $\pt^\Pe > 35 \GeVc$ & \multirow{3}{*}{SingleElectron} \\
    Single tight ID electron & $\pt^\Pe > 38 \GeVc$ & \\
    Single tight ID electron & $\pt^\Pe > 40 \GeVc$ & \\
    \hline
    Single isolated muon & $\pt^\PGm > 27 \GeVc$ & SingleMuon \\
    \bottomrule %-----------------------------------------------------------------------------------------------------------------------
  \end{tabular}
\end{table*}

\begin{table*}
  \caption{Trigger paths used in 2018 collision data. All triggers have prescale = 1.}
  \label{tab:triggerpaths2018}
  \small
  \centering
  \begin{tabular}{ l l l }
    \toprule %--------------------------------------------------------------------------------------------------------------------------
    Trigger description & Requirements & Primary Dataset \\
    \midrule %--------------------------------------------------------------------------------------------------------------------------
    Double isolated electron    & $\pt^{\Pe_1}, \pt^{\Pe_2} > 23, 12 \GeVc$ & \multirow{3}{*}{EGamma} \\
    Double electron             & $\pt^{\Pe_1}, \pt^{\Pe_2} > 25, 25 \GeVc$ & \\
    Single isolated electron    & $\pt^\Pe > 32 \GeVc$                      & \\
    \hline
    Double isolated muon        & $\pt^{\PGm_1}, \pt^{\PGm_2} > 17, 8 \GeVc,\ m_{\PGm\PGm} > 3.8\GeVcc$ & DoubleMuon \\
    \hline
    Iso muon \!+\! iso electron & $\pt^\PGm, \pt^\Pe > 23, 12 \GeVc$                  & \multirow{5}{*}{MuonEG} \\
    Iso muon \!+\! iso electron & $\pt^\PGm, \pt^\Pe >  8, 23 \GeVc$                  & \\
    Iso muon \!+\! iso electron & $\pt^\PGm, \pt^\Pe > 12, 23 \GeVc$                  & \\
    Iso muon \!+\! iso electron & $\pt^\PGm, \pt^\Pe > 23, 12 \GeVc$                  & \\
    Double muon \!+\! electron  & $\pt^{\PGm_1}, \pt^{\PGm_2}, \pt^\Pe > 9,9,9 \GeVc$ & \\
    \hline
    Single isolated muon        & $\pt^\PGm > 24 \GeVc$ & SingleMuon \\
    \bottomrule %-----------------------------------------------------------------------------------------------------------------------
  \end{tabular}
\end{table*}

They are grouped according to the kind of event selected.
Each trigger path requires the presence of one or more particles above a certain transverse momentum threshold.
Each trigger path in these groups additionally requires the event to satisfy conditions regarding particle identification from the sub-detector systems,
or isolation from other objects in a cone around the triggering particle.
The \textit{Double Electron} triggers select events containing in the final state at least two electrons.
Similarly, \textit{Triple Electron} triggers target events containing three electrons.
\textit{Double Muon} and \textit{Triple Muon} triggers target events with at least two and three muons, respectively,
while \textit{Muon Electron} triggers select events containing in the final state at least one muon and one electron.
\textit{Single Muon} triggers select events with at least one muon,
while \textit{Single Electron} triggers select events containing in the final state at least one electron.


\subsection{CMS data}
This analysis uses a data sample recorded by the CMS experiment at a centre-of-mass energy of 13\TeV during 2016, 2017 and 2018 corresponding to $\Lumi = 137 \fbinv$ of data.
Only data that passed the quality certification by all detector subsystems is used in the analysis.
The luminosity measurements have been carried out by the CMS Collaboration
with the methodology described in Reference~\cite{CMS-LUM-17-003},
for each year of data-taking~\cite{CMS-LUM-17-004, CMS-LUM-18-002}.

The samples used correspond to the so called ``UltraLegacy reprocessing'',
which contains the most recent calibrations and identification criteria,
as well as scale factors and uncertainties.
The simulations for 2016 are split into ``preVFP'' and ``postVFP'' to model the effect
of an inefficiency in a readout chip in the tracker, explained in Section~\ref{sec:APV}, which was resolved during data taking.
%% The MINIAOD format is chosen to perform the analysis.

The analysis relies on five different Primary Datasets (PD),
{\it DoubleEG}, {\it DoubleMu}, {\it MuonEG}, {\it SingleElectron}, and {\it SingleMuon},
each of which combines a certain collections of HLT paths, whose exact requirements depend on the year of data
taking. {\it DoubleEG} and {\it SingleElectron} are merged into {\it EGamma} in 2018.
To avoid duplicate events from different primary datasets, events are taken:

\begin{itemize}
\item from DoubleEG if they pass a Double Electron %(\texttt{HLT\_EleXX\_EleYY\_CaloIdXX\_TrackIdXX\_IsoXX(\_DZ)} )
  or Triple Electron trigger; %(\texttt{HLT\_EleXX\_EleYY\_EleZZ\_CaloIdXX\_TrackIdXX}) where XX, YY and ZZ are year-dependent thresholds
\item from DoubleMuon if they pass a Double Muon %(\texttt{HLT\_MuXX\_TrkIsoVVL\_MuYY\_TrkIsoVVL})
  or Triple Muon trigger %(\texttt{HLT\_TripleMu\_XX\_YY\_ZZ})
  and fail the Double Electron and Triple Electron triggers;
\item from MuonEG if they pass a Muon Electron, %(\texttt{HLT\_MuXX\_TrkIsoXX\_EleYY\_CaloIdYY\_TrackIdYY\_IsoYY})
  Muon + Double Electron %(\texttt{HLT\_MuXX\_DiEleYY\_CaloIdYY\_TrackIdYY})
  or Double Muon + Electron trigger %(\texttt{HLT\_DiMuXX\_EleYY\_CaloIdYY\_TrackIdYY})
  and fail the Double Electron, Triple Electron, Double Muon and Triple Muon triggers;
\item from SingleElectron if they pass a Single Electron trigger %(\texttt{HLT\_EleXX\_etaXX\_WPLoose/Tight(\_Gsf)})
  and fail all the above triggers;
\item from SingleMuon if they pass a Single Muon trigger %(\texttt{HLT\_IsoMuXX OR HLT\_IsoTkMuXX})
  and fail all the above triggers.
\end{itemize}

The used data sets are listed in Table~\ref{tab:datasamples}.%, along with the integrated luminosity.

\begin{table}
  \caption{List of data samples used in the analysis. All of the runs for each of the 5 datasets (4 for 2018) are used, for a total of 86 primary datasets.}
  \label{tab:datasamples}
  \centering
  \small
  \begin{tabular}{l l l}
    \toprule
    \textbf{Year} & \textbf{Data streams} &  \textbf{Run and reconstruction version}\\
    \midrule
    2016 &
    \begin{tabular}{@{}l}
      DoubleMuon\\
      DoubleEG\\
      MuonEG\\
      SingleMuon\\
      SingleElectron
    \end{tabular}&
    \begin{tabular}{@{}l}
      Run2016B-ver1\_HIPM\_UL2016\\
      Run2016B-ver2\_HIPM\_UL2016\\
      Run2016C-HIPM\_UL2016\\
      Run2016D-HIPM\_UL2016\\
      Run2016E-HIPM\_UL2016\\
      Run2016F-HIPM\_UL2016\\
      Run2016F-UL2016\\
      Run2016G-UL2016\\
      Run2016H-UL2016
    \end{tabular} \\
    \hline
    \multirow{5}{*}{2017}
      & DoubleMuon     & Run2017B-UL2017\\
      & DoubleEG       & Run2017C-UL2017\\
      & MuonEG         & Run2017D-UL2017\\
      & SingleMuon     & Run2017E-UL2017\\
      & SingleElectron & Run2017F-UL2017\\
    \hline
    \multirow{4}{*}{2018}
      & DoubleMuon & Run2018A-UL2018\\
      & MuonEG     & Run2018B-UL2018\\
      & SingleMuon & Run2018C-UL2018\\
      & EGamma     & Run2018D-UL2018\\
    \bottomrule
  \end{tabular}
\end{table}

\subsection{Simulation}
\label{sec:simulation}
The event generator \MGvATNLO~2.6.2~\cite{MGatNLO, Frederix_2018} is used to simulate the signal and most of the background contributions.
The simulation of $\PQq \PQq / \PQq \Pg \to \PZ \PZ \to 4 \Pl$ is done with \POWHEG~\cite{Nason:2004rx, Frixione:2007vw, Alioli:2010xd, Alioli:2008gx},
while $\Pg \Pg \to \PZ \PZ \to 4 \Pl$ is simulated with \MCFM~\cite{MCFM}.
The simulation of the hadronization and parton shower is done by coupling the matrix element generators with \PYTHIA~8~\cite{Sjostrand:2015, bierlich2022comprehensive} using the \textsc{CP5}~tune~\cite{CMS-GEN-17-001}.
The interaction of the particles with the CMS detector is simulated with \GEANTfour~\cite{GEANT}.
All samples are generated with the NNPDF 3.0 (in 2016) or 3.1 (2017--18) parton distribution functions (PDFs)~\cite{NNPDF2015}.
The simulations include additional interactions in the same and neighbouring bunch crossings, referred to as \pileup{}.
The MC samples are reweighed based on the per-event true number of interactions to match the level of \pileup{} observed in data for each year.

The full list of MC samples and their cross sections are shown in Table~\ref{tab:listofsamples}.
%% It includes the ones detailed above as well as rare SM backgrounds that result in much smaller contributions to the signal regions.
All cross sections used in the analysis are those returned by the generator and reported in Table~\ref{tab:listofsamples}.

The MC samples are employed to optimize the event selection, evaluate the signal efficiency and acceptance and cross check the data-driven estimate of backgrounds.

\subsubsection{Signal}
The signal for this analysis is the production of a photon and two massive vector bosons, one of which is a $\Z$ boson that decays leptonically.
The hard process of the signal is simulated with \MGvATNLO up to an additional jet at Next-to-Leading Order (NLO) with FxFx merging~\cite{Frederix_2012}.

The sample for the fully leptonic $\PZ\PZ\PGg \to 4\Pl\,\PGg$ is
generated without forcing the intermediate vector boson resonances (\ie \verb|p p > l+ l- l+ l- a|),
so as to retain off-shell effects and spin correlations among the leptons in the final state.
For the three lepton channel, $\PW\PZ\PGg \to 3\Pl\,\PGn\,\PGg$ is
generated with a hybrid syntax (\verb|p p > l nu z a|), in which the \PZ resonance is explicit.
For the semileptonic signal samples ($\PZ\mathrm{V}\PGg \to 2\Pl\,2j\,\PGg$, $\mathrm{V} = W,\, Z$) an intermediate syntax is used,
with off-shell contributions for the leptonic decay of the \PZ,
while forcing the intermediate resonance for the hadronically decaying boson $\mathrm{V}$ (\eg \verb|p p > z l+ l- a|).

Tau leptons are included in the generation, but are not part of the signal definition, and are suppressed in the analysis by the kinematic requirements on the Z mass.
No additional studies were conducted on the contamination of taus into the final state.
The decays of vector bosons are performed by \MADSPIN~\cite{MadSpin}, in order to preserve the spin correlations between the leptons and, to some extent, off-shell effects.
The motivation of using the decay chain syntax is twofold, as it allows to speed up the generation and to populate the phase-space probed by the analysis.
The latter is crucial to ensure sufficient statistics.

\subsubsection{Background}
The dominant background for the four charged lepton final state is the production of $\PZ\PZ$,
with minor contributions from massive triboson production.
In the three lepton channel the major background contributions are Drell-Yan processes, $\PZ+\PGg$ production and $\PW\PZ$ production,
with minor contribution from $\PZ\PZ$ production and $\PQt\PAQt$.
The dominant backgrounds in the two lepton channel are Drell-Yan processes and $\PZ+\PGg$ production,
with contributions from $\PW$+jets and $\PW+\PGg$.

The process $\PQq\PAQq \to \PZ\PZ \to 4\Pl$, with \Pl = \Pe, \PGm, \PGt, is simulated with \POWHEG at NLO QCD (LO EW) up to one extra parton,
using dynamical QCD factorization and renormalization scales.
Although the fully differential cross section has already been computed at next-to-next-to-leading order (NNLO) \cite{Grazzini_2015},
this computation is not yet available in a Monte Carlo generator.
To account for this difference, the simulated events are reweighted by
NNLO/NLO k-factors that depend on $m_{\PZ\PZ}$.
Additional NLO electroweak corrections depending on the initial state quark flavour and kinematics
are also applied in the region $m(ZZ) > 2m_{Z}$, where they have been computed.

Aside from the dominant ZZ background mediated by the tree-level processes, there is also a gluon loop-induced ZZ production process,
which is a NNLO diagram and therefore is not included in the nominal ZZ sample.
Though suppressed by the two additional strong couplings, it nevertheless contributes to inclusive ZZ production at the 10\usep\% level.
The gluon-induced process $\Pg \Pg \to \PZ\PZ$ is simulated, separately for the three final states 4\Pe, 2\Pe\PGm and 4\PGm, at LO with \MCFM~7.0~\cite{Campbell_2014}.
An exact calculation at NLO of this process is not yet available,
but it was shown~\cite{Bonvini_2013} that its cross section is well described
in the soft collinear approximation.
Similarly to $\PQq\PAQq \to \PZ\PZ \to 4\Pl$, k-factors are applied as function of $m(\PZ\PZ)$~\cite{Catani_2007,Melnikov_2015}.
Additionally, the cross-section is scaled by a factor $1.7$.

The diboson production $\PW\PZ \to 3\Pl \PGn$, $\PZ \PGg \to 2\Pl \PGg$ and the Drell-Yan process $\PZ \to \Pl \Pl$, %with \Pl = \Pe, \PGm, \PGt,
are simulated at NLO with \MGvATNLO, including off-shell effects.
For the first two, up to one additional parton is included in matrix element calculation,
while for Drell-Yan the calculations include up to two additional partons.

The production of a top-antitop pair $\PQt\PAQt \to 2\Pl 2\PGnl + \text{jets}$, %with \Pl = \Pe, \PGm, \PGt,
is simulated using \POWHEG
at NLO QCD (LO EW) up to 2 additional partons.

The associated production of top, antitop and a vector boson $\PQt\PAQt\PW$ and $\PQt\PAQt\PZ$,
with leptonic decays of the vector boson, were simulated using \MGvATNLO,
while $\PQt\PAQt\PZ\PZ$ and $\PQt\PAQt\PW\PW$ were simulated with inclusive decays at LO using \MADGRAPH.

Massive triboson production in the channels $\PZ\PZ\PZ$, $\PW\PZ\PZ$, $\PW\PW\PZ$ and $\PW\PW\PW$ was simulated
with inclusive decays of the vector bosons, requiring $m_{jj} < 100 \GeV$
using \MGvATNLO at NLO QCD (LO EW).

The subsequent decays of the \PZ and \PW bosons to electrons or muons are performed in \MADSPIN, in order to preserve the spin correlations between the leptons.
%% The FXFX merging scale $q_{cut} = 30\unit{GeV}$ as well as the minimum jet \pt cut $p_{T}^{jet}>15\unit{GeV}$ are identical to the values in the 0,1 jet FXFX sample.

%Leptons are generated requiring $m_{\Pl^{+}\Pl^{-}}> 4 \GeV$ in all samples but \texttt{MadGraph}, in which  $m_{\Pl^{+}\Pl^{-}} > 12 \GeV$,
%and reconstructed  following the same steps of~\cite{HiggsLegacyPaper,ZZXSPaper}.
%Jets are generated with $\pt > 10 \GeV$ and reconstructed following the criteria recommended by Jet-MET group~\cite{JetID}.

\begin{table}
  \caption{List of signal and background samples used in the analysis, with the matrix element generator used and their cross section.}
  \label{tab:listofsamples}
  \centering
  %% \resizebox{.8\textwidth}{!}{
    \begin{tabular}{l l r m{.3\textwidth}}
    \toprule
    Process & Generator & $\sigma$ [pb] & Remarks\\

    \midrule
    \multicolumn{4}{l}{Signal samples}\\
    \hline
    $\PZ\PZ\PGg \to 4\Pl\,\PGg$      & MadGraph (NLO) & 0.02202  &\\ %pp>l+ l- l+ l-a [QCD]; lep = e mu ta
    $\PW\PZ\PGg \to 3\Pl\,\PGn\,\PGg$& MadGraph (NLO) & 0.03844  &\\ %pp>lep nu z a [QCD]; lep = e mu ta
    $\PZ\PZ\PGg \to 2\Pl\,2j\,\PGg$  & MadGraph (NLO) & 0.04978  &\\
    $\PW\PZ\PGg \to 2\Pl\,2j\,\PGg$  & MadGraph (NLO) & 0.08044  &\\

    \midrule
    \multicolumn{4}{l}{Main background samples}\\
    \hline
    $\PZ\PZ\to 4\Pl$ + 0,1 jets      & MadGraph (NLO) & 1.256    &\\
    $\Pg\Pg\to \PZ\PZ\to 4\PGm$      & MCFM (LO)      & 0.001586 &\\
    $\Pg\Pg\to \PZ\PZ\to 4\Pe$       & MCFM (LO)      & 0.001586 &\\
    $\Pg\Pg\to \PZ\PZ\to 2\Pe\,2\PGm$& MCFM (LO)      & 0.003191 &\\

    $\PZ$ + 0,1,2 jets               & MadGraph (NLO) & 6225.2   & off-shell contributions\\% define lep+ = e+ mu+ ta+; generate p p > lep+ lep- j [QCD]; add process p p > lep+ lep- j j [QCD]
    $\PZ\PGg$ + 0,1 jets             & MadGraph (NLO) & 55.48    & off-shell contributions\\% define lep = e+ mu+ ta+ e- mu- ta-; generate p p > lep lep a [QCD]; add process p p > lep lep a j [QCD]
    $\PW\PZ \to 3\Pl\,\PGn$          & MadGraph (NLO) & 5.213    & off-shell contributions\\ % define p = p b b~; define j = j b b~; define ell+ = e+ mu+ ta+; generate p p > ell- vl~ ell+ ell- (+ h.c.); add process p p > ell- vl~ ell+ ell- j (+ h.c)
    $\PQt\PAQt \to 2\Pl\,\PGn$ + jets& Powheg         & 87.3     &\\

    \midrule
    \multicolumn{4}{l}{Rare background samples}\\
    \hline
    $\PQt\PAQt\PZ$ + 0,1 jets        & MadGraph (NLO) & 0.5407   & $\PZ \to 2\Pe, 2\PGm, 2\PGt, 2\PGn$\\ %define l+ = e+ mu+ ta+; generate p p > t t~ l+ l- / h [QCD] @0; add process p p > t t~ vl vl~ / h [QCD] @1; decay t > w+ b, w+ > all all
    $\PQt\PAQt\PW$ + 0,1 jets        & MadGraph (NLO) & 0.2161   & leptonic decay of the \PW\\ %define ell+ = e+ mu+ ta+; generate p p > t t~ ell+ vl [QCD] +h.c.; add process p p > t t~ ell+ vl j +h.c.; decay t > w+ b, w+ > all all +h.c.
    $\PQt\PAQt\PZ\PZ$                & MadGraph (LO)  & 0.001572 & inclusive decays\\ %generate p p > t t~ z z
    $\PQt\PAQt\PW\PW$                & MadGraph (LO)  & 0.007883 & inclusive decays\\ %generate p p > t t~ w+ w-
    $\PZ\PZ\PZ$                      & MadGraph (NLO) & 0.01398  & inclusive decays\\ %generate p p > z z z [QCD]
    $\PW\PZ\PZ$                      & MadGraph (NLO) & 0.05565  & inclusive decays\\ %generate p p > w z z [QCD]
    $\PW\PW\PZ$                      & MadGraph (NLO) & 0.1651   & inclusive decays\\ %generate p p > w w z $$ t t~ [QCD]
    $\PW\PW\PW$                      & MadGraph (NLO) & 0.08058  & inclusive decays\\ %define l+ = e+ mu+; h.c.; define vl = ve vm vt; h.c.; define w = w+ w-; generate p p > w w w $$ t t~ [QCD]
    \bottomrule
  \end{tabular}
  %% }
\end{table}

\subsubsection{Overlap between samples}
\label{sec:samples_overlap}
Among many other operations, the parton shower also simulates the final state radiation
of photons from charged leptons in the final state produced by the matrix element generator.
In particular, for the three samples $\PQq\PAQq \to \PZ\PZ$, $\PW\PZ \to 3\Pl\PGnl$ and $\PZ$+jets
this results in a partial overlap with $\PZ\PZ\PGg$, $\PW\PZ\PGg$ and $\PZ\PGg$ respectively.
Indeed, for the second group of samples the matrix element generator configuration
does not explicitly forbid that the photon is attached to a final-state lepton.

To resolve this ambiguity, events of the sample $\PQq\PAQq \to \PZ\PZ$ that contain
a generated prompt photon with $\pt^\PGg > 20\GeV$ and $|\eta^\PGg| < 2.4$ are removed.
The same procedure is applied to $\PW\PZ \to 3\Pl\PGnl$ and $\PZ$+jets.
Additionally, events in the $\PZ\PGg$ sample are required to have at least one photon
passing the aforementioned requirements.

On the contrary, the samples $\Pg\Pg\to\PZ\PZ\to4\Pl$ do not overlap with the signal sample $\PZ\PZ\PGg$,
and their contribution must be included.
As with $\PQq\PAQq\to\PZ\PZ$, they are considered part of the signal for the measurement of the inclusive production of $4\Pl\PGg$,
and a background for the search for triboson production.
