\label{sec:FSR_cut}
Although the cut on the minimum $\DR$ between the photon and any of the signal leptons greatly reduces the FSR component,
a non negligible contribution from large angle emission remains.
To address that, a cut on the invariant mass of the Z boson(s) is devised.
The emission of a high-energy photon with a large angle alters significantly the four-momentum of a lepton,
therefore the invariant mass of the Z boson reconstructed without that photon will be shifted to lower values.
For this reason the cut requires
the invariant mass of the reconstructed Z to be $m_{\Pl \Pl} > 81 \GeV$.

\note{Alternative cut:\\
  that the invariant mass of the two leptons is closer to the Z peak
  than the mass of the two lepton plus photon system, namely:
  $|m_{\Pl\Pl} - m_{\PZ}| < |m_{\Pl\Pl\PGg} - m_{\PZ}|$.
  In the 4\Pl channel the Z boson used to evaluate this cut is the one that contains the lepton closest to the signal photon.
  This removes events in which the photon is very likely FSR from one of the leptons.
}

The application of this cut on top of the previously described selections that are part of the signal definition
defines the \textbf{triboson fiducial region}.
