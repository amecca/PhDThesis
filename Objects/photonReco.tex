The PF reconstruction of isolated photons is conducted together with electron reconstruction.
This is because the large amount of material in the tracker makes electron emit bremsstrahlung photons, which may convert to \Pep \Pem pairs,
which may in turn produce bremsstrahlung, and so on.

Photon candidates are seeded by ECAL superclusters (SC) with $\ET > 10 \GeV$ with no link to a GSF track.
As for the electron candidates, the energy sum of the HCAL cells within $\DR = 0.15$ must not exceed 10 \% of the supercluster energy.
All ECAL clusters linked to the supercluster are associated with the candidate.

To correct for the energy missed in the association process,
a correction is applied to the total energy of the ECAL clusters with analytical functions of the energy and pseudorapidity, which can be as large as 25 \% at low \pt and maximum tracker thickness ($|\eta| \approx 1.5$).
The direction of the photon is taken to be that of the supercluster.

Photon candidates are retained if they are isolated from other tracks and calorimeter clusters in the event,
and if the ECAL cell energy distribution and the ratio between the HCAL and ECAL energies are compatible with those expected from a photon shower.
The PF selection is looser than the requirements applied at analysis level to select isolated photons.
