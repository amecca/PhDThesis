%\textbf{FIXME: ReRecoed data are used but additional e-scale and smearing corrections NOT yet available and thus NOT yet applied}
Electrons in data are corrected for features in ECAL energy scale
in bins of $\pt$ and $\left| \eta \right|$. Corrections are calculated
on a $\cPZ \to \Pe\Pe$ sample to align the dielectron   
mass spectrum in the data to that in the MC, and to
minimize its width.

The $\cPZ \to \Pe\Pe$ mass resolution in Monte Carlo is made to match
data by applying a pseudo-random Gaussian smearing to electron energies,
with Gaussian parameters varying in bins of $\pt$ and $\left| \eta \right|$.
This has the effect of convoluting the electron energy spectrum with a
Gaussian.

The electron energy scale is measured in data by fitting a Crystal-ball function to the di-electron mass spectrum around the Z peak in the $Z\rightarrow \Pe \Pe$ control region. Results of this procedure year per year can be found in
\todo{cite $HiggsAN,CMS_AN_2016-442,CMS_AN_2017-342$}.
% The energy scale for the 2016, 2017 and 2018 dataset are shown in Fig.~\ref{fig:ele_energy_scaleA},~\ref{fig:ele_energy_scaleB},~\ref{fig:ele_energy_scaleC} (a), respectively, and decently agrees with the MC with the preliminary corrections released so far by EGAMMA POG. % for Moriond. % even without any corrections available at the moment. 

