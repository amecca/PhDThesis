As quarks and gluons have a net colour charge and cannot exist as free particles due to colour-confinement, they can not be observed directly.
After they are produced in the collision, they combine with quarks and anti-quarks spontaneously created from the vacuum
to form a stable configuration of colour-neutral hadrons along the direction of the initial parton.
This happens in a very small timescale, of the order of $10^{-15}$ s, while the involved quarks and gluons are still inside the beam pipe,
and for this reason free quarks and gluons are never observed directly.
The result of this process, called hadronization, is a cone of collimated particles known as a jet.

\subsubsection[The anti-kt clustering algorithm]{The \antikt clustering algorithm} %{Jet Reconstruction}

%% Jets are the experimental signatures of quarks and gluons produced in high-energy processes such as head-on proton-proton collisions.
They are reconstructed using the \antikt clustering algorithm \cite{Cacciari:2008gp} on all the PF candidates.
This algorithm defines the distances $d_{ij}$ between the entities i and j and $d_{iB}$ between object $i$ and the beam (B):
\begin{equation}
\begin{split}
d_{ij} &= min \left( k^{2p}_{T,i}, k^{2p}_{T,j} \right) \frac{\DR_{ij}}{D^2}\\
d_{iB} &= k^{2p}_{T,i}
\end{split}
\end{equation}
where $k_{T, i}$ is the transverse momentum of the object $i$ and $D$ is the distance parameter that can be adjusted.

The clustering proceeds by identifying the smallest of the distances and if it is a $d_{ij}$ recombining entities i and j,
while if it is $d_{iB}$ calling $i$ a jet and removing it from the list of entities.
The distances are recalculated and the procedure repeated until no entities are left.

In particular, \antikt sets $p = -1$, which results in an infrared and collinear safe procedure which is resilient to soft radiation,
meaning that soft emissions do not result in irregularities in the boundaries of the resulting jets.

In this analysis, two possible values of the distance parameter of \antikt are considered: $R = 0.4$ and $R = 0.8$.
Jets are reconstructed from the Particle Flow candidates after removing charged hadrons that are associated with a \pileup{} primary vertex.
