%More details on muon reconstruction can be found in Ref.~\cite{AN-15-277}.
Muon candidates are selected among global muons and tracker muons.
Standalone muon tracks that are only reconstructed in the muon system are not used.

Muons are required to have transverse momentum $\pt > 5 \GeV$ and pseudorapidity $|\eta| < 2.4$.
As for the electrons, the muon track is required to be compatible with the primary vertex,
to suppress contributions from in-flight decays of hadrons, pileup and cosmic rays.
This requirement translates into $d_{xy} < 0.5 \cm$, $d_z < 1 \cm$, where $d_{xy}$ and $d_z$ are
the muon impact parameters with respect to the primary vertex in the transverse plane and longitudinal direction respectively.
These requirements correspond to the analysis definition of \textbf{loose muons}.

Loose muons with \pt below 200\GeV that also pass
the PF loose muon ID \cite{ParticleFlow} are considered \textbf{identified muons} for this analysis.
Muons with \pt $>$ 200\GeV must pass either the PF identification or the Tracker High-\pt ID,
whose definition is reported in Table \ref{tab:highPtID}.
This relaxed definition is used to increase signal efficiency in the high
centre of mass energy regime.
In the laboratory frame of reference, the leptons coming from the decay of
a highly boosted $\cPZ$ will be nearly collinear, and the PF ID loses 
efficiency for muons separated by approximately $\Delta R < 0.4$, which roughly 
corresponds to muons originating from $\cPZ$ bosons with $\pt > 500\GeV$.

\begin{table}
    \begin{small}
    \begin{center}
    \caption{
      The requirements for a muon to pass the Tracker High-$\pt$ ID. Note that
      these are equivalent to the Muon POG High-$\pt$ ID with the global track 
      requirements removed.
      }
    \begin{tabular}{|l|l|}
      \hline
      Plain-text description         & Technical description                 \\
      \hline
      Muon station matching          & Muon is matched to segments           \\
                                     & in at least two muon stations         \\
                                     & \textbf{NB: this implies the muon is} \\
                                     & \textbf{an arbitrated tracker muon.}  \\
      \hline                                                          
      Good $\pt$ measurement         & $\pt / \sigma_{\pt} < 0.3$            \\
      \hline
      Vertex compatibility ($x-y$)   & $d_{xy} < 2$~mm                       \\
      \hline
      Vertex compatibility ($z$)     & $d_{z} < 5$~mm                        \\
      \hline
      Pixel hits                     & At least one pixel hit                \\
      \hline
      Tracker hits                   & Hits in at least six tracker layers   \\
      \hline
    \end{tabular}
    \label{tab:highPtID}
    \end{center}
    \end{small}
\end{table}

An additional \textit{ghost-cleaning} step is performed to deal with situations when a single muon
can be incorrectly reconstructed as two or more muons:

\begin{itemize}

\item Tracker Muons that are not Global Muons are required to be matched to reconstructed segments in at least two stations of the muon system;
\item If two muons are sharing 50\% or more of their segments then the muon with lower quality is removed.

\end{itemize}
