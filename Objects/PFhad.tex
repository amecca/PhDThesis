Once muons, electrons, and isolated photons are identified and removed from the PF blocks,
the remaining particles to be identified are hadrons from jet fragmentation and hadronization.
These particles may be detected as charged hadrons, neutral hadrons, nonisolated photons, and more rarely additional muons.

Within tracker acceptance, all remaining ECAL and HCAL clusters not linked to any track give rise to photons and neutral hadrons respectively.
Outside the tracker, only ECAL clusters not linked to an HCAL cluster are classified as photons.
The remaining HCAL clusters are each linked to one or more tracks.
The sum of the track momenta is then compared to the calibrated calorimetric energy in order to determine the particle content.
If the calorimetric energy is in excess by an amount larger than the expected resolution,
it is interpreted by a photon and possibly to a neutral hadron.
Each track becomes a charged hadron with momentum and energy from the track itself.
All the momenta are refitted jointly using the calorimeter deposits and the track information.

If the calorimeter energy is smaller than the track momenta by more than three standard deviations,
identified global muon tracks are masked.
If the excess persists, misreconstructed tracks with \pt uncertainty larger than 1\GeV
are sequentially masked in decreasing \pt order until the excess disappears or the PF block is empty.
