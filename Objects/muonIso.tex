A Particle Flow based isolation is used to suppress the contamination from muon from hadronic decays inside jets.
The so-called $\Delta\beta$ correction is applied in order to subtract the pileup contribution for the muons, 
whereby $\Delta\beta = \frac{1}{2} \sum^\text{charged had.}_\text{PU} \pt$ gives an estimate of the energy deposit of neutral particles (hadrons and photons) from pile-up vertices.
\todo{Add PILEUP}

The relative isolation for muons is then defined as:
\begin{equation}
\text{RelPFIso} = \frac{1}{\pt^\text{muon}} \left( \sum_\text{charged had.} \pt + \max(0, \sum_\text{neutral had.} \ET + \sum_\text{photon} \ET - \Delta \beta) \right)
\label{eqn:mupfiso}
\end{equation}

where the sums run over the photons, charged and neutral hadrons in a cone with $\DR = 0.3$ around the muon.
Only charged hadrons originating from the primary vertex are included to minimise the pileup contribution.

The isolation cone for muons was optimised and the working point was chosen to be $\text{RelPFiso}(\Delta R = 0.3) < 0.35$. 

Similarly to electrons, a condition on the significance of the 3D impact parameter (\SIPthreeD, see Equation \ref{eq:SIP3D}) is applied,
in order to ensure that muons are consistent with the primary vertex.
Muons are required to satisfy $\SIPthreeD < 4$.
