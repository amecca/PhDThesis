Two methods are used to calibrate the muon momentum scale~\cite{CMS-MUO-16-001}.
The magnitudes of the momentum scale corrections are about 0.2\usep\% and 0.3\usep\% in the barrel and endcap, respectively~\cite{CMS-MUO-16-001}.
The uncertainty in the resolution is estimated to be about 5\usep\% of its value for both techniques.

\paragraph{Rochester corrections\\}
In the first method, muon momentum scale is measured in data by fitting
the di-muon mass spectrum
in a two-step process
in $Z \rightarrow \mu\mu$ events~\cite{RochesterMuon}.
%% Three datasets are used:
%% the first is a simulation with realistic detector conditions of $\PGg/\PZ \to \PGm \PGm$ events (realistic);
%% the second is a simulation with a perfectly aligned detector (ideal);
%% the third contains data.

In the first step, the corrections are derived
for bins of charge, $\eta$ and $\phi$
by requiring that the average $<1/\pt^\mu>$
in data and MC to be the same as that in an ideal simulation with a perfectly aligned detector.
This produces a table of uncorrelated corrections.
%% This step is called $<1/\pt^\mu>$ correction.

The second step aims at correcting the residual mismodelling of detector efficiency in $\eta$, $/\phi$.
In order to be independent of the modelling assumptions
for the \pt and $\eta$ distributons of $\PZ \to \mu\mu$ events,
the reconstructed \PZ mass is required to be equal to that in the ideal MC for each bin in $\eta$ and $\phi$.
%% This step is called $\Delta M/M$ tuning.
%% Next \textit{addittive} corrections are extracted by taking the difference between the \PGmp and \PGmm scale corrections,
%% which are caused by misalignments,
%% and \textit{multiplicative} corrections
%% which are caused by mismodelling of the magnetic field.

\paragraph{Corrections with Kalman filter\\}
The second method uses \PJGy and \PGU(1S) events~\cite{CMS-PAS-SMP-14-007}.
The effect of variations of the magnetic field, misalignment, and mismodelling of the material
on the measured muon curvature are calculated.
Their values are extracted by fitting the dimuon mass with a Kalman filter
in several bins of pseudorapidity, for a total of 44 parameters.
