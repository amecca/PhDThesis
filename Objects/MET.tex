\label{sec:MET}
Some particles, such as neutrinos and those predicted in several BSM scenarios, can escape detection since they barely interact with matter.
By exploiting momentum conservation in the transverse plane, their presence can be inferred by an imbalance in the vectorial sum of the momenta of the other particles:
\begin{equation}
  \ptmiss \mathdefined -\sum_{measured} \vec{\pt}
\end{equation}
with the sum running on all the PF particles.

The measurement of \ptmiss relies on the hermetic detector coverage granted by the calorimeters, in particular HE and HF.
Given its indirect nature, it has a large uncertainty.
In rare cases, an artificially large \ptmiss may be caused by an error in the reconstruction. %the misidentification or misreconstruction of a particle momentum.
The most common reasons are the presence of a cosmic-ray muon,
a misreconstruction of the momentum of a genuine muon,
or particle misidentification, for example an energetic charged hadron reconstructed as a muon.
All these cases are considered by a dedicated post-processing step in PF,
which is able to correct these mishaps in the large majority of these events.
