Isolated global muons are first selected by requiring the
sum of the \pt of the tracks and \ET of calorimeter deposits
within a cone of radius $\DR = 0.3$ to be smaller than 10\usep\% of the muon \pt.

For non-isolated muons, such as those inside jets, the tight-muon selection~\cite{CMS-MUO-10-004} is applied.
It requires that a muon:
is a global muon with $\chi^2/dof < 10$ on its global track fit;
its tracker track must be matched to muon segments in at least two stations,
must use more than 10 hits in the inner tracker,
of which at least one from the Pixel,
and have a $d_{xy} < 2~\text{mm}$ to the primary vertex.
%% This selection suppresses muons from in-flight decays.
Additionally, the muon must either have at least three matched segments in the muon detectors,
or its associated calorimeter deposits must be compatible with the muon hypothesis.

Muons that fail the tight-muon selection are recovered if the standalone track is of high quality
and has a sufficiently large number of hits in the muon detectors.
% (at least 23 DT or 15 CSC hits, out of 32 and 24, respectively)
Alternatively, if a high-quality tracker-only fit with at least 13 hits is obtained,
the muon is selected, provided that the associated calorimeter deposits are compatible with the muon hypothesis.
