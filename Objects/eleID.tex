Electron candidates are preselected using loose cuts on track-cluster matching observables, so as to preserve the highest possible efficiency while rejecting part of the QCD background. To be considered for the analysis, electrons are required to have a
transverse momentum $p^e_T >$ 7 GeV,
a reconstructed pseudorapidity $|\eta^e| <$ 2.5.
They must satisfy a loose primary vertex
constraint defined as $d_{xy} < 0.5$ cm and $d_z < 1\cm$,
in order to suppress electrons from photon conversions, since their tracks do not point to the primary vertex.
Such electrons are called {\bf loose electrons}.

Additionally,
electrons that pass the cut on the \SIPthreeD and the multivariate identification described in
Sections~\ref{sec:eleSIP} and \ref{sec:eleMVA} respectively
are defined {\bf tight electrons}.
This is the selection used by the analysis.

The data-MC discrepancy is corrected using scale factors as is done for the electron selection with data efficiencies measured using the same tag-and-probe technique outlined later (see Section~\ref{sec:eleEffMeas}).
These studies for reconstructions are carried out by the CMS Collaboration and the results are summarized here.
The electron reconstruction scale factors
% are shown Fig.~\ref{fig:ele_rec_scale_factors} and
are applied as a function of the super cluster $\eta$ and electron $\pt$.
