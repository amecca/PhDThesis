%More details on muon reconstruction can be found in Ref.~\cite{AN-15-277}.
The analysis definition of {\bf loose muons} requires
$p_T > 5$, $|\eta| < 2.4$, $dxy< 0.5$ cm, $dz < 1$ cm, where $dxy$ and $dz$ are 
defined w.r.t. the PV and using the 'muonBestTrack'. Muons have to be 
reconstructed by either the Global Muon or Tracker Muon algorithm. Standalone 
Muon tracks that are only reconstructed in the muon system are rejected.
Muons with \verb|muonBestTrackType==2| (standalone) are discarded even if they 
are marked as global or tracker muons. 

Loose muons with $\pt$ below 200\GeV that also pass
%Muon BDT (see below).
the PF loose muon ID are considered {\bf tight muons} for this analysis.
Note that the naming convention used for these IDs differs from the muon POG
naming scheme, in which the ``tight ID'' used here is called the ``loose ID''.

Those with $\pt$ above 200\GeV are considered tight if they pass
either the PF ID or the Tracker
High-$\pt$ ID, the definition of which is shown in Table~\ref{tab:highPtID}.
This relaxed definition is used to increase signal efficiency in the high
centre of mass energy regime.
% This relaxed definition is used to increase signal efficiency for the high-mass
% search. When a very heavy resonance decays to two $\cPZ$ bosons, both bosons
% will be very boosted.
In the lab frame, the leptons coming from the decay of
a highly boosted $\cPZ$ will be nearly collinear, and the PF ID loses 
efficiency for muons separated by approximately $\Delta R < 0.4$, which roughly 
corresponds to muons originating from $\cPZ$ bosons with $\pt > 500\GeV$.

\begin{table}[ht]
    \begin{small}
    \begin{center}
    \caption{
      The requirements for a muon to pass the Tracker High-$\pt$ ID. Note that
      these are equivalent to the Muon POG High-$\pt$ ID with the global track 
      requirements removed.
      }
    \begin{tabular}{|l|l|}
      \hline
      Plain-text description         & Technical description                 \\
      \hline
      Muon station matching          & Muon is matched to segments           \\
                                     & in at least two muon stations         \\
                                     & \textbf{NB: this implies the muon is} \\
                                     & \textbf{an arbitrated tracker muon.}  \\
      \hline                                                          
      Good $\pt$ measurement         & $\frac{\pt}{\sigma_{\pt}} < 0.3$      \\
      \hline
      Vertex compatibility ($x-y$)   & $d_{xy} < 2$~mm                       \\
      \hline
      Vertex compatibility ($z$)     & $d_{z} < 5$~mm                        \\
      \hline
      Pixel hits                     & At least one pixel hit                \\
      \hline
      Tracker hits                   & Hits in at least six tracker layers   \\
      \hline
    \end{tabular}
    \label{tab:highPtID}
    \end{center}
    \end{small}
\end{table}

An additional ``ghost-cleaning'' step is performed to deal with situations when a single muon
can be incorrectly reconstructed as two or more muons:

\begin{itemize}

\item Tracker Muons that are not Global Muons are required to be arbitrated.
\item If two muons are sharing 50\% or more of their segments then the muon with lower quality is removed.

\end{itemize}
