Muons are reconstructed in the CMS detector with high efficiency and purity,
thanks to the clear signature they leave in the muon spectrometer and in the inner tracking system.
The purity is granted by the upstream calorimeters and the steel return yoke that absorb other particles (except neutrinos),
while the inner tracker provides a precise measurement of the muon momentum.
Muon physics objects are reconstructed with dedicated algorithms combining information from different subsystems.
The final collection is composed by three different muon types:

\begin{itemize}
\item Standalone muons, built from the information provided by the outer Muon System.
      One or more segments, each built from hits in a single DT or CSC chamber are combined
      with RPC hits and fitted to build a standalone-muon track.
\item Tracker muons, built by propagating tracks from the inner tracker outward,
      requiring a match with at least one segment made of hits in the DT or CSC.
      The high probability that a tracker muon to have one single matched segment in the muon system
      makes this algorithm very efficient at low momentum ($\PT < 5 \GeV$).
\item Global muons, built by propagating standalone-muon tracks inward to the inner tracker.
      In case of match, the hits from the two different tracks are fitted jointly into a global-muon track.
\end{itemize}

Global and tracker muons that share the same inner track are merged into a single muon object.
The charge and momentum are extracted from the tracker track for muons of \PT $< 200$ \GeV,
since multiple scattering limits the precision of the muon system at low momentum.
Above that threshold, charge and momentum are extracted from the a fit to hits that can come from the entire tracking system.
\todo{maybe describe the various global refits}

The efficiency of muon reconstruction is very high, around 99 \% within the detector acceptance,
thanks to the high efficiency of both the inner track and muon system reconstruction.
