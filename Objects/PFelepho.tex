Electron and photon identification are conducted together.
The former are seeded as described in Section~\ref{sec:eleReco},
while the latter are seeded by ECAL superclusters not linked to any tracks and $\ET \geq 10 \GeV$.
The sum of HCAL energy within $\DR < 0.15$ of the SC centre must not exceed 10\usep\% of the SC energy.
All the clusters linked to the SC or to a tangent of the GSF track are collected.
Tracks linked to these clusters are also associated if their momentum and HCAL clusters are compatible with the electron hypothesis.

The total energy of the ECAL clusters is corrected with analytical function of the energy and pseudorapidity to account for missing association,
up to 25\usep\% at $\abs{\eta} \simeq 1.5$ and low \pt.
This energy is assigned to photon candidate; the photon direction is that of the supercluster.
The estimation of electron momentum
relies on a combination of the energy of the supercluster and the momentum estimate of the GSF track.
