\label{sec:clusters}
Clustering the energy deposits in the calorimeter is necessary to
detect and measure the energy of neutral particles (photons and hadrons),
reconstruct and identify electrons and their bremsstrahlung photons
and improve the energy resolution for charged hadrons.

\note{The following two paragraphs could be moved to the corresponding sections of Chapter 2.}

The signals in the ECAL crystals are reconstructed by fitting the signal pulse
with multiple template functions to subtract the contribution from out-of-time \pileup~\cite{CMS-EGM-18-001}.

During 2016-2017, two separate algorithms were used
to reconstruct signal amplitudes in HCAL~\cite{CMS-PRF-22-001} at HLT and offline,
eventually settling in 2018 on a third common algorithm (MAHI) for both.
The signal amplitude is extracted through an iterative $\chi^2$ fit of the signal pulse
with templates to account for out-of-time \pileup.

First, \textit{cluster seeds} are identified as crystals with local energy maximum above a certain threshold.
Then, \textit{topological clusters} are grown by aggregating crystals with at least one side in common with a crystal already in the cluster,
provided their energy exceeds a threshold defined as twice the noise level in the crystal.

An expectation-maximisation algorithm based on a Gaussian-mixture model~\cite{CMS-NOTE-2005-001}
is then used to reconstruct the \textit{PF clusters} within a topological cluster,
based on the assumption that the number of PF clusters is equal to the number of seeds.
The parameters are the amplitudes and the mean of the Gaussians, while the variance is fixed for each calorimeter.
% and are determined with a maximum likelyhood fit

Finally, the energy of each cell within a topological cluster is shared among all the associated PF clusters according to the cell-cluster distance,
with an iterative determination of the cluster energies and positions.
