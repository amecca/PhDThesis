\label{sec:clusters}
Clustering the energy deposits in the calorimeter is necessary to
detect and measure the energy of neutral stable particles (photons and hadrons),
reconstruct and identify electrons and their bremsstrahlung photons
and improve the energy resolution for charged hadrons.

First, \textit{cluster seeds} are identified as crystals with local energy maximum above a certain threshold.
Then, \textit{topological clusters} are grown by aggregating crystals with at least one one side in common with one already in the cluster,
if their energy exceeds a threshold defined as twice the noise level in the crystal.

An expectation-maximisation algorithm based on a Gaussian-mixture model is then used to reconstruct the \textit{PF clusters} within a topological cluster,
based on the assumption that the number of PF clusters is equal to the number of seeds.
The parameters are the amplitudes and the mean of the Gaussians, while the variance is fixed for each calorimeter.
% and are determined with a maximum likelyhood fit

Finally, the energy of each cell within a topological cluster is shared among all the associated PF clusters according to the cell-cluster distance,
with an iterative determination of the cluster energies and positions.
