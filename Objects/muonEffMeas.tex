Muon efficiencies are measured with the Tag and Probe (T\&P) method performed on
$\cPZ \to \Pgm\Pgm$ and $\JPsi\to\mu\mu$ events in bins of $\pt$ and $\eta$. 
% More
%details on the methodology can be found in Ref.~\cite{CMS_AN_2015-277}. Measurements are extracted using 2018 RunA,B,C,D data while the measurements corresponding to 2016 and 2017 datasets have already been reported in Ref.~\cite{CMS_AN_2016-442} and Ref.~\cite{CMS_AN_2017-342} respectively.
%
The $\Z$ sample is used to measure the muon reconstruction and identification efficiency at high $\pt$,
and the efficiency of the isolation and impact parameter requirements at all $\pt$.
%
The $\JPsi$ sample is used to measure the reconstruction efficiency at low $\pt$,
as it benefits from a better purity in that kinematic regime. In this case,
events are collected using \verb=HLT_Mu7p5_Track2_Jpsi_v*= when probing the
reconstruction and identification efficiency in the muon system, and using the
 \verb=HLT_Mu7p5_L2Mu2_Jpsi_v*= when probing the tracking efficiency.

Results for the muon reconstruction and identification efficiency for $\pt > 20\GeV$
have been derived by the Muon POG.
The probe in this measurement are tracks reconstructed in the inner tracker, and
the passing probes are those that are also reconstructed as a global or tracker muon 
and passing the Muon POG Loose muon identification.
%
Results for low \pt muons were derived using \JPsi events, with the same definitions
of probe and passing probes. The systematic uncertainties are estimated by varying the analytical signal and background shape models used to fit 
the dimuon invariant mass. 
% Details on the procedure can be found in Ref.~\cite{AN-15-277}. 
The efficiency and scale 
factors used for low $\pt$ muons are the ones derived using single muon dataset.


For the impact parameter requirements, the measurement is performed using $\Z$ events. Events are selected with \verb=HLT_IsoMu27_v*= or \verb=HLT_Mu50_v*= triggers.
For this measurement, the probe is a muon passing the POG Loose identification criteria,
and it is considered a passing probe if satisfies the SIP3D, dxy, dz cuts of this analysis.
%
The efficiency to reconstruct a muon track in the inner detector is measured using as probes tracks
reconstructed in the muon system alone. The efficiency and 
data to MC scale factors are measured from Z events as a function of $\eta$ for $\pt > 10\GeV$ and $\pt < 10\GeV$. 

%he values of data to mc scale factors 
%used are from the ReReco version of the full dataset collected in 2018. 

%The tracking efficiency in data and simulation as a function of $\eta$ is shown in Fig.~\ref{fig:MuonIDEff_4}.
%\begin{figure}[htbp]
%  \begin{center}
%    \subfigure[]{\includegraphics[width=0.42\textwidth]{Figures/Muons/Placeholder.png}}
%    \subfigure[]{\includegraphics[width=0.42\textwidth]{Figures/Muons/Placeholder.png}}
%    \caption{Tracking efficiency in data and simulation as a function of $\eta$ for muon $\pt < 10\GeV$(left) and $\pt > 10\GeV$(right) with ReReco data.}
%    \label{fig:MuonIDEff_4}
%\end{center}
%\end{figure}


\clearpage
