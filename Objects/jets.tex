As quarks and gluons have a net colour charge and cannot exist as free particles due to colour-confinement, they can not be observed directly.
After they are produced in the collision, they combine with quarks and anti-quarks spontaneously created from the vacuum
to form a stable configuration of colour-neutral hadrons along the direction of the initial parton.
This happens in a very small timescale, of the order of $10^{-15}$ s, while the involved quarks and gluons are still inside the beam pipe,
and for this reason free quarks and gluons are never observed directly.
The result of this process, called hadronization, is a cone of collimated particles known as a jet.

\subsubsection{Jet Reconstruction}

Jets are the experimental signatures of quarks and gluons produced in high-energy processes such as head-on proton-proton collisions.
They are reconstructed using the \antikt clustering algorithm \cite{Cacciari:2008gp} on PF candidates.
This algorithm defines the distances $d_{ij}$ between the entities i and j and $d_{iB}$ between object $i$ and the beam (B):
\begin{equation}
\begin{split}
d_{ij} &= min \left( k^{2p}_{T,i}, k^{2p}_{T,j} \right) \frac{\DR_{ij}}{D^2}\\
d_{iB} &= k^{2p}_{T,i}
\end{split}
\end{equation}
where $k_{T, i}$ is the transverse momentum of the object $i$ and $D$ is the distance parameter that can be adjusted.

The clustering proceeds by identifying the smallest of the distances and if it is a $d_{ij}$ recombining entities i and j,
while if it is $d_{iB}$ calling $i$ a jet and removing it from the list of entities.
The distances are recalculated and the procedure repeated until no entities are left.

In particular, \antikt sets $p = -1$, which results in and infrared and collinear safe procedure which is resilient to soft radiation,
meaning that soft emissions do not result in irregularities in the boundaries of the resulting jets.

In this analysis, two possible valued of the distance parameter of \antikt are considered: $R = 0.4$ and $R = 0.8$.
Jets are reconstructed from the Particle Flow candidates after removing charged hadrons that are associated to a pileup primary vertex.

\subsubsection{Jet Identification}
\label{sec:jet_ID}
In order to assure a good reconstruction efficiency, identification quality criteria are imposed on jets based on
the energy fraction of the charged, electromagnetic, and neutral hadronic components \cite{CMS-PAS-JME-16-003}.

To reduce instrumental background, the jet ID described in Table \ref{tab:JetID2018} is used.

\begin{table}
  \caption{Jet identification criteria used in \RunII with thresholds used for 2018 data.}
  \label{tab:JetID2018}
  \resizebox{\textwidth}{!}{
  \begin{tabular}{l c c c c}
    \toprule
    Variable                    & $|\eta| \le 2.6$ & $2.6 < |\eta| \le 2.7$ & $2.7 < |\eta| \le 3.0$ & $3.0 < |\eta| \le 5.0$\\
    \midrule
    Neutral Hadron Fraction     & < 0.90           & < 0.90                 & -                      & > 0.2                 \\
    Neutral EM Fraction         & < 0.90           & < 0.99                 & $<0.99\ \land\,>0.02$  & < 0.9                 \\
    Number of Constituents      & > 1              & -                      & -                      & -                     \\
    Muon Fraction               & < 0.80           & < 0.80                 & -                      & -                     \\
    Charged Hadron Fraction     & > 0              & -                      & -                      & -                     \\
    Charged Multiplicity        & > 0              & > 0                    & -                      & -                     \\
    Charged EM Fraction         & < 0.80           & < 0.80                 & -                      & -                     \\
    Number of Neutral Particles & -                & -                      & > 2                    & > 10                  \\
    \bottomrule
  \end{tabular}
  }
\end{table}

In this analysis, the jets are required to be within $|\eta| < 4.7$ and to have a transverse momentum above 30\GeV.

The identification of jets originated by the hadronization of heavy-flavour quarks is achieved through the DeepFlavour tagger \cite{Guest_2016, Sirunyan_2018}.
It uses a deep neural network with input variables from up to six tracks from each jet, information on secondary vertices and on the jet itself (such as the energy and total number of tracks).
It assigns five probabilities to each jet, namely to contain only one \PQb hadron, two or more \PQb hadrons, exactly one \PQc hadron, at least two \PQc and no \PQb hadrons, or none of the above.
In this analysis there is the need to exclude events containing top quarks, since their decay $\PQt \to [\PW \to \Pl\PGn] \PQb$ produces additional charged leptons which may lower the purity of the signal regions.
Thus the need to identify jets induced by the hadronization of \PQb quarks, which is achieved by summing the probabilities of the first two categories.
Working points are defined for this ID, as shown in Table \ref{tab:DeepFlavourBtagWP}.

\begin{table}
  \caption{Working points of the combined b-tag with the DeepFlavour classifier and the corresponding misidentification probability.}
  \label{tab:DeepFlavourBtagWP}
  \centering
  \begin{tabular}{l c c}
    \toprule
    \makecell{Working\\point} & Threshold & \makecell{Misidentification\\probability}\\
    \midrule
    Loose  & 0.0494 & 10  \%\\
    Medium & 0.2770 & 1   \%\\
    Tight  & 0.7264 & 0.1 \%\\
    \bottomrule
  \end{tabular}
\end{table}

In addition this analysis considers also a collection of \textbf{large radius jets} clustered using the same \antikt algorithm with a distance parameter $R = 0.8$.
These jets are cleaned using the Pileup Per Particle Identification (PUPPI) \cite{Bertolini_2014}, which is a method for pileup mitigation.

\paragraph{Jet disambiguation\\}
The PF jet collection contains jets clustered with \antikt on all the particles reconstructed from Particle Flow.
This means that isolated and energetic leptons and photons, which are part of the signal definition,
are likely to be in the core of a jet due to the nature of the algorithm.
Therefore a disambiguation strategy is needed, to avoid double counting entities.
In addition, the jets are cleaned from any of the leptons passing the Loose ID
and photons, either FSR associated to a lepton or candidates passing the kinematic selection (Section \ref{sec:photonID}),
by a separation criterion: $\Delta R(\text{jet,lepton/photon}) > 0.4$.

\subsubsection{Jet Energy Corrections}

The calorimeter response to particles is not linear and it is not straightforward to translate the measured jet energy
to the true particle or parton energy, therefore Jet Energy Corrections (JEC) must be applied.
In this analysis, jet energy corrections are applied to the reconstructed jets, which account for pileup,
non uniformity of detector response, and residual differences between the jet energy scale between data and simulation, and residual calibration for data
\cite{CMS:JEC_2011, CMS-DP-2016-020, CMS-DP-2021-033}.
The corrections, parametrised as function of \pt and $\eta$ of the jets, are applied following the standard procedure used by most of the CMS analyses.

\subsubsection{L1 prefiring}

In 2016 and 2017, the gradual timing shift of ECAL was not properly propagated to \Lone trigger primitives (TP)
resulting in a significant fraction of high eta TP being mistakenly associated to the previous bunch crossing~\cite{CMS:L1trigger_Run2}.
Since \Lone trigger rules forbid two consecutive bunch crossings to fire,
in addition to missing the trigger primitive in the correct bunch crossing,
events can self veto if a significant amount of ECAL energy is found in the region of $2<|\eta|<3$.
This effect is not described by the simulations.

A similar effect is present in the muon system, where the bunch crossing assignment of the muon candidates can be wrong due to the limited time resolution of the muon detectors.
This effect was most pronounced in 2016, but is non-zero for both 2017 and 2018.
The associated prefiring rate is stable for $\pt > 25 \GeV$ but affects the almost entire eta range.
Its magnitude varies between 0\% and a 3\%.

The probability not to prefire (see Equation~\ref{eq:L1prefiring} is calculated for each event and applied as a weight to simulation for 2016 and 2017 samples,
using a software tool developed by CMS \Lone trigger experts.
\begin{equation}
\label{eq:L1prefiring}
w^{pref} = 1 - \Probability(prefire) = \prod_{i\, \in\, \PGg,\, jets,\, \PGm} (1 - \epsilon_i^{pref}(\eta, \pt))
\end{equation}

The Figure~\ref{fig:L1Prefiring} shows the impact of the L1 pre-firing weights on the signal MC.
The impact on the normalisation of the signal is around 0.8\%.

\begin{figure}
\subfigure [L1 pre-firing weights]       {\resizebox{.5\textwidth}{!}{\includegraphics[width=.5\textwidth]{Figures/L1Prefiring_ZZGTo4LG.png}}}
\subfigure [Effect on $m_{4\ell\gamma}$] {\resizebox{.5\textwidth}{!}{\includegraphics[width=.5\textwidth]{Figures/SYS/SR4P/ZZGTo4LG_mZZGloose_L1Prefiring.png}}}
\caption{L1 pre-firing weights and effect of their application on the signal MC in the signal region with four tight leptons and one photon in 2018.
The photon is required to pass the Loose working point of the cut-based ID.}
\label{fig:L1Prefiring}
\end{figure}
