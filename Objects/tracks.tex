The reconstruction of the trajectory of charged particles with high precision at the CMS detector
is a complex task because of the large combinatorics from the high multiplicity of particles and the large number of readout channels.
Additional systematic effects having an impact on the track reconstruction may arise from possible distortions in the tracking
material, inhomegeneities in the magnetic field and misalignment of the detector components.
First local charge clusters are converted to hits using the digitised output from the data acquisition system.
The local track reconstruction output is then buffered for the global track reconstruction, which aims at identifying
hit combinations that match to possible trajectories of the charged particles present in the event.
All steps of the reconstruction are performed using an iterative pattern recognition technique, a Kalman-like fitting procedure adapted for the CMS framework in the Combinatorial Track Finder \cite{billoir.qian:simultaneous} (CTF) algorithm.
At each iteration of CTF, positional information from the hits used in the previous step is discarded and the set of requirements are gradually relaxed.
The tracking algorithm carries out three main substasks.
\begin{description}
\item[Seed finding] involving the generation of the starting points of the iterative sequence, namely pairs or triplets of hits.
\item[Pattern recognition] iteratively performs the following steps:
  \begin{enumerate}
  \item navigation: the current track parameters are used to determine which adjacent layers of the detector can be interseted by extrapolation;
  \item a search is performed within the layer for modules which are compatible with the trajectory;
  \item groups of hits are formed for each module, and a $\chi^2$ test is used to determine their compatibility with the trajectory;
  \item the trajectory is updated using the information from the hits collected in the current iteration.
  \end{enumerate}
  At each iteration, track candidates must satisfy a series of quality criteria, based on cuts on the track impact parameter significance with respect to the beamspot, the number of hits in the inner tracking system and the normalised $\chi^2$ of the track trajectory, and a maximum of 5 candidates is retained.
\item[Final fit] the best-fit value of the track parameters and the covariance matrix are determined by means of a least-squares fit.
\end{description}

% Primary Vertex
The pp collision vertices in an event are reconstructed by grouping tracks consistent with originating at a common point in the luminous region.
The candidate vertex with the largest value of summed physics-object $p^2_T$ is taken to be the primary pp interaction vertex.
The physics objects are the jets, clustered using the anti-kT jet finding algorithm \cite{Cacciari:2008gp, Cacciari:2011ma} with all the tracks assigned to candidate vertices as inputs,
and the associated missing transverse momentum, taken as the negative vector $p_T$ sum of those jets.
