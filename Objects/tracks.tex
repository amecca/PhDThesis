The reconstruction of the trajectory of charged particles with high precision at the CMS detector
is a complex task because of the large combinatorics from the high multiplicity of particles and the large number of readout channels.

First local charge clusters are converted to hits using the digitised output from the data acquisition system.
%% The local track reconstruction output is then buffered for the global track reconstruction, which aims at identifying
%% hit combinations that match to possible trajectories of the charged particles present in the event.

%% Track seeds are built from few hits compatible with a charged particle trajectory (e.g. triplets in the pixel detector).
All steps of the reconstruction are performed using an iterative pattern recognition technique,
a Kalman-like fitting procedure adapted for the CMS framework in the Combinatorial Track Finder \cite{billoir.qian:simultaneous, Speer:2005dp} (CTF) algorithm.
At each iteration of the CTF, positional information from the hits used in the previous step is discarded and the set of requirements are gradually relaxed.
The tracking algorithm carries out three main substasks.
\begin{description}
\item[Seed finding] involving the generation of the starting points of the iterative sequence, namely pairs or triplets of hits.
\item[Pattern recognition] iteratively performs the following steps:
  \begin{enumerate}
  \item navigation: the current track parameters are used to determine which adjacent layers of the detector can be interseted by extrapolation;
  \item a search is performed within the layer for modules which are compatible with the trajectory;
  \item groups of hits are formed for each module, and a $\chi^2$ test is used to determine their compatibility with the trajectory;
  \item the trajectory is updated using the information from the hits collected in the current iteration.
  \end{enumerate}
  At each iteration, track candidates must satisfy a series of quality criteria, based on cuts on the track impact parameter significance with respect to the beamspot, the number of hits in the inner tracking system and the normalised $\chi^2$ of the track trajectory, and a maximum of 5 candidates is retained.
\item[Final fit] the best-fit value of the track parameters and the covariance matrix are determined by means of a least-squares fit.
\end{description}

The first types of seeds to be processed are those that are more likely to belong to high-momentum particles with an origin close to the beam axis.
Later iterations target tracks that are missing one or more hits, have low-momentum or originate from a displaced vertex.
After the final fit, the hits of the newly built tracks are masked for the next iterations.
This reduces the combinatorial complexity of the seed finding and the pattern recognition steps,
maintaining a low misreconstruction rate while increasing tracking efficiency, expecially at low momentum,
with the the added benefit of reducing the computing time up to a factor 2 (for \RunII{} conditions).
