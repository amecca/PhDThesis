\label{sec:eleReco}
%More details on electron reconstruction can be found in Ref.~\cite{ElectronLegacy}. 
Electrons deposit most of their energy in the ECAL and produce hits in the inner tracker.
The electron reconstruction is complicated by the presence of the tracker material between the collision point and the ECAL crystals,
which causes the emission of bremsstrahlung along its trajectory, and the resulting photons have a probability of converting to electron-positron pairs.
These effects are taken into account with dedicated seeding and reconstruction algorithms,
which accounts for the changes in curvature and the spread of the energy in the calorimeter.
%% Their reconstruction algorithm combines information from the two subsystems,
%% by associating a track to an ECAL cluster and estimating the electron momentum using both pieces of information.

The energy deposited by the electron is usually spread over several ECAL crystals, so the first step is clustering the energy deposits (Section \ref{sec:clusters}).
Two approaches are used to seed the electron reconstruction.

The first, traditionally called \textit{ECAL-based} approach, starts from energetic ($\ET > 4\GeV$) ECAL clusters.
The cluster energy and position are used to infer the expected position of hits in the tracker.
Clusters are assembled into \textit{superclusters} (SC),
starting from a seed cluster and aggregating those that are presumed to come from bremsstrahlung or conversion products.

While this procedure is suited for isolated electrons, it is not fit for electrons in jets or with low \pt.
A \textit{tracker-based seeding} complements it, leveraging the large efficiency of iterative tracking for these electrons.
The standard track seeds (see Section \ref{sec:tracks}) are used to initialise the procedure.
Then the track building proceeds iteratively from the track parameters provided in each layer, modelling the electron energy loss with a Bethe-Block function.
To maintain good efficiency in the presence of bremsstrahlung, compatibility requirements between the predicted and the found hits in each layer are quite loose.
If several hits are compatible with the predicted one, different trajectory candidates are created and developed,
with a limit of five candidate trajectories for each layer.
At most one missing hit is allowed per each trajectory.
Once the hits are collected, the track parameters are estimated with a fit that uses a Gaussian Sum Filter (GSF) \cite{CMS-NOTE-2005-001} with 5 components,
instead of the Kalman Filter (KF) \cite{billoir.qian:simultaneous} used for non-electron tracks.

Tracker- and ECAL-based seeds are merged and refitted with 12 GSF components.
Three different charge estimates are inferred from the GSF track curvature,
from the curvature of the closest KF track, and from $\Delta\phi$ between the cluster and the GSF track extrapolated to the vertex.
The default charge estimation for electron candidates is taken as the majority vote of these three estimates.
For Z electrons which pass a loose cut-based selection, this gives misidentification rates at the $10^{-3}$ level in the barrel and around 2\% in the endcaps~\cite{Rembser_2019}.
