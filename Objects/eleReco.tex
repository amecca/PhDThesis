%More details on electron reconstruction can be found in Ref.~\cite{ElectronLegacy}. 

Electron candidates are preselected using loose cuts on track-cluster matching observables, so as to preserve the highest possible efficiency while rejecting part of the QCD background. To be considered for the analysis, electrons are required to have a
transverse momentum $p^e_T >$ 7 GeV, a reconstructed $|\eta^e| <$ 2.5, and to satisfy a loose primary vertex 
constraint defined as $d_{xy} < 0.5$ cm and $d_z < 1$ cm.
Such electrons are called {\bf loose electrons}.

The data-MC discrepancy is corrected using scale factors as is done for the electron selection with data efficiencies measured using the same tag-and-probe technique outlined later (see Section~\ref{sec:eleEffMeas}). 
These studies for reconstructions are carried out by the EGM POG and the results are only summarised here.

The electron reconstruction scale factors 
% are shown Fig.~\ref{fig:ele_rec_scale_factors} and 
are applied as a function of the super cluster $\eta$ and electron $\pt$.


