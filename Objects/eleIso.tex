The electron isolation is archieved through the use of the Particle Flow relative isolation,
which is defined as:
\begin{equation}
\text{RelPFiso} = (\sum_{\text{charged}} \ET + \sum^{\text{corr}}_{\text{neutral}} \ET)/\ET^{\,\Pe}
\label{eqn:elepfrelisoeqn}
\end{equation}
To mitigate the contribution from \pileup, only charged hadrons from the primary vertex are included,
while the corrected neutral component of isolation is then computed using the formula:
\begin{equation}
\label{eqn:neutralea}
  \sum^{\text{corr}}_{\text{neutral}} \ET = \text{max} \left(0,\, \sum^{\text{uncorr}}_{\text{neutral}} \ET + \sum_{\text{photons}} \ET - \ET^\text{PU} \right)
\end{equation}
The contribution from neutral \pileup{} is estimated using:
\begin{equation}
  \ET^{PU} =  \rho \times A_\text{eff}
\label{eqn:purho}
\end{equation}
where $\rho$ is the mean energy density, estimated as the median of the distribution of the transverse energy density per unit area of \pileup{} jets in the event, $\pt^\text{jet}/A_\text{jet}$,
while the effective area $A_{\text{eff}}$ is the geometric area of the isolation cone,
corrected by an $\eta$-dependent factor to account for the dependence of the \pileup{} energy density~\cite{CMS-EGM-13-001}.

%% was optimized in Ref.~\cite{AN-15-277} and the electron isolation working was
The threshold for the electron isolation,
calculated within a cone of radius $\DR = 0.3$
is chosen to be $\text{RelPFiso} < 0.35$. 
