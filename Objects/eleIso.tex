The electron isolation is archieved through the use of the Particle Flow relative isolation,
which is defined as:
\begin{equation}
\text{RelPFiso} = (\sum_{\text{charged}} \ET + \sum^{\text{corr}}_{\text{neutral}} \ET)/\ET^{\text{e}}
\label{eqn:elepfrelisoeqn}
\end{equation} 
where the corrected neutral component of isolation is then computed using the formula:
\begin{equation}
\label{eqn:neutralea}
  \sum^{\text{corr}}_{\text{neutral}} \ET = \text{max} \left( \sum^{\text{uncorr}}_{\text{neutral}} \ET - \ET^{PU},\, 0 \GeV \right)
\end{equation}
and the mean \pileup{} contribution to the isolation cone is obtained as :
\begin{equation}
  \ET^{PU} =  \rho \times A_\text{eff}
\label{eqn:purho}
\end{equation}
where $\rho$ is the mean energy density in the event and the effective area $A_{eff}$ is defined as the ratio
between the slope of the average isolation and that of $\rho$ as a function of the number of vertices.

%% was optimized in Ref.~\cite{AN-15-277} and the electron isolation working was
The threshold for the electron isolation,
calculated within a cone of radius $\DR = 0.3$
is chosen to be $\text{RelPFiso} < 0.35$. 
