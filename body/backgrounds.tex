\section{Backgrounds}

An accurate description of the background process is an essential aspect of any analysis since it affects the extraction of signal yields.
The main background sources differ between the three channels, but generally belong to the category of \textit{reducible background}.
This class contains processes which have a different final state from the signal.
Howerver, either due to additional particles produced in combination which may produce a different signature in the detector,
other collisions in the same bunch crossing (pileup),
or other errors in the reconstruction procedure,
they produce events which may erroneusly pass the selection.
Often these backrounds prove difficult to model in simulation,
and it becomes advisable to use a data-driven method for their estimation.

The other category comprises \textit{irreductible backgrouns},
which are processes that generate a final state with the same particles as the signal,
although the kinematic distributions may be different.
Usually processes in this class are estimated with simulation,
but in some cases it is possible to contrain their normalization in a control region.

\subsection{Four lepton channel}
For the 4\Pl channel the predominant background component is the production of two on-shell \PZ bosons
and a photon that is either radiated as FSR from one of the leptons
or from a misreconstructed jet (e.g. from the decay of a meson like \PGpz or \PGh).
As described in Section \ref{sec:simulation}, it can be either produced from two quarks,
or proceed from a gluon initiated loop.
Additional backgrounds such as $\PQt\PAQt\PZ$ and VVV (V = \PZ, \PW) result in very small contributions.

\subsection{Three lepton channel}
\todo{continue}
