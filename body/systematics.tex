\section{Systematic uncertainties}
The imperfect knowledge of the detector response and of the experimental conditions, as well as the limited precision of fixed-order theoretical calculations concur in increasing the uncertainty on the results.
The effects of these unknowns are accounted for as \textit{systematic uncertainties}, which are then included as nuisance parameter in the model used to perform the statistical analysis and extract the results (Section \ref{sec:statistical_analysis}).

The systematic uncertainties can have different effects.
\textit{Normalization} uncertainties affect the normalization of processes, changing the event yield.
\textit{Shape} uncertainties, in addition, modify the phisical observables (e.g. kinematics) of the events, and result in a different distribution of the event fraction in histogram bins.

Systematics can be divided, depending on their source, into \textit{theoretical} uncertainties, related to theory hypoteses and features of the numerical computations, and \textit{experimental} uncertainties, which depend on the knowledge of the detector response, running conditions, and include the data-driven estimation of backgound processes.
A detailed description of the various sources considered in this analysis follows.

\subsection{Theoretical uncertainties}
Theoretical uncertainties arise from the choice of the Parton Distribution Function (PDF) set,
the uncertainty on the strong coupling constant $\alpha_s$ and
the renormalization and factorization QCD scales, which account for the missing higher order uncertainty in finite order perturbative calculations.
All theoretical systematics act as normalization uncertainties and are correlated among data-taking periods

\subsection{Experimental uncertainties}
Experimental uncertainties have different sources.

The uncertainty on the integrated luminosity varies from 1.2 \% to 2.5 \% in the data-taking years, and the total uncertainty for Run 2 is 1.6 \%.
The luminosity measurements are carried out by experts within the collaboration according to the metodology described in Ref. \cite{CMS:LUM-17-003}, for each year of data-taking \cite{CMS:LUM-17-004, CMS:LUM-18-002}.

The uncertainty on lepton identification and reconstruction efficiency has an effect on the overall event yield ranging from 1 to 15 \% \todo{verify}.

The uncertanties coming from the data-driven estimation of fake lepton and fake photons backgrounds are also considered.
In both cases, the main contributions arise from the mismatch in background composition between the region in which the fake rate is measured and the regions where it its applied.
Additionally, in the case of the photon fake rate, the application region itself is statistically limited, and this introduces an additional uncertainty on the normalization of the fake photon process.
