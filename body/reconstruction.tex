\chapter{Reconstruction}

Collision events registered by the CMS detector are reconstructed by combining information from the different subdetectors.
This process aims to improve the identification of final state particles and the reconstruction precision of their properties.
The global reconstruction strategy is implemented through the particle-flow (PF) algorithm \cite{ParticleFlow}.
Originally applied in the ALEPH experiment at LEP, this algorithm combines raw data from CMS subsystems to reconstruct physics objects and achieving a comprehensive event reconstruction.
The resulting physics objects include charged and neutral hadrons, photons, electrons, and muons, serving as the starting inputs for all subsequent data analyses.
Additionally, PF candidates are used to construct more advanced objects such as jets and missing transverse energy.

\section{Particle Flow}
\label{sec:ParticleFlow}
The algorithm relies on precise track reconstruction and clustering techniques that efficiently distinguish between overlapping showers.
A robust linking procedure is used to combine information related to energy deposits associated with a single particle across different sub-detectors.
The PF sequence begins with identifying tracks of charged particles, followed by extrapolating these tracks to compatible calorimeter energy deposits.
To prevent double counting of leptons as individual particles and jet constituents, isolated leptons are excluded from the jet clustering sequence.

Neutral hadrons and photons are identified through \textit{Charged Hadron Subtraction}, which removes the energy deposits that can be linked to tracks from charged hadrons.
The remaining deposits in ECAL and HCAL are attributed to photons and neutral hadrons,
provided that the cluster energies of neutral hadron candidates exceed their track momentum, accounting for detector resolution effects.

The CMS detector's exceptional performance with the PF algorithm is attributed to the high granularity of the electromagnetic calorimeter, hermeticity of the hadron calorimeter, and a large magnetic field integral.

% NOTE: the content of this chapter is taken (and adapted) from the Analysis Note

%% Electrons and muons are considered candidates for the reconstruction of ZZ final states
%% (``signal leptons'') if  their transverse momentum,
%% $\pt^\ell$, is greater than 7(5)~\GeV and their pseudorapidity,
%% $\left| \eta^\ell \right|$, is less than 2.5 (2.4) for electrons (muons).
%% The physics objects and the ZZ candidate selections used in this analysis are those of the $H \rightarrow ZZ \rightarrow 4\ell$~\cite{HiggsAN,HiggsLegacyPaper} analysis, with minor changes. Here, we just report the main features, referring to that material for control plots/tables etc.

\section{Objects}
\subsection{Tracks}
\label{sec:tracks}
The reconstruction of the trajectory of charged particles with high precision at the CMS detector
is a complex task because of the large combinatorics from the high multiplicity of particles and the large number of readout channels.
Additional systematic effects having an impact on the track reconstruction may arise from possible distortions in the tracking
material, inhomegeneities in the magnetic field and misalignment of the detector components.
First local charge clusters are converted to hits using the digitised output from the tracking system.
The local track reconstruction output is then buffered for the global track reconstruction, which aims at identifying
hit combinations that match to possible trajectories of the charged particles present in the event.
All steps of the reconstruction are performed using an iterative pattern recognition technique, a Kalman-like fitting procedure adapted for the CMS framework in the Combinatorial Track Finder \cite{billoir.qian:simultaneous} (CTF) algorithm.
At each iteration of CTF, positional information from the hits used in the previous step is discarded and the set of requirements are gradually relaxed.
The tracking algorithm carries out three main substasks.
\begin{description}
\item[seed finding] involving the generation of the starting points of the iterative sequence, namely pairs or triplets of hits.
\item[pattern recognition] iteratively performs the following steps:
  \begin{enumerate}
  \item navigation: the current track parameters are used to determine which adjacent layers of the detector can be interseted by extrapolation
  \item a search is performed within the layer for modules which are compatible with the trajectory
  \item groups of hits are formed for each module, and a $\chi^2$ test is used to determine their compatibility with the trajectory
  \item the trajectory is updated using the information from the hits collected in the current iteration
  \end{enumerate}
  at each iteration, track candidates must satisfy a series of quality criteria, based on cuts on the track impact parameter significance with respect to the beamspot, the number of hits in the inner tracking system and the normalised $\chi^2$ of the track trajectory, and a maximum of 5 candidates is retained.
\item[final fit] the best-fit value of the track parameters and the covariance matrix are determined by means of a least-squares fit.
\end{description}

% Primary Vertex
The pp collision vertices in an event are reconstructed by grouping tracks consistent with originating at a common point in the luminous region.
The candidate vertex with the largest value of summed physics-object $p^2_T$ is taken to be the primary pp interaction vertex.
The physics objects are the jets, clustered using the anti-kT jet finding algorithm \cite{Cacciari:2008gp, Cacciari:2011ma} with all the tracks assigned to candidate vertices as inputs,
and the associated missing transverse momentum, taken as the negative vector $p_T$ sum of those jets.

Electron and Muon identification criteria, energy/momentum calibrations, and efficiency measurements, as well as the Final State Radiation (FSR) recovery, follow the same strategy used in Reference \todo{cite HiggsAN}. % and CMS_AN_2016-442 CMS_AN_2017-342
\subsection{Electrons}
\subsubsection{Electron Reconstruction}
\label{sec:eleReco}
%More details on electron reconstruction can be found in Ref.~\cite{ElectronLegacy}. 

Electron candidates are preselected using loose cuts on track-cluster matching observables, so as to preserve the highest possible efficiency while rejecting part of the QCD background. To be considered for the analysis, electrons are required to have a
transverse momentum $p^e_T >$ 7 GeV, a reconstructed $|\eta^e| <$ 2.5, and to satisfy a loose primary vertex 
constraint defined as $d_{xy} < 0.5$ cm and $d_z < 1$ cm.
Such electrons are called {\bf loose electrons}.

The data-MC discrepancy is corrected using scale factors as is done for the electron selection with data efficiencies measured using the same tag-and-probe technique outlined later (see Section~\ref{sec:eleEffMeas}). 
These studies for reconstructions are carried out by the EGM POG and the results are only summarised here.

The electron reconstruction scale factors 
% are shown Fig.~\ref{fig:ele_rec_scale_factors} and 
are applied as a function of the super cluster $\eta$ and electron $\pt$.



\subsubsection{Electron Identification}
\label{sec:eleID}
Reconstructed electrons are identified and isolated by means of a Gradient Boosted Decision Tree (GBDT) multivariate classifier algorithm,
which exploits observables from the electromagnetic cluster, the matching between the cluster and the electron track, observables based exclusively on tracking measurements as well as particle flow isolation sums.
It was developed for the $\PH \to \PZ \PZ^{*} \to 4 \Pl$ analysis~\cite{CMS-PAS-HIG-19-001}
and trained separately for each data taking year on a Drell-Yan plus jets MC sample.
The classifier is trained with the e\textbf{X}treme \textbf{G}radient \textbf{Boost}ing (XGBoost) optimized distributed gradient boosting library~\cite{Chen_2016}
designed to be highly efficient, flexible and portable.

% The MVA values are userFloat("ElectronMVAEstimatorRun2Summer16ULIdIsoValues"), and the cut is done in ZZAnalysis/AnalysisStep/plugins/EleFiller.cc
The full list of observables used can be found in the Table~\ref{tab:ele_ID_input_variables}.

\begin{table}[ht]
  \caption{Overview of input variables to the identification classifier. Variables not used in the Run 2 MVA are marked with  $(\mathord{\cdot})$.}
  \label{tab:ele_ID_input_variables}
  \small
  \centering
  \begin{tabular}{c l}
    \toprule
    Observable type & Observable name \\
    \midrule
    \multirow{6}{*}{Cluster shape}
      & RMS of the energy-crystal number spectrum: $\sigma_{i\eta i\eta}$, $\sigma_{i\varphi i\varphi}$ \\
      & Super cluster width along $\eta$ and $\phi$ \\
      & Ratio of the hadronic energy behind the SC to the SC energy, $H/E$ \\
      & Circularity $(E_{5\times5} - E_{5\times1})/E_{5\times5}$ \\
      & Sum of the seed and adjacent crystal over the SC energy $R_{9}$ \\
      & For endcap training bins: energy fraction in pre-shower $E_\text{PS}/E_\text{raw}$ \\
    \hline
    \multirow{2}{*}{Track-cluster match}
      & Energy-momentum agreement $E_{tot}/p_{in}$, $E_{ele}/p_{out}$, $1/E_{tot} - 1/p_{in}$ \\
      & Position matching $\Delta\eta_{in}$, $\Delta\varphi_{in}$, $\Delta\eta_{seed}$ \\
    \hline
    \multirow{5}{*}{Tracking}
      & Fractional momentum loss $f_{brem} = 1 - p_{out}/p_{in}$ \\
      & Number of hits of the KF and GSF track $N_{KF}$, $N_{GSF}$ $(\mathord{\cdot})$ \\
      & Reduced $\chi^2$ of the KF and GSF track $\chi^{2}_{KF}$, $\chi^{2}_{\textrm{GSF}}$ \\
      & Number of expected but missing inner hits $(\mathord{\cdot})$ \\
      & Probability transform of conversion vertex fit $\chi^2$ $(\mathord{\cdot})$ \\
    \hline
    \multirow{3}{*}{Isolation}
      & Particle Flow photon isolation sum $(\mathord{\cdot})$ \\
      & Particle Flow charged hadrons isolation sum $(\mathord{\cdot})$ \\
      & Particle Flow neutral hadrons isolation sum $(\mathord{\cdot})$ \\
    \hline
    \multirow{1}{*}{For PU-resilience}
      & Mean energy density in the event: $\rho$ $(\mathord{\cdot})$ \\
    \bottomrule
  \end{tabular}
\end{table}


The model is trained on 2016, 2017, and 2018 Drell-Yan with jets MC sample for both signal and background. The separate training for three periods guarantees
optimal performance during the entire \RunII data taking period.


Tables~\ref{tab:ele_ID_WPA}, \ref{tab:ele_ID_WPB} and~\ref{tab:ele_ID_WPC} list the cuts values applied to the MVA output for 2016, 2017, 2018 training, respectively.
For 2018, the corresponding signal and background efficiencies are given as examples.
They are very similar for 2016 and 2017.

For the analysis, loose electrons have to pass this MVA identification and isolation working point.

\begin{table}
  \caption{Minimum BDT score required for passing the electron identification, for 2016 samples.}
  \label{tab:ele_ID_WPA}
  \centering
  \begin{tabular}{c c c c}
    \toprule    %----------------------------------------------------------------------------------------
    \pt range           & $|\eta| < 0.8$ & $0.8 < |\eta| < 1.479$ & $|\eta| > 1.479$ \\
    \midrule    %----------------------------------------------------------------------------------------
    $5 < \pt < 10 \GeV$ &  0.9503        &  0.9461                &  0.9387 \\
    $\pt > 10 \GeV$     &  0.3782        &  0.3587                & -0.5745 \\
    \bottomrule %----------------------------------------------------------------------------------------
  \end{tabular}
\end{table}

\begin{table}
  \caption{Minimum BDT score required for passing the electron identification, for 2017 samples.}
  \label{tab:ele_ID_WPB}
  \centering
  \begin{tabular}{c c c c}
    \toprule    %----------------------------------------------------------------------------------------
    \pt range           & $|\eta| < 0.8$ & $0.8 < |\eta| < 1.479$ & $|\eta| > 1.479$ \\
    \midrule    %----------------------------------------------------------------------------------------
    $5 < \pt < 10 \GeV$ &  0.8521        &  0.8268                &  0.8694 \\
    $\pt > 10 \GeV$     &  0.9825        &  0.9692                &  0.7935 \\
    \bottomrule %----------------------------------------------------------------------------------------
  \end{tabular}
\end{table}

\begin{table}
  \caption{Minimum BDT score required for passing the electron identification and corresponding signal and background efficiencies, for 2018 samples.}
  \label{tab:ele_ID_WPC}
  \centering
  \begin{tabular}{c c c c c}
    \toprule
    $|\eta|$ range                      & \pt range           & Cut on BDT & Signal eff. & Background eff. \\
    \midrule
    \multirow{2}{*}{$|\eta| < 0.8 $}    & $5 < \pt < 10 \GeV$ &  0.8956    &  81.0\,\%   &  4.4\,\% \\
                                        & $\pt > 10 \GeV$     &  0.0424    &  97.1\,\%   &  2.9\,\% \\
    \hline
    \multirow{2}{*}{$0.8<|\eta|<1.479$} & $5 < \pt < 10 \GeV$ &  0.9111    &  79.3\,\%   &  4.6\,\% \\
                                        & $\pt > 10 \GeV$     &  0.0047    &  96.3\,\%   &  3.6\,\% \\
    \hline
    \multirow{2}{*}{$|\eta| > 1.479$}   & $5 < \pt < 10 \GeV$ &  0.9401    & 73.0\,\%    &  3.6\,\% \\
                                        & $\pt > 10 \GeV$     & -0.6042    & 95.7\,\%    &  6.7\,\% \\
    \bottomrule
  \end{tabular}
\end{table}

\subsubsection{Electron Impact Parameter Selection}
\label{sec:eleSIP}
In order to ensure that the electron trajectories are consistent with a common primary vertex
they are required to have an associated track with a small impact parameter with respect to the event primary vertex.
The significance of the impact parameter (SIP) is used:
\begin{equation}
\label{eq:SIP3D}
\SIPthreeD \mathdefined \frac{|\rm IP_{3D}|}{\sigma_{\rm IP}} \ ,
\end{equation}
where ${\rm IP_{3D}}$ is the lepton impact parameter in three dimensions,
that is the distance with respect to the primary interaction vertex the point of closest approach,
and $\sigma_{\rm IP}$ the associated uncertainty.
Electrons for the analysis must satisfy $\SIPthreeD < 4$.

%\subsubsection{Electron Isolation Optimization}
%\label{sec:eleiso}
%The electron isolation is archieved through the use of the Particle Flow relative isolation,
which is defined as:
\begin{equation}
\text{RelPFiso} = (\sum_{\text{charged}} \ET + \sum^{\text{corr}}_{\text{neutral}} \ET)/\ET^{\text{e}}
\label{eqn:elepfrelisoeqn}
\end{equation} 
where the corrected neutral component of isolation is then computed using the formula:
\begin{equation}
\label{eqn:neutralea}
  \sum^{\text{corr}}_{\text{neutral}} \ET = \text{max} \left( \sum^{\text{uncorr}}_{\text{neutral}} \ET - \ET^{PU},\, 0 \GeV \right)
\end{equation}
and the mean pile-up contribution to the isolation cone is obtained as :
\begin{equation}
  \ET^{PU} =  \rho \times A_\text{eff}
\label{eqn:purho}
\end{equation}
where $\rho$ is the mean energy density in the event and the effective area $A_{eff}$ is defined as the ratio
between the slope of the average isolation and that of $\rho$ as a function of the number of vertices.

%% was optimized in Ref.~\cite{AN-15-277} and the electron isolation working was
The threshold for the electron isolation,
calculated within a cone of radius $\DR = 0.3$
is chosen to be $\text{RelPFiso} < 0.35$. 

\subsubsection{Electron Energy Calibrations}
\todo{Remove, maybe?}
\input{Objects/eleCalib}
\subsubsection{Electron Efficiency Measurements}
\todo{Remove, maybe?}
\label{sec:eleEffMeas}
Electron efficiencies are evaluated using the Tag-and-Probe method.
The study was performed on the SingleElectron/EGamma dataset for each year separately.

Tag electrons need to satisfy the following quality requirements:
\begin{itemize}
\item trigger matched to single electron trigger
\item $\pt > 30 \GeV$, supercluster (SC) $|\eta| < 2.17$% but on in EB-EE gap ($1.4442<|\eta|<1.566$)
\item the tag and the probe need to have opposite charge.
%\item tight working point of the Spring16 cut-based electron ID
\end{itemize}

For the bin between 7 and 20\GeV, additional criteria are required:
\begin{itemize}
\item the tag has to pass a cut on the MVA score,
\item $\sqrt{2*\MET*\pt^{tag}*(1-cos(\phi_{MET}-\phi_{tag}))} < 45 \GeV$.
\item tag minimum \pt increased to 50\GeV
\item the charge is determined with the so-called selection method, requiring that all three estimates of the electron charge to agree. \todo{explain how electron charge is determined}
\end{itemize}
These cuts help cleaning the background and make the fits more reliable (and thus, the measurement more precise).

Probe electrons only need to be reconstructed as GsfElectron while the FSR recovery algorithm is not applied.

The nominal MC efficiencies are evaluated from the a Drell-Yan sample simulated with \MADGRAPH at Leading Order in QCD.

For the efficiency measurements a template fit is used.
The $m_{ee}$ signal shape of the passing and failing probes is taken from MC and convoluted with a Gaussian.
The data is then fitted with the convoluted MC template and a CMSShape (an Error-function with a one-sided exponential tail).
For the low \pt bins, a Gaussian is added to the signal model for the failing probes.

%\paragraph{Electron selection efficiency measurements}\mbox{}\\
%\label{par:Efficiency_measurements}

The electron selection efficiency is measured as a function of the probe electron $p_{T}$ and its SC $\eta$, and separately for electrons falling in the ECAL gaps.

%Figure \ref{fig:ele_sel_pt_turn_onA},~\ref{fig:ele_sel_pt_turn_onB},~\ref{fig:ele_sel_pt_turn_onC} and~\ref{fig:ele_ele_eta_turn_onA},\ref{fig:ele_ele_eta_turn_onB},~\ref{fig:ele_ele_eta_turn_onC} show the $p_{T}$ and $\eta$ turn-on curves measured in data, for 2016, 2017 and 2018.
% and the final 2D scale factor is shown in Fig.~\ref{fig:ele_sel_scale_factors} together with the systematic uncertainties. %These scale factors are very similar to the ICHEP figures, but more homogenous across $\eta$ and $p_{T}$ because of the higher statistics and the usage of more stable fitting routines in the new T\&P tool.

%\includegraphics[page=2, width=0.4\textwidth]{Figures/Electrons/ErecoEta}\\

Standard practices for the evaluation of Tag-and-Probe uncertainties for efficiency measurements are followed. Specifically, the following were considered:

\begin{itemize}
   \item Variation of the signal shape from the MC shape to an analytic shape (Crystal Ball) fitted to the MC
   \item Variation of the background shape from a CMS-shape to a simple exponential in fits to data
%   \item Variation of the tag selection: tag $p_{T}>$35~GeV and passes MVA-based 8X ID
   \item Using an NLO MC sample for the signal templates
\end{itemize}

The total uncertainty for the measurement of the scale factors is the quadratic sum of the statistical uncertainties returned from the fit and the aforementioned systematic uncertainties.


\subsection{Muons}
\subsubsection{Muon Reconstruction}
\label{sec:muonReco}
Muons are reconstructed in the CMS detector with high efficiency and purity,
thanks to the clear signature they leave in the muon spectrometer and in the inner tracking system.
The purity is granted by the upstream calorimeters and the steel return yoke that absorb other particles (except neutrinos),
while the inner tracker provides a precise measurement of the muon momentum.
Muon physics objects are reconstructed with dedicated algorithms combining information from different subsystems.
The final collection is composed by three different muon types:

\begin{itemize}
\item Standalone muons, built from the information provided by the outer Muon System.
      One or more segments, each built from hits in a single DT or CSC chamber are combined
      with RPC hits and fitted to build a standalone-muon track.
\item Tracker muons, built by propagating tracks from the inner tracker outward,
      requiring a match with at least one segmant made of hits in the DT or CSC.
      The high probability that a tracker muon to have one single matched segment in the muon system
      makes this algorithm very efficient at low momentum ($\PT < 5 \GeV$).
\item Global muons, built by propagating standalone-muon tracks inward to the inner tracker.
      In case of match, the hits from the two different tracks are fitted jointly into a global-muon track.
\end{itemize}

Global and tracker muons that share the same inner track are merged into a single muon object.
The charge and momentum are extracted from the tracker track for muons of \PT $< 200$ \GeV,
since multiple scattering limits the precision of the muon system at low momentum.
Above that threshold, charge and momentum are extracted from the global fit.

The efficiency of muon reconstruction is very high, around 99 \% within the detector acceptance,
thanks to the high efficiency of both the inner track and muon system reconstruction.

\subsubsection{Muon Identification}
\label{sec:muonID}
%More details on muon reconstruction can be found in Ref.~\cite{AN-15-277}.
Muon candidates are selected among global muons and tracker muons.
Standalone muon tracks that are only reconstructed in the muon system are not used.

Muons are required to have transverse momentum $\pt > 5 \GeV$ and pseudorapidity $|\eta| < 2.4$.
As for the electrons, the muon track is required to be compatible with the primary vertex,
to suppress contributions from in-flight decays of hadrons, pileup and cosmic rays.
This requirement translates into $d_{xy} < 0.5 \cm$, $d_z < 1 \cm$, where $d_{xy}$ and $d_z$ are
the muon impact parameters with respect to the primary vertex in the transverse plane and longitudinal direction respectively.
These requirements correspond to the analysis definition of \textbf{loose muons}.

Loose muons with \pt below 200\GeV that also pass
the PF loose muon ID \cite{ParticleFlow} are considered \textbf{identified muons} for this analysis.
Muons with \pt $>$ 200\GeV must pass either the PF identification or the Tracker High-\pt ID,
whose definition is reported in Table \ref{tab:highPtID}.
This relaxed definition is used to increase signal efficiency in the high
centre of mass energy regime.
In the laboratory frame of reference, the leptons coming from the decay of
a highly boosted $\cPZ$ will be nearly collinear, and the PF ID loses 
efficiency for muons separated by approximately $\Delta R < 0.4$, which roughly 
corresponds to muons originating from $\cPZ$ bosons with $\pt > 500\GeV$.

\begin{table}
    \begin{small}
    \begin{center}
    \caption{
      The requirements for a muon to pass the Tracker High-$\pt$ ID. Note that
      these are equivalent to the Muon POG High-$\pt$ ID with the global track 
      requirements removed.
      }
    \begin{tabular}{|l|l|}
      \hline
      Plain-text description         & Technical description                 \\
      \hline
      Muon station matching          & Muon is matched to segments           \\
                                     & in at least two muon stations         \\
                                     & \textbf{NB: this implies the muon is} \\
                                     & \textbf{an arbitrated tracker muon.}  \\
      \hline                                                          
      Good $\pt$ measurement         & $\pt / \sigma_{\pt} < 0.3$            \\
      \hline
      Vertex compatibility ($x-y$)   & $d_{xy} < 2$~mm                       \\
      \hline
      Vertex compatibility ($z$)     & $d_{z} < 5$~mm                        \\
      \hline
      Pixel hits                     & At least one pixel hit                \\
      \hline
      Tracker hits                   & Hits in at least six tracker layers   \\
      \hline
    \end{tabular}
    \label{tab:highPtID}
    \end{center}
    \end{small}
\end{table}

An additional \textit{ghost-cleaning} step is performed to deal with situations when a single muon
can be incorrectly reconstructed as two or more muons:

\begin{itemize}

\item Tracker Muons that are not Global Muons are required to be matched to reconstructed segments in at least two stations of the muon system;
\item If two muons are sharing 50\% or more of their segments then the muon with lower quality is removed.

\end{itemize}

\subsubsection{Muon Isolation}
\label{sec:muoniso}
A Particle Flow based isolation is used to suppress the contamination from muon from hadronic decays inside jets.
The so-called $\Delta\beta$ correction is applied in order to subtract the \pileup{} contribution for the muons, 
whereby $\Delta\beta = \frac{1}{2} \sum^\text{charged had.}_\text{PU} \pt$
gives an estimate of the energy deposit of neutral particles (hadrons and photons) from \pileup{} vertices.

The relative isolation for muons is then defined as:
\begin{equation}
\text{RelPFIso} = \frac{1}{\pt^\text{muon}} \left( \sum_\text{charged had.} \pt + \max(0, \sum_\text{neutral had.} \ET + \sum_\text{photon} \ET - \Delta \beta) \right)
\label{eqn:mupfiso}
\end{equation}

where the sums run over the photons, charged and neutral hadrons in a cone with $\DR = 0.3$ around the muon.
Only charged hadrons originating from the primary vertex are included to minimise the \pileup{} contribution.

The isolation cone for muons was optimised and the working point was chosen to be $\text{RelPFiso}(\Delta R = 0.3) < 0.35$. 

Similarly to electrons, a condition on the significance of the 3D impact parameter (\SIPthreeD, see Equation \ref{eq:SIP3D}) is applied,
in order to ensure that muons are consistent with the primary vertex.
Muons are required to satisfy $\SIPthreeD < 4$.


Loose muons that pass also the identification, isolation and \SIPthreeD requirements are defined \textbf{tight muons}.

\subsubsection{Muon Energy Calibrations}
\todo{Remove, maybe?}
Two methods are used to calibrate the muon momentum scale~\cite{CMS-MUO-16-001}.
The magnitudes of the momentum scale corrections are about 0.2\usep\% and 0.3\usep\% in the barrel and endcap, respectively~\cite{CMS-MUO-16-001}.
The uncertainty in the resolution is estimated to be about 5\usep\% of its value for both techniques.

\paragraph{Rochester corrections\\}
In the first method, muon momentum scale is measured in data by fitting
the di-muon mass spectrum
in a two-step process
in $Z \rightarrow \mu\mu$ events~\cite{RochesterMuon}.
%% Three datasets are used:
%% the first is a simulation with realistic detector conditions of $\PGg/\PZ \to \PGm \PGm$ events (realistic);
%% the second is a simulation with a perfectly aligned detector (ideal);
%% the third contains data.

In the first step, the corrections are derived
for bins of charge, $\eta$ and $\phi$
by requiring that the average $<1/\pt^\mu>$
in data and MC to be the same as that in an ideal simulation with a perfectly aligned detector.
This produces a table of uncorrelated corrections.
%% This step is called $<1/\pt^\mu>$ correction.

The second step aims at correcting the residual mismodelling of detector efficiency in $\eta$, $/\phi$.
In order to be independent of the modelling assumptions
for the \pt and $\eta$ distributons of $\PZ \to \mu\mu$ events,
the reconstructed \PZ mass is required to be equal to that in the ideal MC for each bin in $\eta$ and $\phi$.
%% This step is called $\Delta M/M$ tuning.
%% Next \textit{addittive} corrections are extracted by taking the difference between the \PGmp and \PGmm scale corrections,
%% which are caused by misalignments,
%% and \textit{multiplicative} corrections
%% which are caused by mismodelling of the magnetic field.

\paragraph{Corrections with Kalman filter\\}
The second method uses \PJGy and \PGU(1S) events~\cite{CMS-PAS-SMP-14-007}.
The effect of variations of the magnetic field, misalignment, and mismodelling of the material
on the measured muon curvature are calculated.
Their values are extracted by fitting the dimuon mass with a Kalman filter
in several bins of pseudorapidity, for a total of 44 parameters.

\subsubsection{Muon Efficiency Measurements}
\todo{Remove, maybe?}
\label{sec:muonEffMeas}
Muon efficiencies are measured with the Tag-and-Probe method performed on
$\cPZ \to \Pgm\Pgm$ and $\JPsi\to\mu\mu$ events in bins of $\pt$ and $\eta$. 
% More
%details on the methodology can be found in Ref.~\cite{CMS_AN_2015-277}. Measurements are extracted using 2018 RunA,B,C,D data while the measurements corresponding to 2016 and 2017 datasets have already been reported in Ref.~\cite{CMS_AN_2016-442} and Ref.~\cite{CMS_AN_2017-342} respectively.
%
The $\Z$ sample is used to measure the muon reconstruction and identification efficiency at high $\pt$,
and the efficiency of the isolation and impact parameter requirements at all $\pt$.
%
The $\JPsi$ sample is used to measure the reconstruction efficiency at low $\pt$,
as it benefits from a better purity in that kinematic regime.
In this case, events are collected using triggers that require a muon with $\pt > 7.5\GeV$
and an additional track (muon) with $\pt > 2 \GeV$ when probing the reconstruction (tracking) efficiency.

Results for the muon reconstruction and identification efficiency for $\pt > 5\GeV$
have been derived by the CMS collaboration.
The probe in this measurement are tracks reconstructed in the inner tracker, and
the passing probes are those that are also reconstructed as a global or tracker muon 
and pass the identification criteria.
%
Results for low \pt muons were derived using \JPsi events, with the same definitions
of probe and passing probes. The systematic uncertainties are estimated by varying the analytical signal and background shape models used to fit 
the dimuon invariant mass. 
% Details on the procedure can be found in Ref.~\cite{AN-15-277}. 
The efficiency and scale 
factors used for low $\pt$ muons are the ones derived using single muon dataset.

For the impact parameter requirements, the measurement is performed using $\Z$ events.
Events are selected with single muon triggers requiring an isolated muon with $\pt > 27\GeV$ or a muon with $\pt > 50 \GeV$.
For this measurement, the probe is a muon passing the identification criteria,
and it is considered a passing probe if satisfies the \SIPthreeD, $d_{xy}$ and $d_z$ cuts of this analysis.
%
The efficiency to reconstruct a muon track in the inner detector is measured using as probes tracks
reconstructed in the muon system alone. The efficiency and 
data to MC scale factors are measured from Z events as a function of $\eta$ and \pt.
The scale factors are shown in Figure~\ref{fig:muoSFRun2}.

\begin{figure}
  \centering
  \subfigure [2016] {\includegraphics[height=.25\textheight]{final_HZZ_SF_2016UL_mupogsysts_newLoose_FINAL.pdf}}\\
  \subfigure [2017] {\includegraphics[height=.25\textheight]{final_HZZ_SF_2017UL_mupogsysts_newLoose_FINAL.pdf}}\\
  \subfigure [2018] {\includegraphics[height=.25\textheight]{final_HZZ_SF_2018UL_mupogsysts_newLoose_FINAL.pdf}}
  \caption{Efficiency scale factors, calculated as ratio between the efficiency in data and in simulation,
  applied to muons for the three years of data taking in \RunII.}
  \label{fig:muoSFRun2}
\end{figure}

%\subsection{Lepton Momentum Scale and Resolution validation using \texorpdfstring{\Zllll}{}}
%\input{Objects/scaleresoZ4l}
\subsection{Photons for FSR recovery}
\label{sec:FSRphotons}
Final State Radiation (FSR) photons emitted by leptons are not included at all in the Particle Flow reconstruction of muon momentum, and may be missed in electron reconstruction, leading to a degradation of the accuracy for the Z bosons momentum and mass.

FSR candidates are required to have $p_{T}^{\gamma} >$ 2 GeV, $|\eta^{\gamma}| <$ 2.4, relative isolation $<$ 1.8 (see Equation \ref{eqn:mupfiso}) and $\Delta R(\ell, \gamma) <$ 0.5 with respect to the nearest signal lepton.
Because the electron reconstruction algorithm already recovers some of the FSR photons, we exclude those that have $\Delta R(e, \gamma) <$ 0.15 or $|\Delta\phi(e, \gamma)| <$ 2 and $|\Delta\eta(e, \gamma)| <$ 0.05 to avoid double counting.
Since FSR photons tend to have higher energies than the ones from pileup, and are expected to be quasi-collinear with the emitting leptons, an FSR candidate is accepted if $\Delta R(\ell, \gamma)^{} / E_{T,\gamma}^{2} <$ 0.012 GeV$^{-2}$.
Accepted FSR photons have their momentum added to the lepton and are excluded from the computation of its isolation.
\\
Since our analysis selection requires signal photons to have at least $\Delta R(\ell, \gamma) > 0.5$ from any signal lepton, there is no double counting.
However, we studied the kinematic distributions of photons passing all the other basic kinematic selections with this cut relaxed.
For example, their distribution in bins of $\Delta R(\ell, \gamma)$ with respect to the closest lepton
can be seen in Figure \ref{fig:dRl_fsr_photons}.
\begin{figure}[ht]
\begin{center}
        \includegraphics[width=0.8\textwidth]{Figures/lead_dRl_kin_vs_fsrMatched_rebinned.png}
\end{center}
\caption{$\Delta R(\ell, \gamma)$ for all the photons passing at least the `kinematic' selection with the $\Delta R$ cut relaxed, and for those selected as FSR in the ZZ$\gamma$ sample 2018.}
\label{fig:dRl_fsr_photons}
\end{figure}
% and the effect of not excluding FSR photons from the computation of the nonprompt rate

\subsection{Jets}
\label{sec:jets}
\subsubsection{Jet Identification}

Jets are reconstructed by using the anti-$k_T$ clustering algorithm out of particle flow candidates, with a distance parameter $R = 0.4$, 
after rejecting the charged hadrons that are associated to a pileup primary vertex.

To reduce instrumental background, the tight working point jet ID suggested by the JetMET Physics Object Group is applied.%~\cite{JetID2018}. 
In addition, jets from Pile-Up are rejected using the PileUp jet ID criteria suggested by the JetMET POG.%~\cite{JetPUID2017}.
It is to be noted that the PU JetID was only derived for 2016 conditions but is also applied to 2017 and 2018 samples. 

In this analysis, the jets are required to be within $|\eta| < 4.7$ area and have a transverse momentum above 30 GeV. 
In addition, the jets are cleaned from any of the tight leptons (passing the SIP and isolation cut computed after FSR correction) 
and FSR photons by a separation criterion: $\Delta R(\text{jet,lepton/photon}) > 0.4$.
\\
In addition this analysis considers also a collection of large radius jets clustered using the same anti-$k_T$ algorithm with a distance parameter $R = 0.8$.
These jets are cleaned using the Pileup Per Particle Identification (PUPPI) \cite{Bertolini_2014}, which is a method for pileup mitigation.

\subsubsection{Jet Energy Corrections}

The calorimeter response to particles is not linear
and it is not straightforward to translate the measured jet energy
to the true particle or parton energy, therefore Jet Energy Corrections must be applied.
In this analysis, standard jet energy corrections are applied to the reconstructed jets,
which consist of L1 Pileup, L2 Relative Jet Correction,
L3 Absolute Jet Correction for both Monte Carlo samples and data,
and also residual calibration for data%.~\cite{JECMC2018}. 

Jet corrections are applied following JetMET Physics Object Group recommendations. The corrections used are as follows:
\begin{itemize}
\item Jet energy scale corrections for data
\begin{itemize}
\item 2016: Summer19UL16\_RunBCDEFGH\_Combined\_V7\_DATA\_AK4PFchs
\item 2017: Summer19UL17\_RunBCDEF\_V5\_DATA\_AK4PFchs
\item 2018: Summer19UL18\_V5\_DATA\_AK4PFchs
\end{itemize}
\item Jet energy scale corrections for MC
\begin{itemize}
\item 2016preVFP: Summer19UL16APV\_V7\_MC\_AK4PFchs
\item 2016postVFP: Summer19UL16\_V7\_MC\_AK4PFchs
\item 2017: Summer19UL17\_V5\_MC\_AK4PFchs
\item 2018: Summer19UL18\_V5\_MC\_AK4PFchs
\end{itemize}
\item Jet energy resolution corrections
\begin{itemize}
\item 2016preVFP: Summer20UL16APV\_JRV3\_MC\_[Pt/Phi]Resolution\_AK4PFchs
\item 2016postVFP: Summer20UL16\_JRV3\_MC\_[Pt/Phi]Resolution\_AK4PFchs
\item 2017: Summer19UL17\_JRV3\_MC\_[Pt/Phi]Resolution\_AK4PFchs
\item 2018: Summer19UL18\_JRV2\_MC\_[Pt/Phi]Resolution\_AK4PFchs
\end{itemize}
\end{itemize}

% \textbf{At the moment only preliminary version of JEC for MC is available. As recommended no JEC is applied to data.}
%Jet Energy Resolutions corrections are, however, NOT applied to 2018 samples (see discussion below).

\paragraph{L1 pre-firing}

In 2016 and 2017, the gradual timing shift of ECAL was not properly propagated to L1 trigger primitives (TP) resulting in a significant fraction of high eta TP being mistakenly associated to the previous bunch crossing.
Since Level 1 rules forbid two consecutive bunch crossings to fire, an unpleasant consequence of this (in addition to not finding the TP in the bx 0) is that events can self veto if a significant amount of ECAL energy is found in the region of $2.<|\eta|<3$.
This effect is not described by the simulations.%~\cite{L1PrefiringTwiki}.
The probability not to prefire is calculated for each event and applied as a weight to simulation for 2016 and 2017 samples.
The official tool is used for this purpose.%~\cite{L1PrefiringTwiki}.

\begin{figure}
\subfigure [L1 pre-firing weights]       {\resizebox{.5\textwidth}{!}{\includegraphics[width=.5\textwidth]{Figures/L1Prefiring_ZZGTo4LG.png}}}
\subfigure [Effect on $m_{4\ell\gamma}$] {\resizebox{.5\textwidth}{!}{\includegraphics[width=.5\textwidth]{Figures/SYS/SR4P/ZZGTo4LG_mZZGloose_L1Prefiring.png}}}
\caption{L1 pre-firing weights and effect of their application on the signal MC in the region SR4P\_P in 2018. The photon is required to pass the Loose cut-based ID}
\label{fig:L1Prefiring}
\end{figure}

The Fig~\ref{fig:L1Prefiring} shows the impact of the L1 pre-firing weights on the signal MC.
The impact on the normalization of the signal is around 0.8\%.

\paragraph{Removal of noisy jets}

Increased jet multiplicity was reported for 2017 data, creating ``horns'' in the jet $\eta$ distribution for $2.5<|\eta_{jet}|<3$.
The issue was linked to an increase of the ECAL noise, PU and bunch-crossing dependent, thus getting worse as luminosity increases.
The problem can only be fixed in the UL ReReco.
For now, we checked the impact of rejecting jets with raw $p_T<50$ GeV in 2.65 $<|\eta| <$ 3.139.
As we see no significant impact in the data/MC agreement, we decided not to use these cuts.

%\paragraph{HEM 15/16 failures}

%Following	a CMS-wide power interlock on June 30, the power-on of CAEN A3100HBP modules that provide low-voltage power to the on-detector HE front-end electronics led to irreversible damage of two sectors on the HE minus side, HEM15 and HEM16 (FIXME: add ref). No significant impact was seen and nothing particular is done to cope with this. 

%https://twiki.cern.ch/twiki/bin/view/CMS/HIGJetMET#Known_JetMET_issues 






\subsection{Photons}
\label{sec:photons}
Photon candidates are reconstructed as Super-Clusters in the ECAL with $E_{T} > 20 GeV$ in the fiducial Barrel region and Endcap region, defined by $|\eta|<1.4442$ and $1.566<|\eta|<2.5$, respectively.
They are required to satisfy the loose cut-based selection \cite{CMS:EGM-17-001} seen in Table \ref{tab:VPhotonID}, which provides $90\%$ signal selection efficiency:

\begin{table}[h!]
\begin{center}
\scalebox{0.75}{
\begin{tabular}{|c|c|c|}
\hline 
\hline 
Requirement &  Barrel$\quad$$|\eta|<$1.4442 & Endcap$\quad$ 1.566$<|\eta|<$2.5 \\
\hline
%% Medium WP
%% $H/E<$ & 0.02197 & 0.0326 \\
%% $\sigma_{i\eta i\eta}<$ & 0.01015 & 0.0272 \\
%% Rho corrected PF charged hadron isolation $<$ & 1.141\ GeV & 1.051\ GeV\\
%% Rho corrected PF neutral hadron isolation$<$ & $1.189 + 0.01512*p_{T}^{\gamma} + 2.259e-05*p_{T}^{\gamma 2}$ & $2.718 + 0.0117*p_{T}^{\gamma} + 2.3e-05*p_{T}^{\gamma 2}$\\
%% Rho corrected PF photon isolation$<$ &$2.08 + 0.004017*p_{T}^{\gamma}$ & $3.867 + 0.0037*p_{T}^{\gamma}$\\
%% 
%% Loose WP
$H/E<$ & 0.04596 & 0.0590 \\
$\sigma_{i\eta i\eta}<$ & 0.0106 & 0.0272 \\
Rho corrected PF charged hadron isolation $<$ & 1.694\ GeV & 2.089\ GeV\\
Rho corrected PF neutral hadron isolation$<$ & $24.032 + 0.01512\, p_{T}^{\gamma} + 2.259 \cdot 10^{-5}\, p_{T}^{\gamma 2}$ & $19.722 + 0.0117\, p_{T}^{\gamma} + 2.3 \cdot 10^{-5}\, p_{T}^{\gamma 2}$\\
Rho corrected PF photon isolation$<$ & $2.876 + 0.004017\, p_{T}^{\gamma}$ & $4.162 + 0.0037\, p_{T}^{\gamma}$\\
Pass Conversion safe electron veto & True & True\\
\hline
\end{tabular}}
\caption[.]{Cut-based Loose photon ID for 94X and later samples}
\label{tab:VPhotonID}
\end{center}
\end{table}

%Barrel:
%
%\begin{itemize}
%\item Single tower H/E $<$ 0.0396 
%\item $\sigma_{i\eta i\eta}<0.01022$ 
%\item pile-up corrected PF charged hadron isolation $<$ 0.441, which charged hadrons originated from the hard interaction primary vertex.
%\item pile-up corrected PF neutral hadron isolation $< 2.725 + 0.0148*p_{T}^{\gamma} + 0.000017*p_{T}^{\gamma 2}$ 
%\item pile-up corrected PF photon isolation $<2.571 + 0.0047*p_{T}^{\gamma}$
%\item Conversion safe electron veto
%\end{itemize}
%
%Endcap:
%
%\begin{itemize}
%\item Single tower H/E $<$ 0.0219
%\item $\sigma_{i\eta i\eta}<0.03001$
%\item pile-up corrected PF charged hadron isolation $<$ 0.442, which charged hadrons originated from the hard interaction primary vertex.
%\item pile-up corrected PF neutral hadron isolation $< 1.715 + 0.0163*p_{T}^{\gamma} + 0.000014*p_{T}^{\gamma 2}$
%\item pile-up corrected PF photon isolation $<3.863 + 0.0034*p_{T}^{\gamma}$
%\item Conversion safe electron veto
%\end{itemize}

Isolation is performed for a cone of radius 0.3.
Pile-up corrected PF isolation is calculated using effective area corrections.
The photon candidates are also required to be separated from the lepton by at least $0.07$ in $\eta-\phi$ space, which highly reduces the contribution from FSR.
The same separation requirement is imposed between the photon candidates and the jets.

\subsubsection{Photon Identification}
\label{sec:photonID}

Both a cut-based and a MVA-based ID are employed and compared in the analysis.\\
Each has its own advantages and disadvantages.
Due to its nature, the cut-based ID can be easily inverted by reversing only one or a few of its cuts.
On the other hand, MVA-based ID is more effective at discriminating between signal and background, providing a higher signal-to-noise ratio.

\paragraph{Cut-based ID}
The cut-based loose photon ID, described in ~\cite{CMS:EGM-17-001} and shown in table~\ref{tab:VPhotonID} is employed.
The corresponding ID scale factors, shown in Figure \ref{fig:phEffSF} are applied following the recommendations provided by the CMS experts on electrons and photons reconstruction.

\begin{figure}
\subfigure [2016preVFP ] {\resizebox{.5\textwidth}{!}{\includegraphics[width=.5\textwidth]{Figures/phEffSF_2016preVFP.pdf} }}
\subfigure [2016postVFP] {\resizebox{.5\textwidth}{!}{\includegraphics[width=.5\textwidth]{Figures/phEffSF_2016postVFP.pdf}}}\\
\subfigure [2017]        {\resizebox{.5\textwidth}{!}{\includegraphics[width=.5\textwidth]{Figures/phEffSF_2017.pdf}}}
\subfigure [2018]        {\resizebox{.5\textwidth}{!}{\includegraphics[width=.5\textwidth]{Figures/phEffSF_2018.pdf}}}
\caption{Photon efficiency scale factors for the POG cut-based Loose ID.}
\label{fig:phEffSF}
\end{figure}

Besides the identification working points, an electron veto selection (CSEV veto) is also applied.
%% The scale factors are shown in ~\ref{tab:eleveto_SFs}.

%% \begin{table}[htbp]
%%  \centering
%%    \begin{tabular}{|c|c|l|l|}
%%    \hline
%%    Year & $p_T$& barrel & endcap\\ \hline
%%    2016 & inclusive &0.9938 $\pm$ 0.0119 & 0.9875 $\pm$ 0.0044\\\hline
%%    2017 & inclusive & 0.9862 $\pm$ 0.0030 & 0.9638 $\pm$ 0.0047\\\hline
%%    \multirow{3}{*}{2018} &10 GeV$<p_{T}^{\gamma}<30$ GeV &0.9869 $\pm$ 0.0043& 0.9535 $\pm$ 0.0054\\
%%    & 30 GeV$<p_{T}^{\gamma}<$60 GeV  &0.9908  $\pm$ 0.0111 & 0.9646 $\pm$ 0.0076\\
%%    & 60 GeV$<p_{T}^{\gamma}<$200 GeV &1.0084  $\pm$ 0.0856& 1.0218 $\pm$ 0.1178\\
%%    \hline
%%    \end{tabular}
%%    \caption{Electron veto scale factors for barrel and endcap corresponding to 2016 to 2018.}
%%    \label{tab:eleveto_SFs}
%%  \end{table}

%\begin{figure}[b]
%  \begin{center}
%    \includegraphics[width=0.8\textwidth]{figs/photon_SFs.pdf}
%    \caption{Photon ID scale factors for cut-based loose Photon selection}
%    \label{fig:PhotonEff}
%  \end{center}
%\end{figure}

\paragraph{Multivariate ID}
The multivariate (MVA) ID employs 14 variables linked to the energy and shower shape of the ECAL supercluster associated with the photon, as well as its isolation from other particles in the event.
These variables include those utilized by the cut-based ID, so the two are not independent.
The full list is shown in Table \ref{tab:MVAvariables}.
The version used is \texttt{RunIIFall17v2}.

\begin{table}[ht]
\centering
\begin{tabular}{l|l}
Name & variable\\
\hline
SCRawE             & superCluster.rawEnergy                                               \\
r9                 & r9                                                                   \\
sigmaIetaIeta      & full5x5\_showerShapeVariables.sigmaIetaIeta                          \\
etaWidth           & superCluster.etaWidth                                                \\
phiWidth           & superCluster.phiWidth                                                \\
covIEtaIPhi        & full5x5\_showerShapeVariables.sigmaIetaIphi                          \\
s4                 & full5x5\_showerShapeVariables.e2x2/full5x5\_showerShapeVariables.e5x5\\
scEta              & superCluster.eta                                                     \\
rho                & fixedGridRhoAll                                                      \\
esEffSigmaRR       & full5x5\_showerShapeVariables.effSigmaRR                             \\
esEnergyOverRawE   & superCluster.preshowerEnergy/superCluster.rawEnergy                  \\
phoIso03           & photonIso                                                            \\
chgIsoWrtChosenVtx & chargedHadronIso                                                     \\
chgIsoWrtWorstVtx  & chargedHadronWorstVtxIso                                             \\
\end{tabular}
\caption[.]{Variables used by the MVA-based ID, version \texttt{RunIIFall17v2}}
\label{tab:MVAvariables}
\end{table}

Two working points are provided centrally by the EGamma POG: \texttt{wp90} and \texttt{wp80}, corresponding to 90 \% and 80 \% prompt photon efficiency respectively.
The cuts on the MVA estimator value that define the two working points are detailed in Table \ref{tab:MVAwpCuts}.

\begin{table}[ht]
\centering
\begin{tabular}{|l|c|c|}
\hline
Name & Barrel & Endcap \\
\hline
\texttt{wp90} & -0.02 & -0.26 \\
\texttt{wp90} &  0.42 &  0.14 \\
\hline
\end{tabular}
\caption[.]{Working points of the photon MVA-based ID, version \texttt{RunIIFall17v2}}
\label{tab:MVAwpCuts}
\end{table}

