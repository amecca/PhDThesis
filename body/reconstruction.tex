\chapter{Reconstruction}

The collision events registered by the CMS detector are reconstructed by combining information from the different subdetectors.
This process aims to improve the identification of the final-state particles and the reconstruction precision of their properties.
The global reconstruction strategy is implemented through the Particle Flow (PF) algorithm \cite{ParticleFlow}.
Originally applied in the ALEPH experiment at LEP, this algorithm combines data from the CMS subsystems to reconstruct physics objects and achieve a comprehensive and coherent event reconstruction.
The resulting physics objects include charged and neutral hadrons, photons, electrons, and muons, serving as the starting inputs for all subsequent data analyses.
Additionally, PF candidates are used to construct more advanced objects such as jets and missing transverse energy.

% NOTE: the content of this chapter is taken (and adapted) from the Analysis Note

%% Electrons and muons are considered candidates for the reconstruction of ZZ final states
%% (``signal leptons'') if  their transverse momentum,
%% $\pt^\ell$, is greater than 7(5)~\GeV and their pseudorapidity,
%% $\left| \eta^\ell \right|$, is less than 2.5 (2.4) for electrons (muons).
%% The physics objects and the ZZ candidate selections used in this analysis are those of the $H \rightarrow ZZ \rightarrow 4\ell$~\cite{HiggsAN,HiggsLegacyPaper} analysis, with minor changes. Here, we just report the main features, referring to that material for control plots/tables etc.

\section{Base elements}
\subsection{Tracks}
\label{sec:tracks}
The reconstruction of the trajectory of charged particles with high precision at the CMS detector
is a complex task because of the large combinatorics from the high multiplicity of particles and the large number of readout channels.
Additional systematic effects having an impact on the track reconstruction may arise from possible distortions in the tracking
material, inhomegeneities in the magnetic field and misalignment of the detector components.
First local charge clusters are converted to hits using the digitised output from the tracking system.
The local track reconstruction output is then buffered for the global track reconstruction, which aims at identifying
hit combinations that match to possible trajectories of the charged particles present in the event.
All steps of the reconstruction are performed using an iterative pattern recognition technique, a Kalman-like fitting procedure adapted for the CMS framework in the Combinatorial Track Finder \cite{billoir.qian:simultaneous} (CTF) algorithm.
At each iteration of CTF, positional information from the hits used in the previous step is discarded and the set of requirements are gradually relaxed.
The tracking algorithm carries out three main substasks.
\begin{description}
\item[seed finding] involving the generation of the starting points of the iterative sequence, namely pairs or triplets of hits.
\item[pattern recognition] iteratively performs the following steps:
  \begin{enumerate}
  \item navigation: the current track parameters are used to determine which adjacent layers of the detector can be interseted by extrapolation
  \item a search is performed within the layer for modules which are compatible with the trajectory
  \item groups of hits are formed for each module, and a $\chi^2$ test is used to determine their compatibility with the trajectory
  \item the trajectory is updated using the information from the hits collected in the current iteration
  \end{enumerate}
  at each iteration, track candidates must satisfy a series of quality criteria, based on cuts on the track impact parameter significance with respect to the beamspot, the number of hits in the inner tracking system and the normalised $\chi^2$ of the track trajectory, and a maximum of 5 candidates is retained.
\item[final fit] the best-fit value of the track parameters and the covariance matrix are determined by means of a least-squares fit.
\end{description}

% Primary Vertex
The pp collision vertices in an event are reconstructed by grouping tracks consistent with originating at a common point in the luminous region.
The candidate vertex with the largest value of summed physics-object $p^2_T$ is taken to be the primary pp interaction vertex.
The physics objects are the jets, clustered using the anti-kT jet finding algorithm \cite{Cacciari:2008gp, Cacciari:2011ma} with all the tracks assigned to candidate vertices as inputs,
and the associated missing transverse momentum, taken as the negative vector $p_T$ sum of those jets.


\subsection{Calorimeter clusters}
\label{sec:clusters}
First, \textit{cluster seeds} are identified as crystals with local energy maximum above a certain threshold.
Then, \textit{topological clusters} are grown by aggregating crystals with at least one one side in common with one already in the cluster,
if their energy exceeds a threshold defined as twice the noise level in the crystal.
An expectation-maximisation algorithm based on a Gaussian-mixture model is then used to reconstruct the \textit{PF clusters} within a topological cluster,
based on the assumption that the number of PF clusters is equal to the number of seeds.
Finally, the energy of each cell within a topological cluster is shared among all the associated PF clusters according to the cell-cluster distance,
with an iterative determination of the cluster energies and positions.


\subsection{Electrons}
\label{sec:eleReco}
%More details on electron reconstruction can be found in Ref.~\cite{ElectronLegacy}. 

Electron candidates are preselected using loose cuts on track-cluster matching observables, so as to preserve the highest possible efficiency while rejecting part of the QCD background. To be considered for the analysis, electrons are required to have a
transverse momentum $p^e_T >$ 7 GeV, a reconstructed $|\eta^e| <$ 2.5, and to satisfy a loose primary vertex 
constraint defined as $d_{xy} < 0.5$ cm and $d_z < 1$ cm.
Such electrons are called {\bf loose electrons}.

The data-MC discrepancy is corrected using scale factors as is done for the electron selection with data efficiencies measured using the same tag-and-probe technique outlined later (see Section~\ref{sec:eleEffMeas}). 
These studies for reconstructions are carried out by the EGM POG and the results are only summarised here.

The electron reconstruction scale factors 
% are shown Fig.~\ref{fig:ele_rec_scale_factors} and 
are applied as a function of the super cluster $\eta$ and electron $\pt$.




\subsection{Muon tracking}
\label{sec:muonReco}
Muons are reconstructed in the CMS detector with high efficiency and purity,
thanks to the clear signature they leave in the muon spectrometer and in the inner tracking system.
The purity is granted by the upstream calorimeters and the steel return yoke that absorb other particles (except neutrinos),
while the inner tracker provides a precise measurement of the muon momentum.
Muon physics objects are reconstructed with dedicated algorithms combining information from different subsystems.
The final collection is composed by three different muon types:

\begin{itemize}
\item Standalone muons, built from the information provided by the outer Muon System.
      One or more segments, each built from hits in a single DT or CSC chamber are combined
      with RPC hits and fitted to build a standalone-muon track.
\item Tracker muons, built by propagating tracks from the inner tracker outward,
      requiring a match with at least one segmant made of hits in the DT or CSC.
      The high probability that a tracker muon to have one single matched segment in the muon system
      makes this algorithm very efficient at low momentum ($\PT < 5 \GeV$).
\item Global muons, built by propagating standalone-muon tracks inward to the inner tracker.
      In case of match, the hits from the two different tracks are fitted jointly into a global-muon track.
\end{itemize}

Global and tracker muons that share the same inner track are merged into a single muon object.
The charge and momentum are extracted from the tracker track for muons of \PT $< 200$ \GeV,
since multiple scattering limits the precision of the muon system at low momentum.
Above that threshold, charge and momentum are extracted from the global fit.

The efficiency of muon reconstruction is very high, around 99 \% within the detector acceptance,
thanks to the high efficiency of both the inner track and muon system reconstruction.


\section{Particle Flow - Global event reconstruction}
\label{sec:ParticleFlow}
The algorithm relies on precise track reconstruction and clustering techniques that efficiently distinguish between overlapping showers.
A robust linking procedure is used to combine information related to energy deposits associated with a single particle across different sub-detectors.
The PF sequence begins with identifying tracks of charged particles, followed by extrapolating these tracks to compatible calorimeter energy deposits.

Neutral hadrons and photons are identified through \textit{Charged Hadron Subtraction}, which removes the energy deposits that can be linked to tracks from charged hadrons.
The remaining deposits in ECAL and HCAL are attributed to photons and neutral hadrons,
provided that the cluster energies of neutral hadron candidates exceed their track momentum, accounting for detector resolution effects.

The CMS detector's exceptional performance with the PF algorithm is attributed to the high granularity of the electromagnetic calorimeter, hermeticity of the hadron calorimeter, and a large magnetic field integral.


\paragraph{Link algorithm\\}

The goal of the link algorithm is to connect the elements resulting from the local reconstruction in the various subdetectors
and produce \textit{PF blocks} which will be used to provide the global event description.
While the linking can test any pair of elements, only the nearest neighbours in the $(\eta,\phi)$ plane are considered to limit the complexity.

Tracks are linked to calorimeter clusters by extrapolating from the outermost hit to the preshower, ECAL and HCAL.
The track is linked to a cluster if its extrapolated position is within the cluster area, accounting for gaps between active elements and position uncertainty.

The energy of photons emitted by electron bremsstrahlung is recovered by extrapolating tangents to the GSF tracks to ECAL from each of the tracker layers.
If the extrapolated position is within a cluster and has distance in $\eta$ smaller than 0.05 to che cluster center, the link is created.
Photons have a significant probability to convert to an $\Pep \Pem$ pair in the tracker material.
A dedicated conversion finder creates links between any two tracks compatible with originating from a photon conversion.
If the converted photon direction, obtained from the sum of the two track momenta, is found to be compatible with one of the aforementioned track tangents, a link is created between each of these two tracks and the original electron track.

Links between clusters belonging to different calorimeters (preshwoer, ECAL and HCAL) are created if the position of the cluster in the more granular detector is within the area of the other.
Charged particle tracks may be linked together through a secondary vertex, provided they pass some quality cuts.
Finally, links between the inner tracker and the muon detector tracks are established to form global and tracker muon tracks, as described in section \ref{sec:muonReco}.

\paragraph{Processing of blocks\\}

Thanks to the granularity of the CMS subdetectors, most PF block only contain a few elements from one particle.
Each PF block is processed separately in the several steps. After each step the elements that were used are removed from the block:
\begin{enumerate}
\item muon candidates are identified and reconstructed;
\item electron identification and reconstruction, including bremsstrahlung recovery, and isolated photon reconstruction are performed in the same step;
\item non-isolated photons, charged and neutral hadrons from parton fragmentation, hadronization, and decays in jets are identified;
\item hadrons that underwent nuclear interactions in the tracker and produced secondary particles are identified and reconstructed;
\end{enumerate}
Finally, after all blocks have been processed and the global event description is available, the event is revisited by a post-processing step.


\subsection{Electrons and isolated photons}
Electron and photon reconstruction are conducted together.
The former are seeded as described in Section~\ref{sec:eleReco},
while the latter are seeded by ECAL superclusters not linked to any tracks and $\ET \geq 10 \GeV$.
The sum of HCAL energy within $\DR < 0.15$ of the SC centre must not exceed 10\usep\% of the SC energy.
All the clusters linked to the SC or to a tangent of the GSF track are collected.
Tracks linked to these clusters are also associated if their momentum and HCAL clusters are compatible with the electron hypothesis.

The total energy of the ECAL clusters is corrected with analytical function of the energy and pseudorapidity to account for missing association,
up to 25\usep\% at $\abs{\eta} \simeq 1.5$ and low \pt.
This energy is assigned to photon candidate; the photon direction is that of the supercluster.
The estimation of electron momentum
relies on a combination of the energy of the supercluster and the momentum estimate of the GSF track.

\subsection{Muons}
\todo{}
\subsection{Hadrons and nonisolated photons}
Once muons, electrons, and isolated photons are identified and removed from the PF blocks,
the remaining particles to be identified are hadrons from jet fragmentation and hadronization.
These particles may be detected as charged hadrons, neutral hadrons, nonisolated photons, and more rarely additional muons.

Within tracker acceptance, all remaining ECAL and HCAL clusters not linked to any track give rise to photons and charged hadrons respectively.
Outside the tracker, only ECAL clusters not linked to an HCAL cluster are classified as photons.
The remaining HCAL clusters are each linked to one or more tracks.
The sum of the track momenta is then compared to the calibrated calorimetric energy in order to determine the particle content.
If the calorimetric energy is in excess by an amount larger than the expected resolution,
it is interpreted by a photon and possibly to a neutral hadron.
Each track becomes a charged hadron with momentum and energy from the track itself.
All the momenta are refitted jointly using the calorimeter deposits and the track information.

If the calorimeter energy is smaller than the track momenta by more than three standard deviations,
identified global muon tracks are masked.
If the excess persists, misreconstructed tracks with \pt uncertainty larger than 1\GeV
are sequentially masked in decreasing \pt order until the excess disappears or the PF block is empty.

%% \subsection{Event post processing}
%% \todo{}

\subsection{Photons for FSR recovery}
\label{sec:FSRphotons}
Final State Radiation (FSR) photons emitted by leptons are not included at all in the Particle Flow reconstruction of muon momentum, and may be missed in electron reconstruction, leading to a degradation of the accuracy for the Z bosons momentum and mass.

FSR candidates are required to have $p_{T}^{\gamma} >$ 2 GeV, $|\eta^{\gamma}| <$ 2.4, relative isolation $<$ 1.8 (see Equation \ref{eqn:mupfiso}) and $\Delta R(\ell, \gamma) <$ 0.5 with respect to the nearest signal lepton.
Because the electron reconstruction algorithm already recovers some of the FSR photons, we exclude those that have $\Delta R(e, \gamma) <$ 0.15 or $|\Delta\phi(e, \gamma)| <$ 2 and $|\Delta\eta(e, \gamma)| <$ 0.05 to avoid double counting.
Since FSR photons tend to have higher energies than the ones from pileup, and are expected to be quasi-collinear with the emitting leptons, an FSR candidate is accepted if $\Delta R(\ell, \gamma)^{} / E_{T,\gamma}^{2} <$ 0.012 GeV$^{-2}$.
Accepted FSR photons have their momentum added to the lepton and are excluded from the computation of its isolation.
\\
Since our analysis selection requires signal photons to have at least $\Delta R(\ell, \gamma) > 0.5$ from any signal lepton, there is no double counting.
However, we studied the kinematic distributions of photons passing all the other basic kinematic selections with this cut relaxed.
For example, their distribution in bins of $\Delta R(\ell, \gamma)$ with respect to the closest lepton
can be seen in Figure \ref{fig:dRl_fsr_photons}.
\begin{figure}[ht]
\begin{center}
        \includegraphics[width=0.8\textwidth]{Figures/lead_dRl_kin_vs_fsrMatched_rebinned.png}
\end{center}
\caption{$\Delta R(\ell, \gamma)$ for all the photons passing at least the `kinematic' selection with the $\Delta R$ cut relaxed, and for those selected as FSR in the ZZ$\gamma$ sample 2018.}
\label{fig:dRl_fsr_photons}
\end{figure}
% and the effect of not excluding FSR photons from the computation of the nonprompt rate


\subsection{Jets}
\label{sec:jets}
\subsubsection{Jet Identification}

Jets are reconstructed by using the anti-$k_T$ clustering algorithm out of particle flow candidates, with a distance parameter $R = 0.4$, 
after rejecting the charged hadrons that are associated to a pileup primary vertex.

To reduce instrumental background, the tight working point jet ID suggested by the JetMET Physics Object Group is applied.%~\cite{JetID2018}. 
In addition, jets from Pile-Up are rejected using the PileUp jet ID criteria suggested by the JetMET POG.%~\cite{JetPUID2017}.
It is to be noted that the PU JetID was only derived for 2016 conditions but is also applied to 2017 and 2018 samples. 

In this analysis, the jets are required to be within $|\eta| < 4.7$ area and have a transverse momentum above 30 GeV. 
In addition, the jets are cleaned from any of the tight leptons (passing the SIP and isolation cut computed after FSR correction) 
and FSR photons by a separation criterion: $\Delta R(\text{jet,lepton/photon}) > 0.4$.
\\
In addition this analysis considers also a collection of large radius jets clustered using the same anti-$k_T$ algorithm with a distance parameter $R = 0.8$.
These jets are cleaned using the Pileup Per Particle Identification (PUPPI) \cite{Bertolini_2014}, which is a method for pileup mitigation.

\subsubsection{Jet Energy Corrections}

The calorimeter response to particles is not linear
and it is not straightforward to translate the measured jet energy
to the true particle or parton energy, therefore Jet Energy Corrections must be applied.
In this analysis, standard jet energy corrections are applied to the reconstructed jets,
which consist of L1 Pileup, L2 Relative Jet Correction,
L3 Absolute Jet Correction for both Monte Carlo samples and data,
and also residual calibration for data%.~\cite{JECMC2018}. 

Jet corrections are applied following JetMET Physics Object Group recommendations. The corrections used are as follows:
\begin{itemize}
\item Jet energy scale corrections for data
\begin{itemize}
\item 2016: Summer19UL16\_RunBCDEFGH\_Combined\_V7\_DATA\_AK4PFchs
\item 2017: Summer19UL17\_RunBCDEF\_V5\_DATA\_AK4PFchs
\item 2018: Summer19UL18\_V5\_DATA\_AK4PFchs
\end{itemize}
\item Jet energy scale corrections for MC
\begin{itemize}
\item 2016preVFP: Summer19UL16APV\_V7\_MC\_AK4PFchs
\item 2016postVFP: Summer19UL16\_V7\_MC\_AK4PFchs
\item 2017: Summer19UL17\_V5\_MC\_AK4PFchs
\item 2018: Summer19UL18\_V5\_MC\_AK4PFchs
\end{itemize}
\item Jet energy resolution corrections
\begin{itemize}
\item 2016preVFP: Summer20UL16APV\_JRV3\_MC\_[Pt/Phi]Resolution\_AK4PFchs
\item 2016postVFP: Summer20UL16\_JRV3\_MC\_[Pt/Phi]Resolution\_AK4PFchs
\item 2017: Summer19UL17\_JRV3\_MC\_[Pt/Phi]Resolution\_AK4PFchs
\item 2018: Summer19UL18\_JRV2\_MC\_[Pt/Phi]Resolution\_AK4PFchs
\end{itemize}
\end{itemize}

% \textbf{At the moment only preliminary version of JEC for MC is available. As recommended no JEC is applied to data.}
%Jet Energy Resolutions corrections are, however, NOT applied to 2018 samples (see discussion below).

\paragraph{L1 pre-firing}

In 2016 and 2017, the gradual timing shift of ECAL was not properly propagated to L1 trigger primitives (TP) resulting in a significant fraction of high eta TP being mistakenly associated to the previous bunch crossing.
Since Level 1 rules forbid two consecutive bunch crossings to fire, an unpleasant consequence of this (in addition to not finding the TP in the bx 0) is that events can self veto if a significant amount of ECAL energy is found in the region of $2.<|\eta|<3$.
This effect is not described by the simulations.%~\cite{L1PrefiringTwiki}.
The probability not to prefire is calculated for each event and applied as a weight to simulation for 2016 and 2017 samples.
The official tool is used for this purpose.%~\cite{L1PrefiringTwiki}.

\begin{figure}
\subfigure [L1 pre-firing weights]       {\resizebox{.5\textwidth}{!}{\includegraphics[width=.5\textwidth]{Figures/L1Prefiring_ZZGTo4LG.png}}}
\subfigure [Effect on $m_{4\ell\gamma}$] {\resizebox{.5\textwidth}{!}{\includegraphics[width=.5\textwidth]{Figures/SYS/SR4P/ZZGTo4LG_mZZGloose_L1Prefiring.png}}}
\caption{L1 pre-firing weights and effect of their application on the signal MC in the region SR4P\_P in 2018. The photon is required to pass the Loose cut-based ID}
\label{fig:L1Prefiring}
\end{figure}

The Fig~\ref{fig:L1Prefiring} shows the impact of the L1 pre-firing weights on the signal MC.
The impact on the normalization of the signal is around 0.8\%.

\paragraph{Removal of noisy jets}

Increased jet multiplicity was reported for 2017 data, creating ``horns'' in the jet $\eta$ distribution for $2.5<|\eta_{jet}|<3$.
The issue was linked to an increase of the ECAL noise, PU and bunch-crossing dependent, thus getting worse as luminosity increases.
The problem can only be fixed in the UL ReReco.
For now, we checked the impact of rejecting jets with raw $p_T<50$ GeV in 2.65 $<|\eta| <$ 3.139.
As we see no significant impact in the data/MC agreement, we decided not to use these cuts.

%\paragraph{HEM 15/16 failures}

%Following	a CMS-wide power interlock on June 30, the power-on of CAEN A3100HBP modules that provide low-voltage power to the on-detector HE front-end electronics led to irreversible damage of two sectors on the HE minus side, HEM15 and HEM16 (FIXME: add ref). No significant impact was seen and nothing particular is done to cope with this. 

%https://twiki.cern.ch/twiki/bin/view/CMS/HIGJetMET#Known_JetMET_issues 






\subsection{Photons}
\label{sec:photons}
The PF reconstruction of isolated photons is conducted together with electron reconstruction.
This is because the large amount of material in the tracker makes electron emit bremsstrahlung photons, which may convert to \Pep \Pem pairs,
which may in turn produce bremsstrahlung, and so on.

Photon candidates are seeded by ECAL superclusters (SC) with $\ET > 10 \GeV$ with no link to a GSF track.
As for the electron candidates, the energy sum of the HCAL cells within $\DR = 0.15$ must not exceed 10 \% of the supercluster energy.
All ECAL clusters linked to the supercluster are associated with the candidate.

To correct for the energy missed in the association process,
a correction is applied to the total energy of the ECAL clusters with analytical functions of the energy and pseudorapidity, which can be as large as 25 \% at low \pt and maximum tracker thickness ($|\eta| \approx 1.5$).
The direction of the photon is taken to be that of the supercluster.

Photon candidates are retained if they are isolated from other tracks and calorimeter clusters in the event,
and if the ECAL cell energy distribution and the ratio between the HCAL and ECAL energies are compatible with those expected from a photon shower.
The PF selection is looser than the requirements applied at analysis level to select isolated photons.


\section{Identification and selction}
\subsection{Primary Vertex}
The pp collision vertices in an event are reconstructed by grouping tracks consistent with originating at a common point in the luminous region.
The candidate vertex with the largest value of summed physics-object $\pt^2$ is taken to be the primary pp interaction vertex.
The physics objects are the jets, clustered using the \antikt jet finding algorithm~\cite{Cacciari:2008gp, Cacciari:2011ma} with all the tracks assigned to candidate vertices as inputs,
and the associated missing transverse momentum, taken as the negative vector \pt sum of those jets.


\subsection{Electrons}
\subsubsection{Isolation Optimization}
\label{sec:eleiso}
The electron isolation is archieved through the use of the Particle Flow relative isolation,
which is defined as:
\begin{equation}
\text{RelPFiso} = (\sum_{\text{charged}} \ET + \sum^{\text{corr}}_{\text{neutral}} \ET)/\ET^{\text{e}}
\label{eqn:elepfrelisoeqn}
\end{equation} 
where the corrected neutral component of isolation is then computed using the formula:
\begin{equation}
\label{eqn:neutralea}
  \sum^{\text{corr}}_{\text{neutral}} \ET = \text{max} \left( \sum^{\text{uncorr}}_{\text{neutral}} \ET - \ET^{PU},\, 0 \GeV \right)
\end{equation}
and the mean pile-up contribution to the isolation cone is obtained as :
\begin{equation}
  \ET^{PU} =  \rho \times A_\text{eff}
\label{eqn:purho}
\end{equation}
where $\rho$ is the mean energy density in the event and the effective area $A_{eff}$ is defined as the ratio
between the slope of the average isolation and that of $\rho$ as a function of the number of vertices.

%% was optimized in Ref.~\cite{AN-15-277} and the electron isolation working was
The threshold for the electron isolation,
calculated within a cone of radius $\DR = 0.3$
is chosen to be $\text{RelPFiso} < 0.35$. 

\subsubsection{Impact Parameter Selection}
\label{sec:eleSIP}
In order to ensure that the electron trajectories are consistent with a common primary vertex
they are required to have an associated track with a small impact parameter with respect to the event primary vertex.
The significance of the impact parameter (SIP) is used:
\begin{equation}
\label{eq:SIP3D}
\SIPthreeD \mathdefined \frac{|\rm IP_{3D}|}{\sigma_{\rm IP}} \ ,
\end{equation}
where ${\rm IP_{3D}}$ is the lepton impact parameter in three dimensions,
that is the distance with respect to the primary interaction vertex the point of closest approach,
and $\sigma_{\rm IP}$ the associated uncertainty.
Electrons for the analysis must satisfy $\SIPthreeD < 4$.

\subsubsection{Electron MVA}
\label{sec:eleMVA}
Reconstructed electrons are identified and isolated by means of a Gradient Boosted Decision Tree (GBDT) multivariate classifier algorithm,
which exploits observables from the electromagnetic cluster, the matching between the cluster and the electron track, observables based exclusively on tracking measurements as well as particle flow isolation sums.
It was developed for the $\PH \to \PZ \PZ^{*} \to 4 \Pl$ analysis~\cite{CMS-PAS-HIG-19-001}
and trained separately for each data taking year on a Drell-Yan plus jets MC sample.
The classifier is trained with the e\textbf{X}treme \textbf{G}radient \textbf{Boost}ing (XGBoost) optimized distributed gradient boosting library~\cite{Chen_2016}
designed to be highly efficient, flexible and portable.

% The MVA values are userFloat("ElectronMVAEstimatorRun2Summer16ULIdIsoValues"), and the cut is done in ZZAnalysis/AnalysisStep/plugins/EleFiller.cc
The full list of observables used can be found in the Table~\ref{tab:ele_ID_input_variables}.

\begin{table}[ht]
  \caption{Overview of input variables to the identification classifier. Variables not used in the Run 2 MVA are marked with  $(\mathord{\cdot})$.}
  \label{tab:ele_ID_input_variables}
  \small
  \centering
  \begin{tabular}{c l}
    \toprule
    Observable type & Observable name \\
    \midrule
    \multirow{6}{*}{Cluster shape}
      & RMS of the energy-crystal number spectrum: $\sigma_{i\eta i\eta}$, $\sigma_{i\varphi i\varphi}$ \\
      & Super cluster width along $\eta$ and $\phi$ \\
      & Ratio of the hadronic energy behind the SC to the SC energy, $H/E$ \\
      & Circularity $(E_{5\times5} - E_{5\times1})/E_{5\times5}$ \\
      & Sum of the seed and adjacent crystal over the SC energy $R_{9}$ \\
      & For endcap training bins: energy fraction in pre-shower $E_\text{PS}/E_\text{raw}$ \\
    \hline
    \multirow{2}{*}{Track-cluster match}
      & Energy-momentum agreement $E_{tot}/p_{in}$, $E_{ele}/p_{out}$, $1/E_{tot} - 1/p_{in}$ \\
      & Position matching $\Delta\eta_{in}$, $\Delta\varphi_{in}$, $\Delta\eta_{seed}$ \\
    \hline
    \multirow{5}{*}{Tracking}
      & Fractional momentum loss $f_{brem} = 1 - p_{out}/p_{in}$ \\
      & Number of hits of the KF and GSF track $N_{KF}$, $N_{GSF}$ $(\mathord{\cdot})$ \\
      & Reduced $\chi^2$ of the KF and GSF track $\chi^{2}_{KF}$, $\chi^{2}_{\textrm{GSF}}$ \\
      & Number of expected but missing inner hits $(\mathord{\cdot})$ \\
      & Probability transform of conversion vertex fit $\chi^2$ $(\mathord{\cdot})$ \\
    \hline
    \multirow{3}{*}{Isolation}
      & Particle Flow photon isolation sum $(\mathord{\cdot})$ \\
      & Particle Flow charged hadrons isolation sum $(\mathord{\cdot})$ \\
      & Particle Flow neutral hadrons isolation sum $(\mathord{\cdot})$ \\
    \hline
    \multirow{1}{*}{For PU-resilience}
      & Mean energy density in the event: $\rho$ $(\mathord{\cdot})$ \\
    \bottomrule
  \end{tabular}
\end{table}


The model is trained on 2016, 2017, and 2018 Drell-Yan with jets MC sample for both signal and background. The separate training for three periods guarantees
optimal performance during the entire \RunII data taking period.


Tables~\ref{tab:ele_ID_WPA}, \ref{tab:ele_ID_WPB} and~\ref{tab:ele_ID_WPC} list the cuts values applied to the MVA output for 2016, 2017, 2018 training, respectively.
For 2018, the corresponding signal and background efficiencies are given as examples.
They are very similar for 2016 and 2017.

For the analysis, loose electrons have to pass this MVA identification and isolation working point.

\begin{table}
  \caption{Minimum BDT score required for passing the electron identification, for 2016 samples.}
  \label{tab:ele_ID_WPA}
  \centering
  \begin{tabular}{c c c c}
    \toprule    %----------------------------------------------------------------------------------------
    \pt range           & $|\eta| < 0.8$ & $0.8 < |\eta| < 1.479$ & $|\eta| > 1.479$ \\
    \midrule    %----------------------------------------------------------------------------------------
    $5 < \pt < 10 \GeV$ &  0.9503        &  0.9461                &  0.9387 \\
    $\pt > 10 \GeV$     &  0.3782        &  0.3587                & -0.5745 \\
    \bottomrule %----------------------------------------------------------------------------------------
  \end{tabular}
\end{table}

\begin{table}
  \caption{Minimum BDT score required for passing the electron identification, for 2017 samples.}
  \label{tab:ele_ID_WPB}
  \centering
  \begin{tabular}{c c c c}
    \toprule    %----------------------------------------------------------------------------------------
    \pt range           & $|\eta| < 0.8$ & $0.8 < |\eta| < 1.479$ & $|\eta| > 1.479$ \\
    \midrule    %----------------------------------------------------------------------------------------
    $5 < \pt < 10 \GeV$ &  0.8521        &  0.8268                &  0.8694 \\
    $\pt > 10 \GeV$     &  0.9825        &  0.9692                &  0.7935 \\
    \bottomrule %----------------------------------------------------------------------------------------
  \end{tabular}
\end{table}

\begin{table}
  \caption{Minimum BDT score required for passing the electron identification and corresponding signal and background efficiencies, for 2018 samples.}
  \label{tab:ele_ID_WPC}
  \centering
  \begin{tabular}{c c c c c}
    \toprule
    $|\eta|$ range                      & \pt range           & Cut on BDT & Signal eff. & Background eff. \\
    \midrule
    \multirow{2}{*}{$|\eta| < 0.8 $}    & $5 < \pt < 10 \GeV$ &  0.8956    &  81.0\,\%   &  4.4\,\% \\
                                        & $\pt > 10 \GeV$     &  0.0424    &  97.1\,\%   &  2.9\,\% \\
    \hline
    \multirow{2}{*}{$0.8<|\eta|<1.479$} & $5 < \pt < 10 \GeV$ &  0.9111    &  79.3\,\%   &  4.6\,\% \\
                                        & $\pt > 10 \GeV$     &  0.0047    &  96.3\,\%   &  3.6\,\% \\
    \hline
    \multirow{2}{*}{$|\eta| > 1.479$}   & $5 < \pt < 10 \GeV$ &  0.9401    & 73.0\,\%    &  3.6\,\% \\
                                        & $\pt > 10 \GeV$     & -0.6042    & 95.7\,\%    &  6.7\,\% \\
    \bottomrule
  \end{tabular}
\end{table}

\subsubsection{Analysis selection for electrons}%% {Electron Selection}
\label{sec:ele_selection}
Electron candidates are preselected using loose cuts on track-cluster matching observables, so as to preserve the highest possible efficiency while rejecting part of the QCD background. To be considered for the analysis, electrons are required to have a
transverse momentum $p^e_T > 7 \GeV$,
a reconstructed pseudorapidity $|\eta^e| <$ 2.5.
They must satisfy a loose primary vertex
constraint defined as $d_{xy} < 0.5$ cm and $d_z < 1\cm$,
in order to suppress electrons from photon conversions, since their tracks do not point to the primary vertex.
Such electrons are called {\bf loose electrons}.

Additionally,
electrons that pass the cut on the \SIPthreeD and the multivariate identification described in
Sections~\ref{sec:eleSIP} and \ref{sec:eleMVA} respectively
are defined {\bf tight electrons}.
This is the selection used by the analysis.

The data-MC discrepancy is corrected using scale factors as is done for the electron selection with data efficiencies measured using the same tag-and-probe technique outlined later (see Section~\ref{sec:eleEffMeas}).
These studies for reconstructions are carried out by the CMS Collaboration and the results are summarized here.
The electron reconstruction scale factors
% are shown Fig.~\ref{fig:ele_rec_scale_factors} and
are applied as a function of the super cluster $\eta$ and electron $\pt$.


\subsection{Muons}
\subsubsection{Muon Isolation}
\label{sec:muoniso}
A Particle Flow based isolation is used to suppress the contamination from muon from hadronic decays inside jets.
The so-called $\Delta\beta$ correction is applied in order to subtract the \pileup{} contribution for the muons, 
whereby $\Delta\beta = \frac{1}{2} \sum^\text{charged had.}_\text{PU} \pt$
gives an estimate of the energy deposit of neutral particles (hadrons and photons) from \pileup{} vertices.

The relative isolation for muons is then defined as:
\begin{equation}
\text{RelPFIso} = \frac{1}{\pt^\text{muon}} \left( \sum_\text{charged had.} \pt + \max(0, \sum_\text{neutral had.} \ET + \sum_\text{photon} \ET - \Delta \beta) \right)
\label{eqn:mupfiso}
\end{equation}

where the sums run over the photons, charged and neutral hadrons in a cone with $\DR = 0.3$ around the muon.
Only charged hadrons originating from the primary vertex are included to minimise the \pileup{} contribution.

The isolation cone for muons was optimised and the working point was chosen to be $\text{RelPFiso}(\Delta R = 0.3) < 0.35$. 

Similarly to electrons, a condition on the significance of the 3D impact parameter (\SIPthreeD, see Equation \ref{eq:SIP3D}) is applied,
in order to ensure that muons are consistent with the primary vertex.
Muons are required to satisfy $\SIPthreeD < 4$.

\subsubsection{Muon Identification}
%More details on muon reconstruction can be found in Ref.~\cite{AN-15-277}.
Muon candidates are selected among global muons and tracker muons.
Standalone muon tracks that are only reconstructed in the muon system are not used.

Muons are required to have transverse momentum $\pt > 5 \GeV$ and pseudorapidity $|\eta| < 2.4$.
As for the electrons, the muon track is required to be compatible with the primary vertex,
to suppress contributions from in-flight decays of hadrons, pileup and cosmic rays.
This requirement translates into $d_{xy} < 0.5 \cm$, $d_z < 1 \cm$, where $d_{xy}$ and $d_z$ are
the muon impact parameters with respect to the primary vertex in the transverse plane and longitudinal direction respectively.
These requirements correspond to the analysis definition of \textbf{loose muons}.

Loose muons with \pt below 200\GeV that also pass
the PF loose muon ID \cite{ParticleFlow} are considered \textbf{identified muons} for this analysis.
Muons with \pt $>$ 200\GeV must pass either the PF identification or the Tracker High-\pt ID,
whose definition is reported in Table \ref{tab:highPtID}.
This relaxed definition is used to increase signal efficiency in the high
centre of mass energy regime.
In the laboratory frame of reference, the leptons coming from the decay of
a highly boosted $\cPZ$ will be nearly collinear, and the PF ID loses 
efficiency for muons separated by approximately $\Delta R < 0.4$, which roughly 
corresponds to muons originating from $\cPZ$ bosons with $\pt > 500\GeV$.

\begin{table}
    \begin{small}
    \begin{center}
    \caption{
      The requirements for a muon to pass the Tracker High-$\pt$ ID. Note that
      these are equivalent to the Muon POG High-$\pt$ ID with the global track 
      requirements removed.
      }
    \begin{tabular}{|l|l|}
      \hline
      Plain-text description         & Technical description                 \\
      \hline
      Muon station matching          & Muon is matched to segments           \\
                                     & in at least two muon stations         \\
                                     & \textbf{NB: this implies the muon is} \\
                                     & \textbf{an arbitrated tracker muon.}  \\
      \hline                                                          
      Good $\pt$ measurement         & $\pt / \sigma_{\pt} < 0.3$            \\
      \hline
      Vertex compatibility ($x-y$)   & $d_{xy} < 2$~mm                       \\
      \hline
      Vertex compatibility ($z$)     & $d_{z} < 5$~mm                        \\
      \hline
      Pixel hits                     & At least one pixel hit                \\
      \hline
      Tracker hits                   & Hits in at least six tracker layers   \\
      \hline
    \end{tabular}
    \label{tab:highPtID}
    \end{center}
    \end{small}
\end{table}

An additional \textit{ghost-cleaning} step is performed to deal with situations when a single muon
can be incorrectly reconstructed as two or more muons:

\begin{itemize}

\item Tracker Muons that are not Global Muons are required to be matched to reconstructed segments in at least two stations of the muon system;
\item If two muons are sharing 50\% or more of their segments then the muon with lower quality is removed.

\end{itemize}


Loose muons that pass also the identification, isolation and \SIPthreeD requirements are defined \textbf{tight muons}.

\subsection{Photon Identification}
\label{sec:photonID}
Photon candidates are required to have $E_{T} > 20 \GeV$ and be in the fiducial Barrel region or Endcap regions,
defined by $|\eta|<1.4442$ and $1.566<|\eta|<2.5$, respectively.
The photon candidates are also required to be separated from the closest lepton (electron or muon) by at least $\DR(\PGg, \Pl) > 0.5$,
which highly suppresses the contribution from FSR,
and is orthogonal to the selection used for lepton FSR recovery (see Section \ref{sec:FSRphotons}).

Different strategies are used to identify prompt (produced at the primary vertex) and isolated
electrons and photons, and separate them from background sources.
The most important background to prompt photons arises from jets fragmenting mainly into light neutral mesons
such as \Pgpz or \PGh, which promptly decay to two photons.
For the energy range of interest, the meson is significantly boosted, such that the two photons from the decay are nearly collinear
and are difficult to distinguish from a single-photon incident on the calorimeter.
%% Different working points are defined to identify either electrons or photons,
%% corresponding to identification efficiencies of approximately 70, 80, and 90 \%, respectively.
%% In all cases data and simulation efficiencies are compatible within 1-5 \% over the full $\eta$ and \ET ranges for electrons and photons.

Both a cut-based and a MVA-based ID are employed and compared in the analysis.
Each has its own advantages and disadvantages.
Due to its nature, the cut-based ID can be easily inverted by reversing only one or a few of its cuts.
On the other hand, MVA-based ID is more effective at discriminating between signal and background, providing a higher signal-to-noise ratio.

\paragraph{Identification variables\\}
One of the most efficient ways to reject photon backgrounds is the use of isolation energy sums,
a generic class of discriminating variables that are constructed from the sum of the reconstructed energy in a cone around photons in different subdetectors.
%% For this purpose, it is convenient to define cones in terms of an $\eta-\phi$ metric;
%% the distance with respect to the reconstructed photon direction is defined by \DR.
A veto region inside the cone is defined, to ensure that the energy from the photon itself is not included in this sum.

Photon isolation exploits the information provided by the PF event reconstruction (Section \ref{sec:ParticleFlow}).
The isolation variables are obtained by summing the transverse momenta of charged hadrons ($I_{ch}$), photons ($I_\PGg$), and neutral hadrons ($I_n$),
inside an isolation cone of $\DR = 0.3$ with respect to the electron or photon direction.
The larger the energy of the incoming electrons or photons, the larger the amount of energy spread around its direction in the various subdetectors.
For this reason, the thresholds applied on the isolation quantities are frequently parameterised as a function of the particle \ET.

The isolation variables are corrected to mitigate the contribution from pileup.
This contribution in the isolation region is estimated as $\rho A_{eff}$,
where $\rho$ is the median of the transverse energy density per unit area in the event
and $A_{eff}$ is the area of the isolation region weighted by a factor that accounts for the dependence of the pileup transverse energy density on the object $\eta$ \cite{CMS:electron-performance-2015}.
The quantity $\rho A_{eff}$ is subtracted from the isolation quantities.

The distributions of $I_\PGg$ before % after?
the $\rho$ corrections are shown in Figure~\ref{fig:Iph_CR2P2F} for photons in the EB and EE.

\begin{figure}
\subfigure [Barrel] {\includegraphics[width=.5\textwidth]{Figures/VVGammaAnalyzer_noLFR/Run2/fullMC/CR2P2F/kinPh_phIso_EB_pow.pdf}}%
\subfigure [Endcap] {\includegraphics[width=.5\textwidth]{Figures/VVGammaAnalyzer_noLFR/Run2/fullMC/CR2P2F/kinPh_phIso_EE_pow.pdf}}
\caption{The PF photon isolation ($I_\PGg$), before the $\rho$ correction, in a cone defined by $\DR = 0.3$ for photons in the EB (left) and in the EE (right).
The events belong to the leptonic control region CR2P2F (see Section \ref{sec:fake_leptons}).
The lower panels display the ratio of the data to the simulation.}
\label{fig:Iph_CR2P2F}
\end{figure}

Another method to reject jets with high electromagnetic content exploits the shape of the electromagnetic shower in the ECAL.
Even if the two photons from neutral hadron decays inside a jet cannot be fully resolved, a wider shower profile is expected, on average,
compared with a single incident electron or photon.
This is particularly true along the $\eta$ axis of the cluster, since the presence of the material combined with the effect of the magnetic field
reduce the discriminating power resulting from the $\phi$ profile of the shower. 
In particular, the following two variables have high discriminating power.

The hadronic over electromagnetic energy ratio ($H/E$) is defined as the ratio between the energy deposited in the HCAL in a cone of radius $\DR = 0.15$
around the supercluster direction and the energy of the photon candidate.
\sieie, the second moment of the log-weighted distribution of crystal energies in $\eta$,
calculated in the $5 \times 5$ matrix around the most energetic crystal in the SC and rescaled to units of crystal size.
The mathematical expression is given below:
\begin{equation}
\label{eq:sieie}
\sieie \mathdefined \sqrt{ \frac{\sum_i^{5 \times 5} w_i(\eta_i - \bar{\eta}_{5 \times 5})}{\sum_i^{5 \times 5} w_i} }
\end{equation}
where $\eta_i$ is the pseudorapidity of the i-th crystal,
$\bar{\eta}_{5 \times 5}$ is the mean pseudorapidity of the $5 \times 5$ cells
and $w_i$ is a weight defined as $w_i = \mathrm{max}(0,\, 4.7 + \mathrm{ln}(E_i/E_{5 \times 5}))$,
which is nonzero if $E_i > 0.9\, \%\; E_{5 \times 5}$.
Looking at the numerator in Equation \ref{eq:sieie}, it is clear that \sieie is proportional to the distance between adjacent crystals,
which is 0.0175 in EB and varies from 0.0175 to 0.0 in EE.
Therefore, the spread of \sieie in EE is twice the one in EB.
The \sieie distribution is expected to be narrow for isolated electrons or photons, and broad for two-photon showers from meson decays.
The distributions of \sieie are shown in Figure \ref{fig:sieie_CR2P2F} for photons in the EB and EE.

\begin{figure}
\subfigure [Barrel] {\includegraphics[width=.5\textwidth]{Figures/VVGammaAnalyzer_noLFR/Run2/fullMC/CR2P2F/kinPh_sieie_EB_pow.pdf}}%
\subfigure [Endcap] {\includegraphics[width=.5\textwidth]{Figures/VVGammaAnalyzer_noLFR/Run2/fullMC/CR2P2F/kinPh_sieie_EE_pow.pdf}}
\caption{Distribution of \sieie for photons in the EB (left) and in the EE (right).
The events belong to the leptonic control region CR2P2F (see Section \ref{sec:fake_leptons}).
The lower panels display the ratio of the data to the simulation.}
\label{fig:sieie_CR2P2F}
\end{figure}

Another important variable is $R_9$, which is defined as the ratio between the energy contained in the $3 \times 3$ array of crystals,
centered around the most energetic crystal of the SC,
to the total energy of the supercluster.

\paragraph{Cut-based ID\\}
The cut-based photon ID~\cite{CMS:EGM-17-001} uses 5 variables: $H/E$, \sieie, $I_{ch}^{corr}$, $I_{n} ^{corr}$ and $I_{\PGg} ^{corr}$.
The Loose working point, shown in Table~\ref{tab:VPhotonID}, is selected for this analysis.
It provides 90\,\% efficiency on signal photons with 86\,\% (77\,\%) background rejection in the Barrel (Endcap).
The efficiency in data is well modelled in simulation, and the ratio between the two, shown in Figure \ref{fig:phEffSF}, is within 5\,\% from unity.
This ratio is referred as Scale Factor (SF), and is applied to simulation to correct for the residual mismodelling and avoid biasing the results.

\begin{table}
  \centering
  \renewcommand{\arraystretch}{1.4}
  \begin{tabular}{c c c}
    \toprule
    Variable                 &  Barrel $\quad |\eta| < 1.4442$     & Endcap $\quad 1.566 < |\eta| < 2.5$\\
    \midrule
    $H/E$                    & $0.04596$                           & $0.0590$                           \\
    $\sigma_{i\eta i\eta}$   & $0.0106$                            & $0.0272$                           \\
    $I_{ch}^{corr} [\GeV]$   & $1.694$                             & $2.089$                            \\
    $I_{n} ^{corr} [\GeV]$   & \renewcommand{\arraystretch}{1}\begin{tabular}{c} $24.032 + 0.01512\, p_{T}^{\gamma} +$\\$+ 2.259 \cdot 10^{-5}\, (p_{T}^{\gamma})^2$ \end{tabular}
                             & \renewcommand{\arraystretch}{1}\begin{tabular}{c} $19.722 + 0.0117\, p_{T}^{\gamma} +$ \\$+ 2.3 \cdot 10^{-5}\, (p_{T}^{\gamma})^2$   \end{tabular}\\
    $I_{\PGg}^{corr} [\GeV]$ & $2.876 + 0.004017\, p_{T}^{\gamma}$ & $4.162 + 0.0037\, p_{T}^{\gamma}$  \\
    \bottomrule
  \end{tabular}
  \caption[.]{Cut thresholds of the Loose working point cut-based photon ID.}
  \label{tab:VPhotonID}
\end{table}

\begin{figure}
  \subfigure [2016preVFP ] {\resizebox{.5\textwidth}{!}{\includegraphics[width=.5\textwidth]{SF/phEffSF_2016preVFP.pdf} }}
  \subfigure [2016postVFP] {\resizebox{.5\textwidth}{!}{\includegraphics[width=.5\textwidth]{SF/phEffSF_2016postVFP.pdf}}}\\
  \subfigure [2017]        {\resizebox{.5\textwidth}{!}{\includegraphics[width=.5\textwidth]{SF/phEffSF_2017.pdf}}}
  \subfigure [2018]        {\resizebox{.5\textwidth}{!}{\includegraphics[width=.5\textwidth]{SF/phEffSF_2018.pdf}}}
  \caption{Photon efficiency scale factors for the cut-based Loose ID.}
  \label{fig:phEffSF}
\end{figure}

Besides the identification working points, an electron veto selection (CSEV veto) is also applied.
%% The scale factors are shown in ~\ref{tab:eleveto_SFs}.

%% \begin{table}[htbp]
%%  \centering
%%    \begin{tabular}{|c|c|l|l|}
%%    \hline
%%    Year & $p_T$& barrel & endcap\\ \hline
%%    2016 & inclusive &0.9938 $\pm$ 0.0119 & 0.9875 $\pm$ 0.0044\\\hline
%%    2017 & inclusive & 0.9862 $\pm$ 0.0030 & 0.9638 $\pm$ 0.0047\\\hline
%%    \multirow{3}{*}{2018} &10 GeV$<p_{T}^{\gamma}<30$ GeV &0.9869 $\pm$ 0.0043& 0.9535 $\pm$ 0.0054\\
%%    & 30 GeV$<p_{T}^{\gamma}<$60 GeV  &0.9908  $\pm$ 0.0111 & 0.9646 $\pm$ 0.0076\\
%%    & 60 GeV$<p_{T}^{\gamma}<$200 GeV &1.0084  $\pm$ 0.0856& 1.0218 $\pm$ 0.1178\\
%%    \hline
%%    \end{tabular}
%%    \caption{Electron veto scale factors for barrel and endcap corresponding to 2016 to 2018.}
%%    \label{tab:eleveto_SFs}
%%  \end{table}

%\begin{figure}[b]
%  \begin{center}
%    \includegraphics[width=0.8\textwidth]{figs/photon_SFs.pdf}
%    \caption{Photon ID scale factors for cut-based loose Photon selection}
%    \label{fig:PhotonEff}
%  \end{center}
%\end{figure}

\paragraph{Multivariate ID\\}
A more sophisticated photon identification strategy, is based on a multivariate technique, based on a Boosted Decision Tree (BDT) \cite{CMS:EGM-17-001}.
This multivariate (MVA) discriminant is built based on 14 input variables,
and provides excellent separation between signal (prompt photons) and background from misidentified jets.
The signal is defined as reconstructed photons from a \PGg + jets simulated sample that are matched at generator level with prompt photons within a cone of size $\DR = 0.1$,
whereas the background is defined by reconstructed photons in the same sample that do not match with a generated photon.
Photon candidates with $\ET > 15 \GeV$, $|\eta| < 2.5$, and satisfying very loose preselection requirements are used for the training of the BDT.
The variables used include the energy and shower shape of the supercluster, as well as various isolations.
Some of these variables are also employed by the cut-based ID, so the two discriminants are not independent.
%% The full list is shown in Table~\ref{tab:MVAvariables}.
%% The version used is \texttt{RunIIFall17v2}.

%% \begin{table}[ht]
%% \caption[.]{Variables used by the MVA-based ID, version \texttt{RunIIFall17v2}}
%% \label{tab:MVAvariables}
%% \centering
%% \begin{tabular}{l|l}
%% Name & variable\\
%% \hline
%% SCRawE             & superCluster.rawEnergy                                               \\
%% r9                 & r9                                                                   \\
%% sigmaIetaIeta      & full5x5\_showerShapeVariables.sigmaIetaIeta                          \\
%% etaWidth           & superCluster.etaWidth                                                \\
%% phiWidth           & superCluster.phiWidth                                                \\
%% covIEtaIPhi        & full5x5\_showerShapeVariables.sigmaIetaIphi                          \\
%% s4                 & full5x5\_showerShapeVariables.e2x2/full5x5\_showerShapeVariables.e5x5\\
%% scEta              & superCluster.eta                                                     \\
%% rho                & fixedGridRhoAll                                                      \\
%% esEffSigmaRR       & full5x5\_showerShapeVariables.effSigmaRR                             \\
%% esEnergyOverRawE   & superCluster.preshowerEnergy/superCluster.rawEnergy                  \\
%% phoIso03           & photonIso                                                            \\
%% chgIsoWrtChosenVtx & chargedHadronIso                                                     \\
%% chgIsoWrtWorstVtx  & chargedHadronWorstVtxIso                                             \\
%% \end{tabular}
%% \end{table}

Two working points are provided centrally by the the dedicated group within CMS:
\texttt{wp90} and \texttt{wp80}, corresponding to 90 \% and 80 \% prompt photon efficiency respectively.
The cuts on the MVA estimator value that define the two working points are detailed in Table~\ref{tab:MVAwpCuts}.

\begin{table}[ht]
\caption[.]{Working points of the photon MVA-based ID, version \texttt{RunIIFall17v2}}
\label{tab:MVAwpCuts}
\centering
\begin{tabular}{lrr}
\toprule
Name & Barrel & Endcap \\
\midrule
\texttt{wp80} & -0.02 & -0.26 \\
\texttt{wp90} &  0.42 &  0.14 \\
\bottomrule
\end{tabular}
\end{table}

As with any other ID, to account for difference in the modelling of its efficiency in simulations and the true efficiency in data,
Scale Factors are measured for each year in different bins of \pt and $\eta$ for \texttt{wp80} (Figure \ref{fig:phEffMVASF_wp80}) and \texttt{wp90} (Figure \ref{fig:phEffMVASF_wp90}).

\begin{figure}
\centering
\subfigure [2016preVFP ] {\includegraphics[width=.5\textwidth]{SF/2016_PhotonsMVAwp80_SF2D.pdf}}%
\subfigure [2017]        {\includegraphics[width=.5\textwidth]{SF/2017_PhotonsMVAwp80_SF2D.pdf}}\\
\subfigure [2018]        {\includegraphics[width=.5\textwidth]{SF/2018_PhotonsMVAwp80_SF2D.pdf}}
\caption{Photon efficiency scale factors for the MVA-based ID with working point \texttt{wp80}.}
\label{fig:phEffMVASF_wp80}
%% }
\end{figure}

\begin{figure}
\centering
\subfigure [2016preVFP ] {\includegraphics[width=.5\textwidth]{SF/2016_PhotonsMVAwp90_SF2D.pdf}}%
\subfigure [2017]        {\includegraphics[width=.5\textwidth]{SF/2017_PhotonsMVAwp90_SF2D.pdf}}\\
\subfigure [2018]        {\includegraphics[width=.5\textwidth]{SF/2018_PhotonsMVAwp90_SF2D.pdf}}
\caption{Photon efficiency scale factors for the MVA-based ID with working point \texttt{wp90}.}
\label{fig:phEffMVASF_wp90}
\end{figure}

A comparison between the cut-based and the MVA-based IDs is shown in Figure~\ref{fig:PhotonIsoAUC}.
It represent the background rejection as a function of the signal efficiency for the raw MVA discriminant
and the three working points of the cut-based ID.

\begin{figure}
\centering
\includegraphics[width=.5\textwidth]{PhotonIsoAUC.png}
\caption{Performance of the photon BDT and cut-based identification algorithms in 2017.
Three different working points (loose, medium and tight) are shown for the cut-based ID. \cite{CMS:photon-performance-2015}}
\label{fig:PhotonIsoAUC}
\end{figure}

\paragraph{Selections used in the analysis\\}
\label{sec:photon_selection}

A common base selection is applied to all the photons, consisting of
\pt > 20 \GeV and
$|\eta| < 1.4442$ OR $1.566 < |\eta| < 2.4$.
Photons must also pass a conversion safe electron veto (CSEV) \cite{CMS:photon-performance-2015} which requires the absence of charged particle tracks,
with a hit in the innermost layer of the pixel detector not matched to a reconstructed conversion vertex, pointing to the photon cluster in the ECAL.
Photons are also rejected in the presence of a seed in the pixel tracker consisting of at least two hits,
which points to the ECAL within some window defined around the photon SC position.
Photons that pass these selections are called \textbf{kinematic photons}, which is the loosest selection used in this analysis.

Photons that also pass the cut thresholds of the Loose cut-based ID (see Table~\ref{tab:VPhotonID}) for H/E, the neutral ($I_n$) and photon ($I_\PGg$) isolations
are defined \textbf{very loose} photons.
They are used as the loose working point in a tight-to-loose method (Section \ref{sec:fake_photons_background}) to estimate the \nonprompt photon background.

The photons that also pass the \sieie and charged isolation ($I_{ch}$), and thus the Loose cut-based ID, are the \textbf{loose photons},
which is the tight selection for the \nonprompt photon background,
and the analysis selection for the results derived using the cut-based ID.

Additionally, kinematic photons that pass the MVA-based working points wp90 and wp80 are also considered,
given the improved performance of the MVA ID over the cut-based one.
The drawback of this ID is that it is not possible to do a data-driven estimation of the \nonprompt background,
due to the limited size of the application region defined by requiring a photon that passes the wp90 but fails the wp80,
as explained in Section~\ref{sec:fake_photons_background}.


\subsection{Jets}
\todo{Move here jet selection}

\subsection{Summary}
The requirements on all objects used for the analysis with 2016, 2017 or 2018 data are summarized in the Table~\ref{tab:objsummary}.
In addition, a ``ghost-cleaning'' procedure is applied to the muons, as described in Section~\ref{sec:muonReco}.

A lepton is declared {\bf loose} if it passes the reconstruction, kinematics and dxy/dz cuts and declared {\bf tight} if it passes in addition the identification, isolation and SIP3D requirements. 

\begin{table}
  \centering
  \caption{Summary of physics object selection for the analysis.}
  \label{tab:objsummary}
  %% \resizebox{\textwidth}{!}{
    \begin{tabular}{l l}
      \toprule
      \multirow{4}{*}{Electrons}
        & $\pt^e > 7 \GeV$   \quad $\abs{\eta^e} < 2.5$ \\
        & $d_{xy} < 0.5 \cm$ \quad $d_{z} < 1 \cm$      \\
        & $\SIPthreeD < 4$                              \\
        & MVA ID with isolation cuts from Section~\ref{sec:eleMVA}\\
      \midrule
      \multirow{5}{*}{Muons}
        & Global or Tracker Muon                                 \\
        %% & Discard muons with muonBestTrackType==2 even if they are global or tracker muons \\
        & $\pt^{\mu} > 5 \GeV$ \quad $\abs{\eta^{\mu}} < 2.4$    \\
        & $d_{xy} < 0.5 \cm$   \quad $d_{z} < 1 \cm$             \\
        & $\SIPthreeD < 4$                                       \\
        & BDT with ID, isolation from Section~\ref{sec:muo_selection}\\
      \midrule
      \multirow{4}{*}{FSR photons}
        & $\pt^{\PGg} > 2 \GeV$ \quad $\abs{\eta^{\PGg}} < 2.4$     \\
        & ${\cal I}_{\mathrm{PF}}^{\PGg} < 1.8$                     \\
        & $\DR(\ell,\,\PGg) < 0.5$                                  \\
        & $\frac{\DR(\Pl,\,\PGg)}{(\pt^{\PGg})^2} < 0.012 \GeV^{-2}$\\
        \noalign{\vspace{.3ex}} % small vertical space
      \midrule
      \multirow{5}{*}{Signal photons}
        & $\pt^{\PGg} > 20 \GeV$                        \\
        & $\abs{\eta^{\PGg}} < 2.4 \; \land \abs{\eta^{\PGg}} \notin (1.4442, 1.566) $\\
        & $\DR(\ell,\,\PGg) > 0.5$                      \\
        & Conversion Safe Electron Veto                 \\
        & Cut-based (Loose wp) or MVA (wp90 or wp80) ID \\
      \midrule
      \multirow{5}{*}{Jets}
        & $\pt^{\mathrm{jet}} > 30 \GeV$ \quad $\abs{\eta^{\mathrm{jet}}} < 4.7$ \\
        & $\DR(\ell/\PGg,\, \mathrm{jet}) > 0.4$    \\
        & Cut-based jet ID (tight WP)               \\
        & Jet \pileup{} ID (tight WP)               \\
        & DeepFlavour b-veto (medium WP)            \\
      \bottomrule
    \end{tabular}
  %% }
\end{table}


\section{Additional studies}
\note{Maybe move to next chapter}
\subsection{Electrons}
\subsubsection{Electron Energy Calibrations}
\input{Objects/eleCalib}
\subsubsection{Electron Efficiency Measurements}
\label{sec:eleEffMeas}
Electron efficiencies are evaluated using the Tag-and-Probe method.
The study was performed on the SingleElectron/EGamma dataset for each year separately.

Tag electrons need to satisfy the following quality requirements:
\begin{itemize}
\item trigger matched to single electron trigger
\item $\pt > 30 \GeV$, supercluster (SC) $|\eta| < 2.17$% but on in EB-EE gap ($1.4442<|\eta|<1.566$)
\item the tag and the probe need to have opposite charge.
%\item tight working point of the Spring16 cut-based electron ID
\end{itemize}

For the bin between 7 and 20\GeV, additional criteria are required:
\begin{itemize}
\item the tag has to pass a cut on the MVA score,
\item $\sqrt{2*\MET*\pt^{tag}*(1-cos(\phi_{MET}-\phi_{tag}))} < 45 \GeV$.
\item tag minimum \pt increased to 50\GeV
\item the charge is determined with the so-called selection method, requiring that all three estimates of the electron charge to agree. \todo{explain how electron charge is determined}
\end{itemize}
These cuts help cleaning the background and make the fits more reliable (and thus, the measurement more precise).

Probe electrons only need to be reconstructed as GsfElectron while the FSR recovery algorithm is not applied.

The nominal MC efficiencies are evaluated from the a Drell-Yan sample simulated with \MADGRAPH at Leading Order in QCD.

For the efficiency measurements a template fit is used.
The $m_{ee}$ signal shape of the passing and failing probes is taken from MC and convoluted with a Gaussian.
The data is then fitted with the convoluted MC template and a CMSShape (an Error-function with a one-sided exponential tail).
For the low \pt bins, a Gaussian is added to the signal model for the failing probes.

%\paragraph{Electron selection efficiency measurements}\mbox{}\\
%\label{par:Efficiency_measurements}

The electron selection efficiency is measured as a function of the probe electron $p_{T}$ and its SC $\eta$, and separately for electrons falling in the ECAL gaps.

%Figure \ref{fig:ele_sel_pt_turn_onA},~\ref{fig:ele_sel_pt_turn_onB},~\ref{fig:ele_sel_pt_turn_onC} and~\ref{fig:ele_ele_eta_turn_onA},\ref{fig:ele_ele_eta_turn_onB},~\ref{fig:ele_ele_eta_turn_onC} show the $p_{T}$ and $\eta$ turn-on curves measured in data, for 2016, 2017 and 2018.
% and the final 2D scale factor is shown in Fig.~\ref{fig:ele_sel_scale_factors} together with the systematic uncertainties. %These scale factors are very similar to the ICHEP figures, but more homogenous across $\eta$ and $p_{T}$ because of the higher statistics and the usage of more stable fitting routines in the new T\&P tool.

%\includegraphics[page=2, width=0.4\textwidth]{Figures/Electrons/ErecoEta}\\

Standard practices for the evaluation of Tag-and-Probe uncertainties for efficiency measurements are followed. Specifically, the following were considered:

\begin{itemize}
   \item Variation of the signal shape from the MC shape to an analytic shape (Crystal Ball) fitted to the MC
   \item Variation of the background shape from a CMS-shape to a simple exponential in fits to data
%   \item Variation of the tag selection: tag $p_{T}>$35~GeV and passes MVA-based 8X ID
   \item Using an NLO MC sample for the signal templates
\end{itemize}

The total uncertainty for the measurement of the scale factors is the quadratic sum of the statistical uncertainties returned from the fit and the aforementioned systematic uncertainties.



\subsection{Muons}
\subsubsection{Muon Energy Calibrations}
\note{Remove, maybe?}
Two methods are used to calibrate the muon momentum scale~\cite{CMS-MUO-16-001}.
The magnitudes of the momentum scale corrections are about 0.2\usep\% and 0.3\usep\% in the barrel and endcap, respectively~\cite{CMS-MUO-16-001}.
The uncertainty in the resolution is estimated to be about 5\usep\% of its value for both techniques.

\paragraph{Rochester corrections\\}
In the first method, muon momentum scale is measured in data by fitting
the di-muon mass spectrum
in a two-step process
in $Z \rightarrow \mu\mu$ events~\cite{RochesterMuon}.
%% Three datasets are used:
%% the first is a simulation with realistic detector conditions of $\PGg/\PZ \to \PGm \PGm$ events (realistic);
%% the second is a simulation with a perfectly aligned detector (ideal);
%% the third contains data.

In the first step, the corrections are derived
for bins of charge, $\eta$ and $\phi$
by requiring that the average $<1/\pt^\mu>$
in data and MC to be the same as that in an ideal simulation with a perfectly aligned detector.
This produces a table of uncorrelated corrections.
%% This step is called $<1/\pt^\mu>$ correction.

The second step aims at correcting the residual mismodelling of detector efficiency in $\eta$, $/\phi$.
In order to be independent of the modelling assumptions
for the \pt and $\eta$ distributons of $\PZ \to \mu\mu$ events,
the reconstructed \PZ mass is required to be equal to that in the ideal MC for each bin in $\eta$ and $\phi$.
%% This step is called $\Delta M/M$ tuning.
%% Next \textit{addittive} corrections are extracted by taking the difference between the \PGmp and \PGmm scale corrections,
%% which are caused by misalignments,
%% and \textit{multiplicative} corrections
%% which are caused by mismodelling of the magnetic field.

\paragraph{Corrections with Kalman filter\\}
The second method uses \PJGy and \PGU(1S) events~\cite{CMS-PAS-SMP-14-007}.
The effect of variations of the magnetic field, misalignment, and mismodelling of the material
on the measured muon curvature are calculated.
Their values are extracted by fitting the dimuon mass with a Kalman filter
in several bins of pseudorapidity, for a total of 44 parameters.

\subsubsection{Muon Efficiency Measurements}
\note{Remove, maybe?}
\label{sec:muonEffMeas}
Muon efficiencies are measured with the Tag-and-Probe method performed on
$\cPZ \to \Pgm\Pgm$ and $\JPsi\to\mu\mu$ events in bins of $\pt$ and $\eta$. 
% More
%details on the methodology can be found in Ref.~\cite{CMS_AN_2015-277}. Measurements are extracted using 2018 RunA,B,C,D data while the measurements corresponding to 2016 and 2017 datasets have already been reported in Ref.~\cite{CMS_AN_2016-442} and Ref.~\cite{CMS_AN_2017-342} respectively.
%
The $\Z$ sample is used to measure the muon reconstruction and identification efficiency at high $\pt$,
and the efficiency of the isolation and impact parameter requirements at all $\pt$.
%
The $\JPsi$ sample is used to measure the reconstruction efficiency at low $\pt$,
as it benefits from a better purity in that kinematic regime.
In this case, events are collected using triggers that require a muon with $\pt > 7.5\GeV$
and an additional track (muon) with $\pt > 2 \GeV$ when probing the reconstruction (tracking) efficiency.

Results for the muon reconstruction and identification efficiency for $\pt > 5\GeV$
have been derived by the CMS collaboration.
The probe in this measurement are tracks reconstructed in the inner tracker, and
the passing probes are those that are also reconstructed as a global or tracker muon 
and pass the identification criteria.
%
Results for low \pt muons were derived using \JPsi events, with the same definitions
of probe and passing probes. The systematic uncertainties are estimated by varying the analytical signal and background shape models used to fit 
the dimuon invariant mass. 
% Details on the procedure can be found in Ref.~\cite{AN-15-277}. 
The efficiency and scale 
factors used for low $\pt$ muons are the ones derived using single muon dataset.

For the impact parameter requirements, the measurement is performed using $\Z$ events.
Events are selected with single muon triggers requiring an isolated muon with $\pt > 27\GeV$ or a muon with $\pt > 50 \GeV$.
For this measurement, the probe is a muon passing the identification criteria,
and it is considered a passing probe if satisfies the \SIPthreeD, $d_{xy}$ and $d_z$ cuts of this analysis.
%
The efficiency to reconstruct a muon track in the inner detector is measured using as probes tracks
reconstructed in the muon system alone. The efficiency and 
data to MC scale factors are measured from Z events as a function of $\eta$ and \pt.
The scale factors are shown in Figure~\ref{fig:muoSFRun2}.

\begin{figure}
  \centering
  \subfigure [2016] {\includegraphics[height=.25\textheight]{final_HZZ_SF_2016UL_mupogsysts_newLoose_FINAL.pdf}}\\
  \subfigure [2017] {\includegraphics[height=.25\textheight]{final_HZZ_SF_2017UL_mupogsysts_newLoose_FINAL.pdf}}\\
  \subfigure [2018] {\includegraphics[height=.25\textheight]{final_HZZ_SF_2018UL_mupogsysts_newLoose_FINAL.pdf}}
  \caption{Efficiency scale factors, calculated as ratio between the efficiency in data and in simulation,
  applied to muons for the three years of data taking in \RunII.}
  \label{fig:muoSFRun2}
\end{figure}

%\subsection{Lepton Momentum Scale and Resolution validation using \texorpdfstring{\Zllll}{}}
%\input{Objects/scaleresoZ4l}

Electron and Muon identification criteria, energy/momentum calibrations, and efficiency measurements,
as well as the Final State Radiation (FSR) recovery, follow the same strategy used in Reference \todo{cite HiggsAN}. % and CMS_AN_2016-442 CMS_AN_2017-342
