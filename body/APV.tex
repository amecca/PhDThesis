\label{sec:APV}
In late 2015 and early 2016, the CMS strip tracker encountered a signal-to-noise ratio deterioration and a loss of hit detection on tracks, particularly as instantaneous luminosity increased.
Investigation revealed that the problem stemmed from saturation in the preamplifier of the strip readout chip (APV25) under high occupancies.
Lowering the operating temperature in \RunII unexpectedly prolonged preamplifier discharge time, resulting in charge buildup and a nonlinear response.
Muon reconstruction efficiency was also affected by preamplifier saturation.
%% The preamplifier's response was linear up to 3 MIPs, but nonlinear beyond.
This issue was resolved by adjusting the drain speed of the preamplifier \cite{Butz:2018dum} through the preamplifier feedback voltage bias (VFP), achieving a recovery of the hit efficiency to the same level as in Run~1~\cite{CMS-TRK-20-001}.

A model for preamplifier saturation was developed and integrated into simulations, with adjustments yielding better data-model agreement.
As a consequence of these changes, for the year 2016, the detector simulations before and after the adjustment differ substantially.
The two periods, which correspond to luminosity of around 19.5 fb$^{-1}$ and 16.8 fb$^{-1}$,
are therefore analysed separately and are referred to as ``2016preVFP'' and ``2016postVFP''.
