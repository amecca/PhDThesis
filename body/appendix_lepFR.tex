\chapter{Details of Lepton fake rate application}
\label{sec:leptonFR_details}
Let us denote a true lepton as \Pl and a fake as j.
The number of events with two fakes, regardless of the region they are classified into, is $N^{jj}$ and those with one fake is $N^{j\Pl+\Pl j}$.

Then, the number of events with two fakes that are classified into the various regions is:
\begin{subequations}
  \begin{align}
    \label{eq:lepFR_jj_4P}
    N^{jj}_{4P}   &= \sum_{i \ins N^{jj}} f_3^i f_4^i
    \\
    \label{eq:lepFR_jj_3P1F}
    N^{jj}_{3P1F} &= \sum_{i \ins N^{jj}} f_3^i(1-f_4^i) + (1-f_3^i)f_4^i
    \\
    \label{eq:lepFR_jj_2P2F}
    N^{jj}_{2P2F} &= \sum_{i \ins N^{jj}} (1-f_3^i)(1-f_4^i)
  \end{align}
\end{subequations}

while the number of events with one fake $N^{j\Pl+\Pl j}$ is:
\begin{subequations}
  \begin{align}
    \label{eq:lepFR_jl_4P}
    N^{j\Pl+\Pl j}_{4P}   &= \sum_{i \ins N^{j\Pl+\Pl j}} f_3^i + f_4^i
    \\
    \label{eq:lepFR_jl_3P1F}
    N^{j\Pl+\Pl j}_{3P1F} &= \sum_{i \ins N^{j\Pl+\Pl j}} (1-f_3^i) + (1-f_4^i)
  \end{align}
\end{subequations}
since we assume that no significant contribution from events with three prompt leptons is present in CR2P2F.

%% However, the events with two fakes that are classified into CR3P1F have to be subtracted in order to...

From Equation \ref{eq:lepFR_jj_2P2F} we can measure the number of events with two fakes from the yield in CR2P2F:
\begin{equation}
  \label{eq:lepFR_jj}
  N^{jj} = \sum_{i \ins 2P2F} \frac{1}{1-f_3^i} \frac{1}{1-f_4^i}
\end{equation}

Combining this result with Equation \ref{eq:lepFR_jj_4P}, it follows that the contribution to the signal region is:
\begin{equation}
  \label{eq:lepFR_jj_4P_result}
  N^{jj}_{4P} = \sum_{i \ins 2P2F} \frac{f_4^i}{1-f_4^i} \frac{f_4^i}{1-f_4^i}
\end{equation}

\paragraph{Background in CR3P1F\\}
The computation of the contribution from events with only one fake is not as straightforward, since the CR3P1F region itself has a background
from events with two fakes that were promoted with a probability dependent on the fake rate.

For simplicity, let us consider the component $\Pl j$, where the third lepton is prompt and the fourth is fake.
In the sub-region $PF$ of CR3P1F, where the third lepton passes the selection and the fourth fails, the contribution from $jj$ is:
\begin{equation}
    N^{jj}_{PF} = \sum_{i \ins N^{jj}} f_3^i (1-f_4^i) = \sum_{i \ins 2P2F} \frac{f_3^i}{1-f_3^i}
\end{equation}
which must be subtracted from the $PF$ yield when computing the contribution of $\Pl j$ to 4P:
\begin{equation}
  N^{\Pl j}_{4P} = \sum_{i \ins PF} \frac{f_4^i}{1-f_4^i} - \sum_{j \ins 2P2F} \frac{f_3^j}{1-f_3^j} \frac{f_4^j}{1-f_4^j}
\end{equation}

Repeating the same procedure for the other sub-region $FP$ and the $j\Pl$ component, the result for CR3P1F = $FP$ + $PF$:
\begin{equation}
  \label{eq:lepFR_jl_4P_result}
  N^{j\Pl+\Pl j}_{4P} = \sum_{i \ins 3P1F} \left( \frac{f_3^i}{1-f_3^i} + \frac{f_4^i}{1-f_4^i} \right) - 2 \sum_{j \ins 2P2F} \frac{f_4^j}{1-f_4^j} \frac{f_4^j}{1-f_4^j}
\end{equation}

\paragraph{Final result\\}
Summing the CR3P1F (Equation \ref {eq:lepFR_jl_4P_result}) and the CR2P2F (Equation \ref{eq:lepFR_jj_4P_result}) contributions to the signal region:
\begin{equation}
  N^{bkg}_{4P} = \sum_{i \ins 3P1F} \left( \frac{f_3^i}{1-f_3^i} + \frac{f_4^i}{1-f_4^i} \right) - \sum_{j \ins 2P2F} \frac{f_4^j}{1-f_4^j} \frac{f_4^j}{1-f_4^j}
\end{equation}

This is the result reported in Equation \ref{eq:lepFR_4P}, since for every event in CR3P1F one of $f_3$ and $f_4$ is always 0, since the lepton passes the selection.
