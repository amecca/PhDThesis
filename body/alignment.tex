\chapter{Alignment of the CMS Silicon Tracker}
\newcommand{\MPII}{\textsc{millepede}-II\xspace}
\newcommand{\HIPPY}{\textsc{HipPy}\xspace}
\newcommand{\MINRES}{\textsc{MinRes}\xspace}

The CMS silicon tracker is the largest in the world, both in terms of surface area and number of sensors.
In order to harness its exceptional resolution, a precise knowledge of the position and orientation of the modules is necessary.
During the installation, the mechanical alignment alignment of the modules resulted in a precision of O(100\mum),
which is much larger than the design hit resolution of the modules of O(10\mum).

Consequently, an additional refinement addressing the positional accuracy, orientation, and surface deformations of the sensors becomes imperative.
This refinement, commonly denoted as the tracker alignment, is characterised by the derivation of a set of parameters known as the tracker alignment constants.
These constants, which amount to around $200\,000$, are used during the track reconstruction to determine the true position of the hits.

Changes in the environment such as temperature variations and the ramping of the magnetic field induce movements in the tracker structures,
motivating regular updates of the alignment constants to maintain the target precision.
The method used in CMS to derive the constants consists in performing track fits with the corresponding track parameters unconstrained.

\paragraph{Hierarchical alignment\\}
The alignment parameters of the CMS tracker are organised in a hierarchy that follows the one of the modules themselves (see Figure \ref{fig:tracker_hierarchy}).

Each element of the hierarchy has its own six parameters (3 translations and 3 rotations).
This adds redundant degrees of freedom, since the movements of the large structures can be equivalently expressed by shifts of elements within their sub-hierarchies.
The redundancy is removed by linear equality constraints,
imposed on the original equation system with Lagrange multipliers
or by elimination.
\todo{rephrase!!!}
This treatment of alignables allows the fitting of only the large mechanical structures with varying granularity,
which is especially useful when the number of track is limited, e.g. when restarting data-taking after a commissioning period.

\section{Track-based alignment}
The track-based alignment consists in the determination of the correct alignment constants using the measured hits and reconstructed tracks.
The alignment parameters $\vec{p}\,$ are fitted by minimising the following $\chi^2$ function:

\begin{equation}
  \chi^2(\vec{p}, \vec{q}\,) \sum_j^{\rm tracks} \sum_i^{\rm hits} \left( \frac{m_{ij} - f_{ij}(\vec{p}, \vec{q_j})}{\sigma_{ij}^{m}} \right)
\end{equation}

where
\begin{itemize}
  \item The $\vec{p}$ are the (global) alignment parameters.
  \item The $\vec{q}$ are the parameters of the tracks (e.g. the track curvature or the deflection at a given detector layer).
    The $\vec{q_j}$ are the parameters of the $j$-th track.
  \item The $m_{ij}$ are the measurements (hits).
  \item The $f_{ij}$ are the predicted measurements using the track parameters and the alignment constants.
  \item The $\sigma^m_{ij}$ are the uncertainties of the measurements, due to local hit resolution and alignment uncertainty.
\end{itemize}

The alignment procedure allows for a variable subset of the parameters to be fitted, while keeping the other fixed.
This strategy is used for example when resuming data taking after commissioning or a magnet cycle, where initially only the high level structures are aligned,
leading to only 36 parameters.

The fitting procedure extracts the best values for the $\vec{p}\,$ regardless of the (potentially millions) of track parameters $\vec{q_j}\,$,
thus allowing alignment campaigns with large datasets and many degrees of freedom.
This is achieved by linearising the $\chi^2(\vec{p_0}\,+\Delta p, \vec{q_0}\,+\Delta q)$ as deviations from a previous set of alignment parameters.
Its minimisation can be expressed by a set of linear equations containing the measurements and the derivatives in the alignables or track parameters,
and treated like a matrix inversion problem:
\begin{equation}
  \label{eq:alignment_full_matrix}
  C \times \binom{\Delta\vec{p}}{\Delta\vec{q}} = \vec{b}
\end{equation}
Employing block matrix algebra it is possible to factorise the problem such that only a sub-matrix involving the $\Delta p$ has to be inverted,
while maintaining all the correlations from the track parameters~\cite{blobel2002new}.
Equation~\ref{eq:alignment_full_matrix} is reduced to:
\begin{equation}
  \label{eq:alignment_reduced_matrix}
  C' \times \Delta\vec{p} = \vec{b'}
\end{equation}
The $C'$ and $\vec{b'}$ have a significantly smaller sizes than $C$ and $\vec{b}$,
of the order of $\mathcal{O}(10\,000)$.

\section{Alignment algorithms}
Two independent track-based alignment algorithms are used by the CMS Collaboration, \MPII and \HIPPY.
The former also performs a global matrix inversion, while the latter neglects the blocks
relating the global alignables to the track parameters and iterates to improve the approximation.
Both are maintained and used, thus enabling cross-checks.

\subsection[Millepede-II]{\MPII}
The \MPII algorithm~\cite{blobel2002new,Blobel:2006yh,terascale-wiki} has been discussed in the context of CMS in Reference~\cite{CMS-TRK-11-002}.
In addition to CMS, The Belle~II experiment~\cite{PROC-CTD19-098} is also a main user.
The algorithm consists of two steps:
\begin{description}
  \item[\textsc{Mille}] This is integrated into the experiment-specific track-fitting software.
    Calculates and stores the track residuals with errors and the derivatives of the
    track (local) and module (global) parameters.
    The track fitting method must fit all the hits simultaneously and provide the full covariance matrix.
    The solution is based on the general broken lines method.
  \item[\textsc{Pede}] This is an independent Fortran programme that solves the linear equation system needed for the alignment.
    It uses the \textsc{Mille} output to perform the local fits and construct the global matrix $C'$ from Equation~\ref{eq:alignment_reduced_matrix}.
    An overview of the solution methods that are implemented is given in Table~\ref{tab:MP_solvers}.
\end{description}

\begin{table}
  \caption{Comparison of the solution methods implemented in \MPII.
  The computation time is reported as a function of the number of parameters $n$ and the number of iterations $n_{\rm it}$ if applicable.}
  \label{tab:MP_solvers}
  \centering
  \begin{tabular}{l l l l}
    \toprule
    Method                             & Computing time              & Solution type & Error calculation \\
    \midrule
    Gauss-Jordan inversion             & $\sim n^3$                  & Exact         & Yes \\
    Cholesky decomposition             & $\sim n^3$                  & Exact         & Skipped \\
    \MINRES \cite{Choi_2011,Choi_2014} & $\sim n^2 \times n_{\rm it}$ & Approximate   & No \\
    \bottomrule
  \end{tabular}
\end{table}

\subsection[HipPy]{\HIPPY}
Based on the hits-and-impact-points algorithm~\cite{karimaki2003sensor,CMS-NOTE-2006-018}
and improved for the BaBar alignment~\cite{BaBar_alignment},
it has been used extensively during the tracker commissioning
and CMS startup in \Run1~\cite{CMS-NOTE-2009-002}
and further improved in \Run2.
The improved algorithm is named hits-and-impact-points-past-year-1 (\HIPPY).

Unlike \MPII, it is local in that the alignment of each sensor is determined independently of the others.
The tracks are fitted, after applying the corresponding constraints,
with a given set of alignment conditions.
The $\chi^2$, expressed as a function of the alignment parameter, is minimised on a per-module basis.
Different weights can be assigned to different types of tracks,
and a new set of alignment conditions is computed and passed back to the track fitting.
The process is repeated until convergence is reached.

One disadvantage is that multiple iterations,
ranging from a few dozens to a hundred,
each requiring CPU-intensive track fits,
are required to solve correlations between alignables.
The advantages include the native integration with CMS software, providing features such as
the CMS Kalman filter code for track propagation
and the implementation of constraints such as mass or vertex constraints.
Each iteration of the algorithm is a very simple application of a small matrix inversion.
This simplicity and dependence on the CMS software makes the \HIPPY algorithm complementary to \MPII.

\section{Datasets}
Different types of datasets are generally used for the alignment.
The main benefit of a diverse set of tracks is that each track topology correlates a different set of alignables.

The main datasets are:
\begin{itemize}
\item \textbf{Collisions}
  \begin{itemize}
  \item Minimum bias, a sample of randomly chosen events passing the L1 trigger (Section \ref{sec:L1trigger}).
  \item Isolated muons: events passing one of the single muon HLT triggers with isolation cuts on a cone of radius \DR = 0.1
  \item Dimuon resonances: pairs of muons with a mass close to the \PZ or $\Upsilon$ %% \PGU looks too much like a Y
  \end{itemize}
\item \textbf{Cosmic muons}
  \begin{itemize}
  \item Cosmic RUns at ZEro Tesla (CRUZET), before magnet ramp up (no \pt requirements since it is not measured)
  \item Cosmic Runs At Four Tesla (CRAFT), after the magnet ramp-up
  \item Interfill cosmics, taken between LHC fills, with conditions similar to CRAFT
  \item Cosmics During Collisions (CDC)
  \end{itemize}
\end{itemize}
All of the tracks have a minimum \pt requirement, ranging from 1 to 5\GeV depending on the dataset.

Collisions datasets contain tracks propagating outwards from the interaction point, which correlate the modules radially.
These datasets contain several million tracks which are essential for a full module level alignment.

CMS has dedicated algorithms for the reconstruction of cosmic tracks both for commissioning and calibration.
CRUZET and CRAFT are available for alignment before the start of LHC collisions, to provide an alignment after a shutdown period.
Cosmic ray muons cross the whole detector, connecting modules located in the top and bottom halves.
This is fundamental to constrain several types of systematic distortions.

\section{Systematic misalignments}
Systematic shifts of the assumed position of the modules with respect to the real ones
are called systematic misalignments of the tracker geometry. They
may cause biases in the track reconstruction and negatively impact the physics performance.

A noteworthy class of misalignment is composed by the \textit{weak modes}.
They are transformations that do not affect significantly the $\chi^2$ of the tracks,
and are caused by a lack of constraints in the alignment fit.
While these shifts do not degrade the tracking performance,
they may affect certain topologies and correlations among tracks that are important for physics measurements.

\subsection{Validation of misaligments}
To check the quality of a set of alignment constants and the effect of misaligments,
measurements of several variables that have known values under perfectly aligned conditions are carried out.

\subsubsection{Geometry comparison} % GC
The geometry produced by the alignment is compared with a reference geometry,
such as the one used as a starting point for the procedure or the ideal conditions.
This serve as a guide to help visualise the effect of the procedure
and may permit to notice certain known unphysical distortions.

The global shift and rotation of large structures are removed and the module displacements
$\Delta z$, $\Delta r$ and $\Delta \phi$, as well as the rotations,
are plotted as function of $z$, $r$ and $\phi$.

\subsubsection{Cosmic ray muon track} % MTS
The upper and lower portions of cosmic ray muon tracks are reconstructed separately,
and the two sets of track parameters are compared at the point of closest approach to the beamline.
The two halves should have the same parameters, while systematic differences suggest a misaligment.

Cosmic ray and collision tracks have different topologies, so distortions that appear as
weak modes in collision may be constrained with cosmic tracks.
In particular they connect the top and bottom halves of the detector in a single track,
which is not possible with collisions, and help constrain the z position of the modules.
Since cosmic ray muon tracks are collected even before the start of the collisions,
this validation is and early tool in the commissioning of the detector before the start of LHC operations.

\subsubsection{Dimuon validation} % Zmumu

\subsubsection{Track-vertex impact parameter} % PV

\subsection{Modelling of systematic distortions}
\note{maybe}

\cite{CMS-TRK-20-001}
