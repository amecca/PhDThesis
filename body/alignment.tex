\chapter{Alignment of the CMS Silicon Tracker}

The CMS silicon tracker is the largest in the world, both in terms of surface area and number of sensors.
In order to harness its exceptional resolution, a precise knowlegde of the position and orientation of the modules is necessary
During the installation, the mechanical alignment alignment of the modules resulted in a precision of O(100 \mum),
which is much larger than the design hit resolution of the modules of O(10 \mum).

Consequently, an additional refinement addressing the positional accuracy, orientation, and surface deformations of the sensors becomes imperative.
This refinement, commonly denoted as the tracker alignment, is characterized by the derivation of a set of parameters known as the tracker alignment constants.
These constants, which amount to around $200\,000$, are used during the track reconstruction to determine the true position of the hits.

Changes in the environment such as temperature variations and the ramping of the magnetic field induce movements in the tracker structures,
motivating regular updates of the alignment constants to maintain the target precision.
The method used in CMS to derive the constants consists in performing track fits with the corresponding track parameters unconstrained.

\paragraph{Hierarchical alignment\\}
The alignment parametes of the CMS tracker are organized in a hierarchy that follows the one of the modules themselves (see Figure \ref{fig:tracker_hierarchy}).

Each element of the hierarchy has its own six parameters (3 translations and 3 rotations).
This adds redundant degrees of freedom, since the movements of the large structures can be equivalently expressed by the parameters of their components.
The redundancy is removed by linear equality constraints, imposed on the original equation system with Lagrange multipliers.
\todo{rephrase!!!}
This treatment of alignables allows the fitting of only the large mechanical structures with varying granularity,
which is expecially useful when the number of track is limited, e.g. when restarting data-taking after a commisioning period.

\section{Track-based alignment}
The track-based alignment consist in deriving the alignment parametes $\vec{p}$ by minimising the following $\chi^2$ function:

\begin{equation}
  \chi^2(\vec{p}, \vec{q}\,) \sum_j^{\rm tracks} \sum_i^{\rm hits} \left( \frac{m_{ij} - f_{ij}(\vec{p}, \vec{q_j})}{\sigma_{ij}^{m}} \right)
\end{equation}

where
\begin{itemize}
  \item The $\vec{p}$ are the alignment parameters.
  \item The $\vec{q}$ are the parameters of the tracks (e.g. the track curvature or the deflection at a given detector layer).
    The $\vec{q_j}$ are the parameters of the $j$-th track.
  \item The $m_{ij}$ are the measurements (hits).
  \item The $f_{ij}$ are the predicted measurements using the track parameters and the alignment constants.
  \item The $\sigma^m_{ij}$ are the uncertainties of the measurements, due to local hit resolution and alignment uncertainty.
\end{itemize}

The alignment procedure allows for a variable subset of the parameters to be fitted, while keeping the other fixed.


This strategy is used for example when resuming data taking after commissioning or a magnet cycle, where initially only the high level structures are aligned
