\chapter*{Introduction}
\addcontentsline{toc}{chapter}{Introduction}
\markboth{INTRODUCTION}{INTRODUCTION}

The Standard Model of particle physics summarizes our current understanding
of the elementary constituents of matter and their fundamental interactions.

The extension to nonabelian groups of the concept of gauge theory was first proposed
by Yang and Mills~\cite{PhysRev.96.191} in 1954.
Three years later, Wu demonstrated experimentally that parity is not conserved in weak interactions~\cite{PhysRev.105.1413}.
The genesis of the electroweak theory can be attributed to the seminal work of Glashow~\cite{GLASHOW1961579},
Salam~\cite{Salam:1968rm}, and Weinberg~\cite{PhysRevLett.19.1264},
who independently proposed a gauge theory based on the SU(2) $\times$ U(1) symmetry group.
This groundbreaking framework provided a unified description of electromagnetic and weak interactions.

The introduction in 1964 of the Higgs mechanism by Brout and Englert~\cite{PhysRevLett.13.321} and Higgs~\cite{PhysRevLett.13.508, HIGGS1964132}
allowed the generation of masses for vector gauge bosons and for fermions,
whilst maintaining electroweak gauge invariance.
Quantum chromodynamics, which describes the strong interaction, reached the modern formulation
around 1974 with the introduction of the concept of asymptotic freedom.
All of these results culminated in the construction of the Standard Model.

At the heart of the Standard Model lies the electroweak sector,
where electromagnetic and weak forces are unified under a gauge theory framework.
One of the distinguishing features of the electroweak theory is its non-abelian structure,
characterized by the non-commutativity of the gauge group's generators.
This non-trivial algebraic property gives rise to rich and intricate phenomena.
While the discovery of the Higgs boson in 2012~\cite{ATLASHiggsDiscovery, CMS-HIG-12-028}
marked an exceptional success for the Standard Model,
several areas remain to be studied.

Precise measurements in the electroweak sector can help to constrain the values of fundamental parameters within the Standard Model.
For example, measurements of the masses of the W and Z bosons, as well as their couplings to fermions and to themselves,
provide crucial input for precision calculations of various physical processes.
These measurements can then be used to extract values for parameters such as the weak mixing angle or the Higgs boson mass,
which in turn inform our understanding of the underlying symmetries and dynamics of the theory.

Furthermore, high precision tests of the electroweak sector contribute to our efforts to probe for new physics beyond the Standard Model.
While the Standard Model has been successful in describing many phenomena,
it could be viewed as a low-energy effective theory,
with new physics potentially emerging at higher energy scales.
Subtle deviations from Standard Model predictions may be indicative of new physics
and could also provide valuable clues about its nature,
potentially leading to the discovery of new phenomena or the elucidation of unresolved fundamental questions.

The simultaneous production of three electroweak vector bosons, either \PW, \PZ or \PGg,
is a class of very rare processes that are perfect candidates for high precision tests of the theory.
Triboson production is sensitive to the mechanism of the electroweak symmetry breaking
and provide important consistency checks that are complementary to the study of the Higgs boson.
They are also sensitive to potential anomalous couplings between gauge bosons,
which are predicted by many theories that extend the Standard Model,
and thus can provide valuable insights into the possible nature of new physics, or constrain new models.

The analysis presented in this thesis focuses on the search for triboson production, either $\PZ\PZ\PGg$ or $\PW\PZ\PGg$,
using data collected by the CMS experiment at LHC in the 2016--2018 period, known as \Run2.
The analysis is structured in three orthogonal channels,
characterized by the number of charged leptons.
The four and three lepton channels target the fully leptonic decays
of $\PZ\PZ\PGg$ or $\PW\PZ\PGg$ respectively,
whereas the two lepton channel is employed to search for
$\mathrm{V}\PZ\PGg$, where $\mathrm{V} = \PW,\PZ$ decays to two quarks and the \PZ decays to leptons.
Additionally, the measurement of the production cross section of the
processes $\Pp\Pp\to4\Pl\PGg$ and $\Pp\Pp\to3\Pl\PGn\PGg$ is also performed.

This work is structured as follows.
In chapter~\ref{sec:theory}, a brief review of the Standard Model, its particle content and the interactions is presented,
followed by a review of the experimental literature on triboson production.
Then in chapter~\ref{sec:experiment} the CERN LHC is introduced,
and the CMS detector used to collect the data for this analysis is described.
The subsequent chapter~\ref{sec:reconstruction} details the techniques used for the reconstruction
of the particles and the global event description,
as well as the identification and selection criteria used for the objects used in the analysis.
Chapter~\ref{sec:strategy} explains the characteristics of the signal and background processes,
the datasets employed, the techniques used to estimate the background contributions,
the event selection and the sources of systematic uncertainty that affect the results.
Finally, chapter~\ref{sec:results} highlights the main results of this work,
with a focus on the four lepton channel, and preliminary findings in the three lepton channel.
