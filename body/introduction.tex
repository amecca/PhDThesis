\chapter{Theoretical framework}
\section{The Standard Model of particle physics}
The Standard Model of particle physics is a quantum field theory based on gauge invariance principles.
At its core, the Standard Model postulates a set of elementary particles, divided into fermions, with half-integer spin and bosons, with integer spin, which interact through three of the four fundamental forces: electromagnetism, the weak nuclear force, and the strong nuclear force.

The fundamental interactions obey a local gauge symmetry, which is associated to a Lie group $U(1)_Y \otimes SU(2)_T \otimes SU(3)_C$, where Y, T and C denote hypercharge, weak isospin and colour.

The interactions are mediated by bosons such as the photon, the W and Z bosons and the gluons.
Gluons correspond to the generators of the SU(3)$_C$ gauge symmetry, while the photon and W$^+$, W$^-$ and Z$^0$ come from the mixing of the generators of the combined U(1)$_Y$ $\otimes$ SU(2)$_T$, as explained later in Section \ref{EWSB}.
The fundamental bosons all have spin 1, except for the Higgs Boson, which is the only known fundamental particle with spin 0.

All of the fundamental fermions have spin $\frac{1}{2}$ and include quarks, which constitute protons and neutrons, as well as leptons like electrons and neutrinos.
Quarks and leptons are further divided into three generations, each with two particles with different weak isospin, for a total of twelve elementary particles, as detailed in Table \ref{tab:fermions}; each fermion has a corresponding antiparticle.
Fermions belonging to different generations have different masses, which are not predicted by the model and must be determined experimentally.

The colour charge is exclusive to quarks, that carry a colour, and antiquarks, which carry an anticolour, while leptons and bound states with no net colour charge are called ``white''.
Similarly, gluons carry both a colour and an anticolour charge, while all the other bosons are white.
Colour charge can exist in three states, arbitrarily labeled blue, green, and red, complemented by an anticolour, antiblue, antigreen, and antired.

Only the W$^+$ and the W$^-$ bosons carry electromagnetic charge, while the others are neutral.
Only left-handed fermions and right handed antifermions can be engaged in charged weak interactions, while the neutral weak interaction can involve both types, although with different couplings.

The Standard Model has not only successfully described a wide range of experimental observations but has also guided the discovery of new particles, like the Higgs boson, which was observed at the Large Hadron Collider in 2012 \cite{ATLASHiggsDiscovery, CMSHiggsDiscovery}.
This framework plays a pivotal role in our comprehension of the universe, spanning from the conditions just moments after the Big Bang to the inner workings of atomic nuclei.

\begin{table}[tbh]
	\centering
	\caption{Fundamental fermions, classified by family.}
	\label{tab:fermions}
	\begin{tabular}{ c c c c }
		\hline
		 & 1$^{\text{st}}$ gen. & 2$^{\text{nd}}$ gen. & 3$^{\text{rd}}$ gen. \\%& Electric charge\\
		\hline
		\multirow{2}{*}{Quarks}  & u       & c         & t          \\%& +$\dfrac{2}{3}$ \\
		                         & d       & s         & b          \\%& -$\dfrac{1}{3}$ \\
		\hline
		\multirow{2}{*}{Leptons} & $\nu_e$ & $\nu_\mu$ & $\nu_\tau$ \\%& 0              \\
		                         & e$^-$   & $\mu$$^-$ & $\tau$$^-$ \\%& -1             \\
		\hline
	\end{tabular}
\end{table}

\section{Electroweak Symmetry Breaking}
\label{EWSB}
Electromagnetic and weak interactions can be described as a unified electroweak interaction at high energies, mediated by massless bosons B, that generates U(1)$_Y$, and W$^{1, 2, 3}$, generators of SU(2)$_T$.
However, while experiments show that weak interactions are mediated by massive bosons, the inclusion of a mass term of the form $-\frac{1}{2} M^2 W^a_\mu W^{a\, \mu}$ violates the U(1) $\otimes$ SU(2) gauge invariance.
Moreover, the addition in the Lagrangian of a mass term to fermions in the form $-m_q \bar\psi \psi$ would violate the SU(2) invariance, since the left and right handed components of the field $\psi$ behave differently under such gauge transformation.

In the SM, the Electroweak Symmetry Breaking (EWSB), through the Brout-Englert-Higgs Mechanism, shortly called Higgs Mechanism, generates mass terms for gauge bosons while preserving gauge invariance.
The core idea is the spontaneous symmetry breaking, a concept originally elaborated in condensed matter physics.
In 1964 Brout and Englert \cite{PhysRevLett.13.321} and Higgs \cite{PhysRevLett.13.508, HIGGS1964132}

The EWSB, causes also a mixing of the gauge boson fields, resulting in:
\begin{equation}
\begin{split}
  W^\pm = \dfrac{1}{\sqrt{2}} \left( W^1 \mp W^2 \right)
  \\
  \begin{pmatrix} \gamma \\ Z^0 \end{pmatrix} = \begin{pmatrix} \cos\theta_W & \sin\theta_W \\ -\sin\theta_W & \cos\theta_W \end{pmatrix} \begin{pmatrix} B \\ W^3\end{pmatrix}
\end{split}
\end{equation}
The only field that remains massless is the photon, while the other three become massive by interacting with three of the four components of the Higgs field $\psi$ through the Brout-Englert-Higgs Mechanism. The W mass is 80.38 GeV/c$^2$ and the Z mass is 91.19 GeV/c$^2$ \cite{Workman:2022ynf}.
From the broken symmetry a new particle emerges, corresponding to the fourth component of the Higgs field: the Higgs Boson.
The remaining (unbroken) symmetry is U(1)$_Q$ $\otimes$ SU(2)$_L$.

Because of the mixing, the resulting electric charge of the fermions, that is the coupling to the photon field, is a combination of the hypercharge and weak isospin.

\section{Triboson production}
The simultaneous production of three electroweak bosons is a class of extremely rare processes that offer an interesting insight into the mechanisms of the electroweak sector of the Standard Model.
Triboson events are extremely rare which involve the simultaneous emission of three electroweak gauge bosons -- photons, W and Z bosons.
Due to the strikingly low cross section the study of these processes is extremely challenging,
although some have final states with a very low background.
This offers the opportunity to test the predictions of the Standard Model with unparalled precision in a complementary way with respect to the study of the Higgs Boson.

%
% CMS
% SMP-15-008 & WGG, ZGG           &  8 TeV & http://dx.doi.org/10.1007/JHEP10(2017)072
% SMP-17-013 & WWW                & 13 TeV & http://dx.doi.org/10.1103/PhysRevD.100.012004
% SMP-19-014 & WWW, WWZ, WZZ, ZZZ & 13 TeV & http://dx.doi.org/10.1103/PhysRevLett.125.151802
% SMP-19-013 & WGG, ZGG           & 13 TeV & http://dx.doi.org/10.1007/JHEP10(2021)174
% SMP-22-006 & WWG, HG            & 13 TeV & https://cds.cern.ch/record/2875047
%
% ATLAS
% STDM-2013-05 & WGG           &  8 TeV & https://journals.aps.org/prl/abstract/10.1103/PhysRevLett.115.031802
% STDM-2019-17 & WZG           & 13 TeV & submitted https://atlas.web.cern.ch/Atlas/GROUPS/PHYSICS/PAPERS/STDM-2019-17
% STDM-2014-01 & ZG, ZZG       & 13 TeV & https://journals.aps.org/prd/abstract/10.1103/PhysRevD.93.112002
% STDM-2021-09 & ZGG           & 13 TeV & https://link.springer.com/article/10.1140/epjc/s10052-023-11579-8
% STDM-2016-05 & WWG, WZG      & 13 TeV & https://link.springer.com/article/10.1140/epjc/s10052-017-5180-3
% STDM-2015-07 & WWW           &  8 TeV & https://link.springer.com/article/10.1140/epjc/s10052-017-4692-1
% STDM-2016-06 & GGG           &  8 TeV & https://www.sciencedirect.com/science/article/pii/S0370269318302533
% STDM-2017-22 & WWW, WWZ, WZZ & 13 TeV & https://www.sciencedirect.com/science/article/pii/S0370269319306355
% HDBS-2019-16 & WWW           & 13 TeV & https://journals.aps.org/prl/abstract/10.1103/PhysRevLett.129.061803
% STDM-2018-33 & WGG           & 13 TeV & submitted https://atlas.web.cern.ch/Atlas/GROUPS/PHYSICS/PAPERS/STDM-2018-33
%
