\chapter{Results}

In this chapter the results of the search for $\PZ\PZ\PGg$ and $\PW\PZ\PGg$ triboson production,
as well as the measurement of the cross section of $\Pp\Pp \to 4\Pl \PGg$ are presented.
This analysis uses the data collected by the CMS experiment during the 2016-2018 period at a centre-of-mass energy of 13\TeV.
The analysis strategy is to search for excess events over the estimated background yield from known Standard Model processes.
Different identification strategies for the photon and several kinematic distributions are explored.

First the statistical tools used for the extraction of the parameters of interest and interpretation are described.
Then the expected result before unblinding will be discussed, and the best strategy selected.
Finally, the observed results will be reported.

\section{Statistical analysis}
\label{sec:statistical_analysis}
A statistical analysis is conducted to assess the presence of the signal process in the observed data,
quantify the significance of the excess over background predictions and measure its cross section.
The statistical inference is performed with the \textsc{Combine Tool} software package~\cite{CMS-NOTE-2011-005}.

\subsection{Likelihood and nuisance parameters}
The likelihood function is defined as the probability density function for a set of parameters of a model
that quantifies the agreement with a certain set of experimental observables (data).
The model adopted for this analysis defines a signal strength modifier $\mu$,
that multiplies the production cross section of the signal process and leaves all the other processes unchanged.
Each independent source of systematic uncertainty described in Section \ref{sec:systematics} is assigned a nuisance parameter $\theta_i$, and the full set is denoted $\vec\theta$.
They are of no direct interest for this analysis, but must be considered in the fitting procedure to extract correct results.
They enter the model through their probability density function $p_i(\tilde{\theta_i}|\theta_i)$,
which is the probability of measuring a certain value of the parameter given that the true value is $\theta_i$.
Furthermore, the expected yields of background, $b$, and signal, $s$, depend on the value of the nuisance parameters.

The global likelihood function is thus defined as:
\begin{equation}
  \label{eq:likelihood_full}
  \Likelihood(data\, |\, \mu, \vec\theta\,) = \prod_c \Likelihood_c(data\, |\, \mu \cdot s(\vec\theta\,) + b(\vec\theta\,)) \cdot \prod_i p_i(\tilde{\theta_i}\, |\, \theta_i)
\end{equation}
where c runs over all the channels, which are the four data-taking periods (2016preVFP, 2016postVFP, 2017, 2018).
The extraction of the signal strength proceeds through the maximisation of the complete likelihood function by varying the parameter of interest $\mu$ and the nuisances.
The $\Likelihood_c$ functions are the PDF of the binned distributions in each channel, and are given by the product of Poisson probabilities for every bin $j$ to observe $n_j$ events:
\begin{equation}
  \label{eq:likelihood_bin}
  \Likelihood_c(data\, |\, \mu \cdot s(\vec\theta\,) + b(\vec\theta\,)) = \prod_j \frac{\mu \cdot s_j(\vec\theta\,) + b_j(\vec\theta\,)}{n_j!} e^{-(\mu \cdot s_j(\vec\theta\,) + b_j(\vec\theta\,))}
\end{equation}

\subsection{Treatment of nuisance parameters}
Systematics uncertainties can be categorized into two main classes: the ones that affect only the event yield, and those that have an impact also on the shape of the predicted distributions.
Most of the uncertainties of the first class are parametrized with a log-normal distribution:
\begin{equation}
  \label{eq:lnNdef}
  \Probability(\tilde{\theta}\,|\,\theta) = \frac{1}{\sqrt{2 \pi} \text{ln} k} \cdot \frac{1}{\tilde{\theta}} \cdot \text{exp} \left( -\frac{(\text{ln}(\tilde{\theta}/\theta_m))^2}{2 \text{ln}^2 k} \right)
\end{equation}
which is the distribution of a random variable whose logarithm is normally distributed, with mean $\mu$ = $\text{ln}(\theta_m)$ and standard deviation $\sigma$ = $\text{ln}(k)$.
%% where the parameters $theta_s$ and $k$ can be defined in terms of the mean and standard deviation of a normally distributed variable: $\theta_m = e^{\mu}$ and $k = e^{\sigma}$.
The log-normal is used instead of a Gaussian because it enforces the positive-definite normalisation for the nuisance modelled, which is usually multiplying an event yield and thus cannot be negative.

The remaining systematics in the first class are those that represent a background coming from a statistically limited control region, such as the fake leptons and photons.
These are dominated by the statistically uncertainty in the control region and are modelled with a Gamma distribution:
\begin{equation}
  \label{eq:gammadef}
  \Probability(\theta\,|\,N\alpha) = \frac{1}{\Gamma(N) \alpha^N} \theta^{N-1} e^{-\theta/\alpha}
\end{equation}
where $N$ is the number of events in the control region, $\theta$ is the average transfer factor and $\Gamma(x)$ is the Gamma function.

The shape uncertainties of the second class are accounted for by interpolating the event fraction for each bin of three histograms: the one obtained for the central value, and the two obtained by shifting the nuisance parameter up and down by one standard deviation.

\subsection{Quantifying an excess}
To quantify the statistical significance of an excess of events over the background-only hypothesis, the following test statistic is used:
\begin{equation}
  \label{eq:test_statistic}
  t_0 = -2\text{ln} \frac {\Likelihood(data\,|\,0,\widehat{\vec{\theta_0}}\,)} {\Likelihood(data\,|\,\hat\mu,\widehat{\vec\theta}_{\hat\mu})}\,,\quad \text{with}\, \hat\mu \ge 0
\end{equation}

The numerator is evaluated under the background-only hypothesis ($\mu$ = 0), and $\widehat{\vec{\theta_0}}$ is the set of values of nuisance parameters that maximizes it under this null hypothesis.
The denominator is evaluated under the alternative signal + background hypothesis,
and the values $\hat{\mu}$ and $\widehat{\vec{\theta}}_{\hat\mu}$
are those that maximize the likelihood in this hypothesis.
This quantity is positive for a signal-like excess ($\mu$ > 0) and becomes 0 in the absence of an excess ($\mu$ = 0).

The significance of an excess is expressed in terms of the local \textit{p-value}, which is the probability to obtain a value of the test statistic $t_0$ greater than or equal to the one observed in experimental data, under the background-only hypothesis:
\begin{equation}
  \label{eq:pvalue}
  p_0 = \Probability(t_0 \ge t_0^{obs}\, |\, \mu = 0)
\end{equation}
That is, $p_0$ is the probability that a local statistical fluctuation of the background yield
is compatible with the observed data in the background only hypothesis,
at least as much as the signal hypothesis.

The p-value is usually expressed as a \textit{significance} $Z$ using the Gaussian one-sided integral:
\begin{equation}
  \label{eq:significance}
  p_0 = \int_Z^\infty \frac{1}{\sqrt{2\pi}}e^{-x^2/2}dx
\end{equation}

The conventional values of $Z$ = 3$\sigma$ and $Z$ = 5$\sigma$, corresponding to p-values of $1.3 \cdot 10^{-3}$ and $2.8 \cdot 10^{-7}$, are used to claim evidence for and the discovery of a new phenomenon respectively.


\section{Comparison of strategies}
As mentioned in Sections \ref{sec:evt_photon_selection} and \ref{sec:FSR_cut},
different approaches are used to select the photon
and results are reported both including and excluding the FSR contribution.

Additionally, several event variables are tested for the final fit to the signal strength.

The first is the invariant mass of the $\PZ\PZ\PGg$ system.
It is expected to be higher for triboson events where the photon comes from the hard scattering than for FSR.
Additionally, it is expected to be sensitive to Beyond Standard Model (BSM) contributions, especially in the high-energy tail.

The second variable is the transverse momentum of the photon.
As the $\PZ\PZ\PGg$ mass, it is presumed to be sensitive to BSM physics in the high-momentum tail.

The third option considered is the MVA-based ID working point passed by the photon.
Three bins are defined:
\begin{itemize}
\item \makebox[7em][l]{\bf wp80} The photon passes the \texttt{wp80} working point;
\item \makebox[7em][l]{\bf $\mathrm{wp90} \land \mathrm{!wp80}$} The photon passes \texttt{wp90}, but fails \texttt{wp80};
\item \makebox[7em][l]{\bf None} The photon fails \texttt{wp90};
\end{itemize}
The yield of the simulation is scaled in each bin according to the appropriate scale factors (see Section \ref{sec:photonID}).

This section reports the results for all of the combination of the aforementioned choices of the analysis strategy.

\section{Four lepton channel}
\providecommand{\impactswidthscale}{0.6}

\providecommand{\descriptionFakePhoton}[1]{\if d#1%
The data-driven estimate for \nonprompt photons is used%
\else
\Nonprompt photons are estimated from simulation%
\fi}

\providecommand{\captionScan}[5]{
Likelihood scan for the signal strength parameter
on the #1,
using the #2 working point of the photon #3.
\descriptionFakePhoton{#4}.
The additional selection to suppress the FSR component is #5applied.
The effect of groups of nuisance parameters on the uncertainty is assessed by sequentially fixing their value in the fit.
}

\providecommand{\captionImpact}[5]{
  Distribution and impacts of the systematic uncertainties on the signal strength fit
  on the #1,
  using the #2 working point of the photon #3.
  \descriptionFakePhoton{#4}.
  The additional selection to suppress the FSR component is #5applied.
}

\providecommand{\qqZZnonpro}{\ensuremath{\PQq\PAQq\to\PZ\PZ\to4\Pl(\PGg_\text{NP})}}
\providecommand{\WZnonpro}{\ensuremath{\PW\PZ\to3\Pl\PGn(\PGg_\text{NP})}}
\providecommand{\ZGprompt}{\ensuremath{\PZ\PGg\to\Plp\Plm\PGg}}
\providecommand{\DYnonpro}{\ensuremath{\PZ\to\Plp\Plm(\PGg_\text{NP})}}
 % Template captions for results
\subsection{Inclusive cross section}
\label{sec:impacts_inclusive}
\providecommand{\impactswidthscale}{0.6}
%% The \nonprompt photon contribution is estimated using the data-driven approach.
%% The variable considered is $m_{\PZ\PZ\PGg}$.
%% The impacts of the systematic uncertainties on the expected results is shown in Figure \ref{fig:inclusive_cutID_phoCR_mZZGloose}.

\begin{figure}
  \centering
  \includegraphics[height=0.33\textheight]{Figures/dataMC/Run2/phoCR/SR4P/SYS_mZZGloose_central_pow.pdf}
  \hfill
  \includegraphics[height=0.33\textheight]{Figures/combine/noPixVeto/Run2_SR4P_phoCR_lepCR_mZZGloose_impacts.pdf}
  \caption{\captionImpact{mass of the $\PZ\PZ\PGg$ system}{Loose}{cut-based ID}{d}{not }}
  \label{fig:inclusive_cutID_phoCR_mZZGloose}
\end{figure}

\begin{figure}
  \centering
  \includegraphics[height=0.33\textheight]{Figures/dataMC/Run2/lepCR/SR4P/lead_loose_pt_pow.pdf}
  \hfill
  \includegraphics[height=0.33\textheight]{Figures/combine/noPixVeto/Run2_SR4P_phoMC_lepCR_loosept_impacts.pdf}
  \caption{\captionImpact{transverse momentum of the photon}{Loose}{cut-based ID}{s}{not }}
  \label{fig:inclusive_cutID_phoMC_loosept}
\end{figure}

\begin{figure}
  \centering
  \includegraphics[height=0.33\textheight]{Figures/dataMC/Run2/lepCR/SR4P/SYS_wp90pt_central_pow.pdf}
  \hfill
  \includegraphics[height=0.33\textheight]{Figures/combine/noPixVeto/Run2_SR4P_phoMC_lepCR_mZZGwp90_impacts.pdf}
  \caption{\captionImpact{mass of the $\PZ\PZ\PGg$ system}{\texttt{wp90}}{MVA ID}{s}{not }}
  \label{fig:inclusive_mvaID_phoMC_mZZGwp90}
\end{figure}

\begin{figure}
  \centering
  \includegraphics[height=0.33\textheight]{Figures/dataMC/Run2/lepCR/SR4P/SYS_wp80pt_central_pow.pdf}
  \hfill
  \includegraphics[height=0.33\textheight]{Figures/combine/noPixVeto/Run2_SR4P_phoMC_lepCR_mZZGwp80_impacts.pdf}
  \caption{\captionImpact{mass of the $\PZ\PZ\PGg$ system}{\texttt{wp80}}{MVA ID}{s}{not }}
  \label{fig:inclusive_mvaID_phoMC_mZZGwp80}
\end{figure}

\begin{figure}
  \centering
  \includegraphics[height=0.33\textheight]{Figures/dataMC/Run2/lepCR/SR4P/SYS_MVAcut_central_pow.pdf}
  \hfill
  \includegraphics[height=0.33\textheight]{Figures/combine/noPixVeto/Run2_SR4P_phoMC_lepCR_MVAcut_impacts.pdf}
  \caption{Distribution and impacts of the systematic uncertainties on the signal strength fit
    on the yield in the various bins of the photon MVA ID.
    \descriptionFakePhoton{s}.
    The FSR cut is not applied.
  }
  \label{fig:inclusive_kin_phoMC_MVAcut}
\end{figure}


\subsection{Triboson with FSR cut}
\todo{significance with FSR cut}

The significances obtained with the various strategies for the identification of the photon and the estimation of the fake photon background are illustrated in Table~\ref{tab:summary_significances_inclusive}.
\begin{table}
  \caption{Summary of the results on the signal significance with the various strategies.}
  \label{tab:summary_significances_inclusive}
  \begin{tabular}{lllll}
    \toprule
    FSR cut                      & Photon ID                          & \nonprompt \PGg & Variable         & Significance\\
    \midrule
    \multirow{5}{*}{Not applied} & \multirow{2}{*}{Cut-based (Loose)} & data-driven     & $m_{\PZ\PZ\PGg}$ & 3.06 $\sigma$\\
                                 &                                    & simulation      & $\pt^\PGg$       & 3.20 $\sigma$\\
                                 & MVA (\texttt{wp90})                & simulation      & $m_{\PZ\PZ\PGg}$ & 3.34 $\sigma$\\
                                 & MVA (\texttt{wp80})                & simulation      & $m_{\PZ\PZ\PGg}$ & 3.26 $\sigma$\\
                                 & Kinematic                          & simulation      & MVA score        & 3.32 $\sigma$\\
    \hline
    \multirow{5}{*}{Applied}     & \multirow{2}{*}{Cut-based (Loose)} & data-driven     & $m_{\PZ\PZ\PGg}$ & X $\sigma$  \\
                                 &                                    & simulation      & $\pt^\PGg$       & X $\sigma$  \\
                                 & MVA (\texttt{wp90})                & simulation      & $m_{\PZ\PZ\PGg}$ & X $\sigma$  \\
                                 & MVA (\texttt{wp80})                & simulation      & $m_{\PZ\PZ\PGg}$ & X $\sigma$  \\
                                 & Kinematic                          & simulation      & MVA score        & X $\sigma$  \\
    \bottomrule
  \end{tabular}
\end{table}


\subsection{Likelihood scans inclusive}
\todo{temp name}
% Likelihood scans with nuisance groups without the FSR cut
\label{sec:likelihood_scans_inclusive}

\begin{figure}
  \centering
  \includegraphics[height=.33\textheight]{Figures/VVGammaAnalyzer/Run2/phoCR/SR4P/SYS_mZZGloose_central_pow\dataMCblind .pdf}
  \includegraphics[height=.33\textheight]{Figures/combine/inclusive/scan_\expobs_Run2_SR4P_phoCR_lepCR_mZZGloose.pdf}
  \caption{\captionScan{mass of the $\PZ\PZ\PGg$ system}{Loose}{cut-based ID}{d}{not }}
  \label{fig:scan_Run2_SR4P_phoCR_lepCR_mZZGloose}
\end{figure}

\begin{figure}
  \centering
  \includegraphics[height=.33\textheight]{Figures/VVGammaAnalyzer/Run2/lepCR/SR4P/SYS_mZZGloose_central_pow\dataMCblind .pdf}
  \hfill
  \includegraphics[height=.33\textheight]{Figures/combine/inclusive/scan_\expobs_Run2_SR4P_phoMC_lepCR_mZZGloose.pdf}
  \caption{\captionScan{mass of the $\PZ\PZ\PGg$ system}{Loose}{cut-based ID}{s}{not }}
  \label{fig:scan_Run2_SR4P_phoMC_lepCR_mZZGloose}
\end{figure}

\begin{figure}
  \centering
  \includegraphics[height=.33\textheight]{Figures/VVGammaAnalyzer/Run2/lepCR/SR4P/SYS_loosept_central_pow\dataMCblind .pdf}
  \hfill
  \includegraphics[height=.33\textheight]{Figures/combine/inclusive/scan_\expobs_Run2_SR4P_phoMC_lepCR_loosept.pdf}
  \caption{\captionScan{transverse momentum of the photon}{Loose}{cut-based ID}{s}{not }}
  \label{fig:scan_Run2_SR4P_phoMC_lepCR_loosept}
\end{figure}

\begin{figure}
  \centering
  \includegraphics[height=.33\textheight]{Figures/VVGammaAnalyzer/Run2/lepCR/SR4P/SYS_mZZGwp90_central_pow\dataMCblind .pdf}
  \hfill
  \includegraphics[height=.33\textheight]{Figures/combine/inclusive/scan_\expobs_Run2_SR4P_phoMC_lepCR_mZZGwp90.pdf}
  \caption{\captionScan{mass of the $\PZ\PZ\PGg$ system}{\texttt{wp90}}{MVA ID}{s}{not }}
  \label{fig:scan_Run2_SR4P_phoMC_lepCR_mZZGwp90}
\end{figure}

\begin{figure}
  \centering
  \includegraphics[height=.33\textheight]{Figures/VVGammaAnalyzer/Run2/lepCR/SR4P/SYS_wp90pt_central_pow\dataMCblind .pdf}
  \hfill
  \includegraphics[height=.33\textheight]{Figures/combine/inclusive/scan_\expobs_Run2_SR4P_phoMC_lepCR_wp90pt.pdf}
  \caption{\captionScan{transverse momentum of the photon}{\texttt{wp90}}{MVA ID}{s}{not }}
  \label{fig:scan_Run2_SR4P_phoMC_lepCR_wp90pt}
\end{figure}

\begin{figure}
  \includegraphics[height=.33\textheight]{Figures/VVGammaAnalyzer/Run2/lepCR/SR4P/SYS_mZZGwp80_central_pow\dataMCblind .pdf}
  \hfill
  \centering
  \includegraphics[height=.33\textheight]{Figures/combine/inclusive/scan_\expobs_Run2_SR4P_phoMC_lepCR_mZZGwp80.pdf}
  \caption{\captionScan{mass of the $\PZ\PZ\PGg$ system}{\texttt{wp80}}{MVA ID}{s}{not }}
  \label{fig:scan_Run2_SR4P_phoMC_lepCR_mZZGwp80}
\end{figure}

\begin{figure}
  \centering
  \includegraphics[height=.33\textheight]{Figures/VVGammaAnalyzer/Run2/lepCR/SR4P/SYS_MVAcut_central_pow\dataMCblind .pdf}
  \hfill
  \includegraphics[height=.33\textheight]{Figures/combine/inclusive/scan_\expobs_Run2_SR4P_phoMC_lepCR_MVAcut.pdf}
  \caption{Likelihood scan for the signal strength parameter
    on the yield in the various bins of the photon MVA ID.
    \descriptionFakePhoton{s}.
    The FSR cut is not applied.
    The effect of groups of nuisance parameters on the uncertainty is assessed by sequentially fixing their value in the fit.
  }
  \label{fig:scan_Run2_SR4P_phoMC_lepCR_MVAcut}
\end{figure}

The uncertainty on the signal strength due to the limited amount of data ranges between 0.35 and 0.50,
and is by far the largest contribution to the overall uncertainty.
This feature is observed for all of the aforementioned strategies for the statistical inference.
The other groups of systematics have much lower impacts.
The effect of the theoretical uncertainties is around 0.03--0.05,
while the impact of the luminosity is around 0.03
and the rest of the experimental uncertainties amount to 0.06--0.08 of the signal strength.

The data-driven estimate, when separated from the other experimental systematics,
adds an uncertainty of 0.05--0.08 on the signal strength when estimating the fake photon background
with the data-driven method (Figure~\ref{fig:scan_Run2_SR4P_phoCR_lepCR_mZZGloose}).
When estimating only the fake lepton background from data, the impact of its uncertainty on the signal strength is minimal.
These results are summarized in Table~\ref{tab:scanl_SR4P_inclusive}.

\subsection{Likelihood scans in triboson fiducial region}
\begin{figure}
  \centering
  \includegraphics[height=.33\textheight]{Figures/dataMC_FSRcut/Run2/phoCR/SR4P/SYS_mZZGloose_central_pow_\dataMCblind .pdf}
  \hfill
  \includegraphics[height=.33\textheight]{Figures/combine/FSRcut/scan_\expobs_Run2_SR4P_phoCR_lepCR_mZZGloose.pdf}
  \caption{\captionScan{mass of the $\PZ\PZ\PGg$ system}{Loose}{cut-based ID}{d}{}}
  \label{fig:scan_FSRcut_Run2_SR4P_phoCR_lepCR_mZZGloose}
\end{figure}

\begin{figure}
  \centering
  \includegraphics[height=.33\textheight]{Figures/dataMC_FSRcut/Run2/lepCR/SR4P/SYS_mZZGloose_central_pow_\dataMCblind .pdf}
  \hfill
  \includegraphics[height=.33\textheight]{Figures/combine/FSRcut/scan_\expobs_Run2_SR4P_phoMC_lepCR_mZZGloose.pdf}
  \caption{\captionScan{mass of the $\PZ\PZ\PGg$ system}{Loose}{cut-based ID}{s}{}}
  \label{fig:scan_FSRcut_Run2_SR4P_phoMC_lepCR_mZZGloose}
\end{figure}

\begin{figure}
  \centering
  \includegraphics[height=.33\textheight]{Figures/dataMC_FSRcut/Run2/lepCR/SR4P/SYS_loosept_central_pow_\dataMCblind .pdf}
  \hfill
  \includegraphics[height=.33\textheight]{Figures/combine/FSRcut/scan_\expobs_Run2_SR4P_phoMC_lepCR_loosept.pdf}
  \caption{\captionScan{transverse momentum of the photon}{Loose}{cut-based ID}{s}{}}
  \label{fig:scan_FSRcut_Run2_SR4P_phoMC_lepCR_loosept}
\end{figure}

\begin{figure}
  \centering
  \includegraphics[height=.33\textheight]{Figures/dataMC_FSRcut/Run2/lepCR/SR4P/SYS_mZZGwp90_central_pow_\dataMCblind .pdf}
  \hfill
  \includegraphics[height=.33\textheight]{Figures/combine/FSRcut/scan_\expobs_Run2_SR4P_phoMC_lepCR_mZZGwp90.pdf}
  \caption{\captionScan{mass of the $\PZ\PZ\PGg$ system}{\texttt{wp90}}{MVA ID}{s}{}}
  \label{fig:scan_FSRcut_Run2_SR4P_phoMC_lepCR_mZZGwp90}
\end{figure}

\begin{figure}
  \centering
  \includegraphics[height=.33\textheight]{Figures/dataMC_FSRcut/Run2/lepCR/SR4P/SYS_wp90pt_central_pow_\dataMCblind .pdf}
  \hfill
  \includegraphics[height=.33\textheight]{Figures/combine/FSRcut/scan_\expobs_Run2_SR4P_phoMC_lepCR_wp90pt.pdf}
  \caption{\captionScan{transverse momentum of the photon}{\texttt{wp90}}{MVA ID}{s}{}}
  \label{fig:scan_FSRcut_Run2_SR4P_phoMC_lepCR_wp90pt}
\end{figure}

\begin{figure}
  \includegraphics[height=.33\textheight]{Figures/dataMC_FSRcut/Run2/lepCR/SR4P/SYS_mZZGwp80_central_pow_\dataMCblind .pdf}
  \hfill
  \centering
  \includegraphics[height=.33\textheight]{Figures/combine/FSRcut/scan_\expobs_Run2_SR4P_phoMC_lepCR_mZZGwp80.pdf}
  \caption{\captionScan{mass of the $\PZ\PZ\PGg$ system}{\texttt{wp80}}{MVA ID}{s}{}}
  \label{fig:scan_FSRcut_Run2_SR4P_phoMC_lepCR_mZZGwp80}
\end{figure}

\begin{figure}
  \centering
  \includegraphics[height=.33\textheight]{Figures/dataMC_FSRcut/Run2/lepCR/SR4P/SYS_MVAcut_central_pow_\dataMCblind .pdf}
  \hfill
  \includegraphics[height=.33\textheight]{Figures/combine/FSRcut/scan_\expobs_Run2_SR4P_phoMC_lepCR_MVAcut.pdf}
  \caption{Likelihood scan for the signal strength parameter
    on the yield in the various bins of the photon MVA ID.
    \descriptionFakePhoton{s}.
    The FSR cut is applied.
    The effect of groups of nuisance parameters on the uncertainty is assessed by sequentially fixing their value in the fit.
  }
  \label{fig:scan_FSRcut_Run2_SR4P_phoMC_lepCR_MVAcut}
\end{figure}


\section{Yields and kinematic distributions}
\label{sec:yields}
After the selections described in Section \ref{sec:event_selection}, the event yields are extracted for signal and backgrounds for each of the four periods of \Run2, for the signal and control regions.

\subsection{Four lepton channel}
The pre-fit yields for the signal and background processes for the signal region
with four leptons passing the tight selection and a photon passing the cut-based ID (SR4P\_1P),
can be seen in Table \ref{tab:Run2_SR4P_phoCR_lepCR} (Table~\ref{tab:Run2_SR4P_phoMC_lepCR})
when using the data-driven (simulation) to estimate tha fake photon background.
Alternatively, the expected yields obtained using the
working point \texttt{wp90} of the MVA ID are shown in Table~\ref{tab:Run2_SR4P_phoMC_lepCR_wp90}.

Additional tables, including also the observed number of data events, are provided
for the fake photon application region (Table \ref{tab:yields_Run2_CR4P_1F_lepCR}),
and for the fake lepton application regions (Tables \ref{tab:yields_Run2_CR3P1F_1P} and \ref{tab:yields_Run2_CR2P2F_1P}).

The yields in the triboson fiducial region are shown
in Table~\ref{tab:yield_SR4P_1P_FSRcut_Loose} for the Loose working point of the cut-based ID,
and in Table~\ref{tab:yield_SR4P_1P_FSRcut_wp90} when using the \texttt{wp80} of the MVA ID instead.

\begin{table}
  \caption{Yields from the signal region SR4P\_1P, with four leptons passing the tight selection and a photon passing the cut-based ID.
  The \nonprompt and misidentified photons are estimated with the data-driven method
  and thus only the events containing a prompt generated photon are included from the main background samples.
  }
  \label{tab:Run2_SR4P_phoCR_lepCR}
  % Note: this is from the variable mZZGloose
  \resizebox{\textwidth}{!}{%
  \begin{tabular}{lccccc}
    \toprule
    {}                            & 2016preVFP         & 2016postVFP        & 2017               & 2018               & \Run2               \\
    \midrule
    $\PZ\PZ\PGg\to4\Pl\PGg$       &  1.962 $\pm$ 0.047 &  1.714 $\pm$ 0.040 &  4.021 $\pm$ 0.100 &  5.864 $\pm$ 0.142 &  13.561 $\pm$ 0.184 \\
    $\Pg\Pg\to\PZ\PZ\to4\Pe$      &  0.031 $\pm$ 0.001 &  0.030 $\pm$ 0.001 &  0.076 $\pm$ 0.002 &  0.104 $\pm$ 0.003 &   0.240 $\pm$ 0.004 \\
    $\Pg\Pg\to\PZ\PZ\to2\Pe2\PGm$ &  0.063 $\pm$ 0.003 &  0.057 $\pm$ 0.002 &  0.069 $\pm$ 0.003 &  0.104 $\pm$ 0.004 &   0.292 $\pm$ 0.006 \\
    $\Pg\Pg\to\PZ\PZ\to4\PGm$     &  0.056 $\pm$ 0.001 &  0.046 $\pm$ 0.001 &  0.117 $\pm$ 0.003 &  0.159 $\pm$ 0.004 &   0.378 $\pm$ 0.005 \\
    $\PZ\PZ\PZ$                   &  0.007 $\pm$ 0.005 &  0.000 $\pm$ 0.000 &  0.017 $\pm$ 0.008 &  0.029 $\pm$ 0.010 &   0.053 $\pm$ 0.014 \\
    $\PQt\PAQt\PZ$+jets           &  0.011 $\pm$ 0.002 &  0.010 $\pm$ 0.002 &  0.036 $\pm$ 0.005 &  0.041 $\pm$ 0.006 &   0.098 $\pm$ 0.009 \\
    Fake photons                  &  1.357 $\pm$ 0.655 &  1.011 $\pm$ 0.509 &  1.368 $\pm$ 0.519 &  2.451 $\pm$ 0.691 &   6.187 $\pm$ 1.198 \\
    $\PW\PZ\PZ$                   &  0.000 $\pm$ 0.000 &  0.021 $\pm$ 0.012 &  0.016 $\pm$ 0.015 &  0.065 $\pm$ 0.027 &   0.102 $\pm$ 0.033 \\
    $\PW\PW\PZ$                   &  0.000 $\pm$ 0.000 &  0.040 $\pm$ 0.040 &  0.041 $\pm$ 0.041 &  0.082 $\pm$ 0.058 &   0.163 $\pm$ 0.082 \\
    \noalign{\vspace{.3ex}}\hline\noalign{\vspace{.3ex}}
    Total                         &  3.488 $\pm$ 0.656 &  2.927 $\pm$ 0.512 &  5.759 $\pm$ 0.531 &  8.900 $\pm$ 0.709 &  21.074 $\pm$ 1.215 \\
    \bottomrule
  \end{tabular}
  }
\end{table}

\begin{table}
  \caption{Yields from the signal region SR4P\_1P, with four leptons passing the tight selection and a photon passing the cut-based ID.
    The \nonprompt and misidentified photons are taken from the simulation.
  }
  \label{tab:Run2_SR4P_phoMC_lepCR}
  % Note: this is from the variable mZZGloose
  \resizebox{\textwidth}{!}{%
  \begin{tabular}{lccccc}
    \toprule
    {}                            & 2016preVFP          & 2016postVFP       & 2017               & 2018               & \Run2               \\
    \midrule
    $\PZ\PZ\PGg\to4\Pl\PGg$       &  1.962 $\pm$ 0.047 &  1.714 $\pm$ 0.040 &  4.021 $\pm$ 0.100 &  5.864 $\pm$ 0.142 &  13.561 $\pm$ 0.184 \\
    $\PQq\PAQq\to\PZ\PZ\to4\Pl$   &  0.243 $\pm$ 0.011 &  0.210 $\pm$ 0.009 &  0.610 $\pm$ 0.018 &  0.858 $\pm$ 0.025 &   1.921 $\pm$ 0.034 \\
    $\Pg\Pg\to\PZ\PZ\to4\Pe$      &  0.036 $\pm$ 0.001 &  0.035 $\pm$ 0.001 &  0.088 $\pm$ 0.002 &  0.124 $\pm$ 0.003 &   0.283 $\pm$ 0.004 \\
    $\Pg\Pg\to\PZ\PZ\to2\Pe2\PGm$ &  0.077 $\pm$ 0.003 &  0.070 $\pm$ 0.003 &  0.086 $\pm$ 0.003 &  0.127 $\pm$ 0.005 &   0.360 $\pm$ 0.007 \\
    $\Pg\Pg\to\PZ\PZ\to4\PGm$     &  0.065 $\pm$ 0.001 &  0.054 $\pm$ 0.001 &  0.139 $\pm$ 0.003 &  0.192 $\pm$ 0.005 &   0.449 $\pm$ 0.006 \\
    $\PZ\PZ\PZ$                   &  0.007 $\pm$ 0.005 &  0.003 $\pm$ 0.003 &  0.021 $\pm$ 0.009 &  0.050 $\pm$ 0.014 &   0.082 $\pm$ 0.017 \\
    $\PW\PZ\PZ$                   &  0.008 $\pm$ 0.008 &  0.028 $\pm$ 0.014 &  0.038 $\pm$ 0.020 &  0.088 $\pm$ 0.031 &   0.163 $\pm$ 0.041 \\
    $\PQt\PAQt\PZ$+jets           &  0.012 $\pm$ 0.003 &  0.011 $\pm$ 0.002 &  0.048 $\pm$ 0.006 &  0.054 $\pm$ 0.007 &   0.124 $\pm$ 0.010 \\
    Fake leptons                  &  0.047 $\pm$ 0.066 &  0.000 $\pm$ 0.000 &  0.089 $\pm$ 0.119 &  0.117 $\pm$ 0.091 &   0.253 $\pm$ 0.164 \\
    $\PW\PW\PZ$                   &  0.000 $\pm$ 0.000 &  0.040 $\pm$ 0.040 &  0.041 $\pm$ 0.041 &  0.082 $\pm$ 0.058 &   0.163 $\pm$ 0.082 \\
    Total                         &  2.458 $\pm$ 0.082 &  2.164 $\pm$ 0.059 &  5.181 $\pm$ 0.164 &  7.557 $\pm$ 0.183 &  17.360 $\pm$ 0.266 \\
    \noalign{\vspace{.3ex}}\hline\noalign{\vspace{.3ex}}
    Total                         &  3.488 $\pm$ 0.656 &  2.927 $\pm$ 0.512 &  5.759 $\pm$ 0.531 &  8.900 $\pm$ 0.709 &  21.074 $\pm$ 1.215 \\
    \bottomrule
  \end{tabular}
  }
\end{table}

\begin{table}
  \caption{Yields from the signal region SR4P\_1P, with four leptons passing the tight selection
  and a photon passing the \texttt{wp90} of the MVA based ID.
  The \nonprompt and misidentified photons are taken from the simulation.
  }
  \label{tab:Run2_SR4P_phoMC_lepCR_wp90}
  % Note: this is from the variable mZZGloose
  \resizebox{\textwidth}{!}{%
  \begin{tabular}{lccccc}
    \toprule
    {}                            & 2016preVFP          & 2016postVFP       & 2017               & 2018               & \Run2               \\
    \midrule
    $\PZ\PZ\PGg\to4\Pl\PGg$       &  2.026 $\pm$ 0.047 &  1.781 $\pm$ 0.041 &  4.294 $\pm$ 0.103 &  6.126 $\pm$ 0.143 & 14.227 $\pm$ 0.187 \\
    $\PQq\PAQq\to\PZ\PZ\to4\Pl$   &  0.192 $\pm$ 0.009 &  0.158 $\pm$ 0.008 &  0.452 $\pm$ 0.015 &  0.652 $\pm$ 0.022 &  1.453 $\pm$ 0.029 \\
    $\Pg\Pg\to\PZ\PZ\to4\Pe$      &  0.036 $\pm$ 0.001 &  0.034 $\pm$ 0.001 &  0.090 $\pm$ 0.002 &  0.120 $\pm$ 0.003 &  0.280 $\pm$ 0.004 \\
    $\Pg\Pg\to\PZ\PZ\to2\Pe2\PGm$ &  0.076 $\pm$ 0.003 &  0.068 $\pm$ 0.003 &  0.086 $\pm$ 0.003 &  0.131 $\pm$ 0.005 &  0.361 $\pm$ 0.007 \\
    $\Pg\Pg\to\PZ\PZ\to4\PGm$     &  0.066 $\pm$ 0.001 &  0.053 $\pm$ 0.001 &  0.141 $\pm$ 0.003 &  0.189 $\pm$ 0.004 &  0.449 $\pm$ 0.006 \\
    $\PZ\PZ\PZ$                   &  0.010 $\pm$ 0.006 &  0.003 $\pm$ 0.003 &  0.023 $\pm$ 0.009 &  0.040 $\pm$ 0.012 &  0.076 $\pm$ 0.016 \\
    $\PQt\PAQt\PZ$+jets           &  0.014 $\pm$ 0.003 &  0.012 $\pm$ 0.002 &  0.045 $\pm$ 0.006 &  0.057 $\pm$ 0.007 &  0.129 $\pm$ 0.010 \\
    Fake leptons                  &  0.045 $\pm$ 0.066 &  0.103 $\pm$ 0.117 &  0.107 $\pm$ 0.118 &  0.022 $\pm$ 0.081 &  0.276 $\pm$ 0.196 \\
    $\PW\PZ\PZ$                   &  0.000 $\pm$ 0.000 &  0.020 $\pm$ 0.012 &  0.063 $\pm$ 0.022 &  0.076 $\pm$ 0.029 &  0.159 $\pm$ 0.038 \\
    $\PW\PW\PZ$                   &  0.000 $\pm$ 0.000 &  0.000 $\pm$ 0.000 &  0.040 $\pm$ 0.040 &  0.080 $\pm$ 0.057 &  0.121 $\pm$ 0.070 \\
    \noalign{\vspace{.3ex}}\hline\noalign{\vspace{.3ex}}
    Total                         &  2.464 $\pm$ 0.082 &  2.233 $\pm$ 0.124 &  5.342 $\pm$ 0.165 &  7.493 $\pm$ 0.178 & 17.532 $\pm$ 0.285 \\
    \bottomrule
  \end{tabular}
  }
\end{table}

\begin{table}
\caption{Yields from the fake photon application region CR4P\_1F, with four leptons passing the tight selection and a photon passing the VeryLoose ID but failing the cut-based ID Loose.}
\label{tab:yields_Run2_CR4P_1F_lepCR}
\resizebox{\textwidth}{!}{%
  \begin{tabular}{lccccc}
    \toprule
    {}                            & 2016preVFP         & 2016postVFP        & 2017               & 2018                & \Run2               \\
    \midrule
    $\PZ\PZ\PGg\to4\Pl\PGg$       &  0.302 $\pm$ 0.019 &  0.294 $\pm$ 0.017 &  0.837 $\pm$ 0.047 &   1.191 $\pm$ 0.064 &   2.623 $\pm$ 0.083 \\
    $\PQq\PAQq\to\PZ\PZ\to4\Pl$   &  2.400 $\pm$ 0.033 &  2.283 $\pm$ 0.030 &  6.428 $\pm$ 0.058 &   9.507 $\pm$ 0.084 &  20.617 $\pm$ 0.111 \\
    $\Pg\Pg\to\PZ\PZ\to4\Pe$      &  0.060 $\pm$ 0.001 &  0.059 $\pm$ 0.001 &  0.168 $\pm$ 0.003 &   0.258 $\pm$ 0.005 &   0.546 $\pm$ 0.006 \\
    $\Pg\Pg\to\PZ\PZ\to2\Pe2\PGm$ &  0.157 $\pm$ 0.004 &  0.158 $\pm$ 0.004 &  0.224 $\pm$ 0.005 &   0.322 $\pm$ 0.008 &   0.860 $\pm$ 0.011 \\
    $\Pg\Pg\to\PZ\PZ\to4\PGm$     &  0.107 $\pm$ 0.002 &  0.092 $\pm$ 0.002 &  0.253 $\pm$ 0.004 &   0.385 $\pm$ 0.006 &   0.837 $\pm$ 0.008 \\
    $\PZ\PZ\PZ$                   &  0.039 $\pm$ 0.011 &  0.025 $\pm$ 0.011 &  0.067 $\pm$ 0.016 &   0.069 $\pm$ 0.019 &   0.200 $\pm$ 0.030 \\
    $\PW\PZ\PZ$                   &  0.064 $\pm$ 0.022 &  0.071 $\pm$ 0.023 &  0.120 $\pm$ 0.034 &   0.120 $\pm$ 0.038 &   0.376 $\pm$ 0.060 \\
    $\PW\PW\PZ$                   &  0.048 $\pm$ 0.048 &  0.046 $\pm$ 0.046 &  0.000 $\pm$ 0.000 &   0.036 $\pm$ 0.036 &   0.131 $\pm$ 0.076 \\
    $\PQt\PAQt\PZ$+jets           &  0.056 $\pm$ 0.006 &  0.050 $\pm$ 0.005 &  0.171 $\pm$ 0.011 &   0.253 $\pm$ 0.015 &   0.530 $\pm$ 0.020 \\
    Fake leptons                  &  0.030 $\pm$ 0.052 &  0.051 $\pm$ 0.047 &  0.059 $\pm$ 0.092 &   0.050 $\pm$ 0.095 &   0.191 $\pm$ 0.149 \\
    \noalign{\vspace{.3ex}}\hline\noalign{\vspace{.3ex}}
    Total                         &  3.263 $\pm$ 0.084 &  3.130 $\pm$ 0.078 &  8.327 $\pm$ 0.124 &  12.191 $\pm$ 0.154 &  26.911 $\pm$ 0.229 \\
    Data                          &  5                 &  4                 &  7                 &  13                 &  29                 \\
    \bottomrule
  \end{tabular}
  }
\end{table}

\begin{table}
\caption{Yields in CR3P1F\_1P, one of the fake lepton application regions, with three leptons passing the tight selection, one passing only a loose selection, and a photon passing the cut-based ID.}
\label{tab:yields_Run2_CR3P1F_1P}
\resizebox{\textwidth}{!}{%
  \begin{tabular}{lccccc}
  \toprule
  {}                            & 2016preVFP          & 2016postVFP       & 2017               & 2018               & \Run2              \\
  \midrule
  $\PZ\PZ\PGg\to4\Pl\PGg$       &  0.143 $\pm$ 0.013 &  0.139 $\pm$ 0.011 &  0.390 $\pm$ 0.031 &  0.506 $\pm$ 0.041 &  1.178 $\pm$ 0.055 \\
  $\PW\PZ\PGg\to3\Pl\PGnl\PGg$  &  0.213 $\pm$ 0.011 &  0.191 $\pm$ 0.008 &  0.415 $\pm$ 0.020 &  0.711 $\pm$ 0.030 &  1.530 $\pm$ 0.039 \\
  $\PZ\PGg\to\Pl\Pl\PGg$        &  0.750 $\pm$ 0.285 &  0.000 $\pm$ 0.000 &  0.463 $\pm$ 0.388 &  0.492 $\pm$ 0.447 &  1.705 $\pm$ 0.657 \\
  $\PQq\PAQq\to\PZ\PZ\to4\Pl$   &  0.103 $\pm$ 0.007 &  0.085 $\pm$ 0.006 &  0.235 $\pm$ 0.011 &  0.328 $\pm$ 0.016 &  0.751 $\pm$ 0.021 \\
  $\Pg\Pg\to\PZ\PZ\to4\Pe$      &  0.011 $\pm$ 0.001 &  0.008 $\pm$ 0.000 &  0.022 $\pm$ 0.001 &  0.037 $\pm$ 0.002 &  0.078 $\pm$ 0.002 \\
  $\Pg\Pg\to\PZ\PZ\to2\Pe2\PGm$ &  0.012 $\pm$ 0.001 &  0.012 $\pm$ 0.001 &  0.016 $\pm$ 0.001 &  0.022 $\pm$ 0.002 &  0.062 $\pm$ 0.003 \\
  $\Pg\Pg\to\PZ\PZ\to4\PGm$     &  0.003 $\pm$ 0.000 &  0.003 $\pm$ 0.000 &  0.007 $\pm$ 0.001 &  0.008 $\pm$ 0.001 &  0.020 $\pm$ 0.001 \\
  $\PW\PZ\to3\Pl\PGn$           &  0.047 $\pm$ 0.027 &  0.035 $\pm$ 0.027 &  0.135 $\pm$ 0.087 &  1.101 $\pm$ 0.319 &  1.317 $\pm$ 0.333 \\
  $\PZ\PZ\PZ$                   &  0.003 $\pm$ 0.003 &  0.000 $\pm$ 0.000 &  0.010 $\pm$ 0.006 &  0.007 $\pm$ 0.005 &  0.020 $\pm$ 0.008 \\
  $\PW\PZ\PZ$                   &  0.009 $\pm$ 0.009 &  0.000 $\pm$ 0.000 &  0.001 $\pm$ 0.012 &  0.012 $\pm$ 0.012 &  0.022 $\pm$ 0.019 \\
  $\PQt\PAQt\PZ$+jets           &  0.043 $\pm$ 0.005 &  0.039 $\pm$ 0.004 &  0.068 $\pm$ 0.007 &  0.114 $\pm$ 0.010 &  0.263 $\pm$ 0.014 \\
  Drell-Yan + jets              &  0.000 $\pm$ 0.000 &  0.000 $\pm$ 0.000 &  0.000 $\pm$ 0.000 &  0.680 $\pm$ 3.520 &  0.680 $\pm$ 3.520 \\
  \noalign{\vspace{.3ex}}\hline\noalign{\vspace{.3ex}}
  Total                         &  1.337 $\pm$ 0.287 &  0.512 $\pm$ 0.031 &  1.760 $\pm$ 0.399 &  4.017 $\pm$ 3.563 &  7.625 $\pm$ 3.597 \\
  Data                          &  1.000 $\pm$ 1.000 &  0.000 $\pm$ 0.000 &  2.000 $\pm$ 1.414 &  5.000 $\pm$ 2.236 &  8.000 $\pm$ 2.828 \\
  \bottomrule
  \end{tabular}
  }
\end{table}

\begin{table}
\caption{Yields in CR2P2F\_1P, one of the fake lepton application regions, with two leptons passing the tight selection, two passing only a loose selection, and a photon passing the cut-based ID.}
\label{tab:yields_Run2_CR2P2F_1P}
\resizebox{\textwidth}{!}{%
  \begin{tabular}{lccccc}
    \toprule
    {} & 2016preVFP & 2016postVFP & 2017 & 2018 & Run2 \\
    \midrule
    $\PZ\PZ\PGg\to4\Pl\PGg$      &   0.009 $\pm$ 0.004 &  0.005 $\pm$ 0.002 &   0.006 $\pm$ 0.004 &    0.034 $\pm$ 0.010 &    0.054 $\pm$ 0.012 \\
    $\PW\PZ\PGg\to3\Pl\PGnl\PGg$ &   0.023 $\pm$ 0.003 &  0.020 $\pm$ 0.003 &   0.034 $\pm$ 0.006 &    0.053 $\pm$ 0.008 &    0.130 $\pm$ 0.011 \\
    Drell-Yan + jets             &   6.231 $\pm$ 3.599 &  1.855 $\pm$ 1.855 &  15.810 $\pm$ 5.603 &  37.763 $\pm$ 15.469 &  61.659 $\pm$ 16.943 \\
    $\PZ\PGg\to\Pl\Pl\PGg$       &   6.600 $\pm$ 0.799 &  0.000 $\pm$ 0.000 &  13.784 $\pm$ 1.461 &   22.222 $\pm$ 2.340 &   42.606 $\pm$ 2.872 \\
    $\PQq\PAQq\to\PZ\PZ\to4\Pl$  &   0.009 $\pm$ 0.002 &  0.007 $\pm$ 0.002 &   0.015 $\pm$ 0.003 &    0.023 $\pm$ 0.004 &    0.054 $\pm$ 0.006 \\
    $\PW\PZ\to3\Pl\PGnl$         &   0.002 $\pm$ 0.022 &  0.027 $\pm$ 0.019 &   0.123 $\pm$ 0.075 &    0.151 $\pm$ 0.136 &    0.302 $\pm$ 0.158 \\
    $\PQt\PAQt\PZ$+jets          &   0.023 $\pm$ 0.004 &  0.020 $\pm$ 0.003 &   0.035 $\pm$ 0.005 &    0.055 $\pm$ 0.007 &    0.133 $\pm$ 0.010 \\
    $\PQt\PZ\PQq$                &   0.005 $\pm$ 0.008 &  0.001 $\pm$ 0.006 &   0.018 $\pm$ 0.009 &    0.024 $\pm$ 0.013 &    0.048 $\pm$ 0.019 \\
    $\PQt\PW$                    &   0.057 $\pm$ 0.057 &  0.000 $\pm$ 0.000 &   0.088 $\pm$ 0.099 &    0.000 $\pm$ 0.000 &    0.144 $\pm$ 0.114 \\
    \noalign{\vspace{.3ex}}\hline\noalign{\vspace{.3ex}}
    Total                        &   6.616 $\pm$ 0.789 & 11.123 $\pm$ 6.842 & 28.359 $\pm$ 10.849 &  50.717 $\pm$ 14.889 &  96.814 $\pm$ 19.667 \\
    Data                         &  17                 &  9                 & 23                  &  28                  &  77                  \\
    \bottomrule
  \end{tabular}
  }
\end{table}

\begin{table}
  \caption{Yields in the fiducial triboson region of the four lepton channel, using the cut-based ID selection for the photon.}
  \label{tab:yield_SR4P_1P_FSRcut_Loose}
  \resizebox{\textwidth}{!}{%
  \begin{tabular}{lrrrrr}
    \toprule
    {}                           & 2016preVFP         & 2016postVFP        & 2017               & 2018               & \Run2              \\
    \midrule
    $\PZ\PZ\PGg\to4\Pl\PGg$      &  0.841 $\pm$ 0.031 &  0.728 $\pm$ 0.026 &  1.761 $\pm$ 0.066 &  2.485 $\pm$ 0.093 &  5.815 $\pm$ 0.121 \\
    $\PQq\PAQq\to\PZ\PZ\to4\Pl$  &  0.197 $\pm$ 0.010 &  0.167 $\pm$ 0.008 &  0.504 $\pm$ 0.016 &  0.701 $\pm$ 0.023 &  1.569 $\pm$ 0.031 \\
    Fake leptons                 &  0.064 $\pm$ 0.066 &  0.000 $\pm$ 0.000 &  0.106 $\pm$ 0.115 &  0.102 $\pm$ 0.086 &  0.272 $\pm$ 0.157 \\
    $\Pg\Pg\to\PZ\PZ\to4\PGm$    &  0.009 $\pm$ 0.001 &  0.010 $\pm$ 0.000 &  0.024 $\pm$ 0.001 &  0.036 $\pm$ 0.002 &  0.079 $\pm$ 0.002 \\
    $\Pg\Pg\to\PZ\PZ\to2\Pe2\PGm$&  0.021 $\pm$ 0.002 &  0.017 $\pm$ 0.001 &  0.023 $\pm$ 0.002 &  0.032 $\pm$ 0.002 &  0.094 $\pm$ 0.004 \\
    $\Pg\Pg\to\PZ\PZ\to4\Pe$     &  0.020 $\pm$ 0.001 &  0.016 $\pm$ 0.001 &  0.042 $\pm$ 0.002 &  0.060 $\pm$ 0.003 &  0.137 $\pm$ 0.003 \\
    $\PZ\PZ\PZ$                  &  0.004 $\pm$ 0.004 &  0.003 $\pm$ 0.003 &  0.014 $\pm$ 0.007 &  0.032 $\pm$ 0.011 &  0.052 $\pm$ 0.014 \\
    $\PW\PZ\PZ$                  &  0.000 $\pm$ 0.000 &  0.014 $\pm$ 0.010 &  0.023 $\pm$ 0.017 &  0.068 $\pm$ 0.028 &  0.106 $\pm$ 0.034 \\
    $\PQt\PAQt\PZ$+jets          &  0.005 $\pm$ 0.002 &  0.004 $\pm$ 0.001 &  0.022 $\pm$ 0.004 &  0.023 $\pm$ 0.005 &  0.055 $\pm$ 0.006 \\
    \noalign{\vspace{.3ex}}\hline\noalign{\vspace{.3ex}}
    Total                        &  1.160 $\pm$ 0.073 &  0.959 $\pm$ 0.030 &  2.520 $\pm$ 0.135 &  3.539 $\pm$ 0.132 &  8.178 $\pm$ 0.205 \\
    \bottomrule
  \end{tabular}
  }
\end{table}

\begin{table}
  \caption{Yields in the fiducial triboson region of the four lepton channel, using the \texttt{wp90} of the MVA ID selection for the photon.}
  \label{tab:yield_SR4P_1P_FSRcut_wp90}
  \resizebox{\textwidth}{!}{%
  \begin{tabular}{lrrrrr}
    \toprule
    {}                            & 2016preVFP         & 2016postVFP        & 2017               & 2018               & \Run2              \\
    \midrule
    $\PZ\PZ\PGg\to4\Pl\PGg$       &  0.804 $\pm$ 0.030 &  0.694 $\pm$ 0.025 &  1.703 $\pm$ 0.065 &  2.374 $\pm$ 0.089 &  5.576 $\pm$ 0.117 \\
    $\PQq\PAQq\to\PZ\PZ\to4\Pl$   &  0.088 $\pm$ 0.006 &  0.066 $\pm$ 0.005 &  0.209 $\pm$ 0.010 &  0.292 $\pm$ 0.014 &  0.656 $\pm$ 0.019 \\
    $\Pg\Pg\to\PZ\PZ\to4\Pe$      &  0.007 $\pm$ 0.000 &  0.007 $\pm$ 0.000 &  0.017 $\pm$ 0.001 &  0.025 $\pm$ 0.001 &  0.057 $\pm$ 0.002 \\
    $\Pg\Pg\to\PZ\PZ\to2\Pe2\PGm$ &  0.011 $\pm$ 0.001 &  0.010 $\pm$ 0.001 &  0.016 $\pm$ 0.001 &  0.024 $\pm$ 0.002 &  0.061 $\pm$ 0.003 \\
    $\Pg\Pg\to\PZ\PZ\to4\PGm$     &  0.014 $\pm$ 0.001 &  0.012 $\pm$ 0.001 &  0.031 $\pm$ 0.001 &  0.043 $\pm$ 0.002 &  0.100 $\pm$ 0.003 \\
    $\PZ\PZ\PZ$                   &  0.003 $\pm$ 0.003 &  0.003 $\pm$ 0.003 &  0.010 $\pm$ 0.006 &  0.024 $\pm$ 0.009 &  0.041 $\pm$ 0.012 \\
    $\PQt\PAQt\PZ$+jets           &  0.006 $\pm$ 0.002 &  0.005 $\pm$ 0.002 &  0.019 $\pm$ 0.003 &  0.016 $\pm$ 0.004 &  0.047 $\pm$ 0.006 \\
    Fake leptons                  &  0.063 $\pm$ 0.066 &  0.000 $\pm$ 0.000 &  0.109 $\pm$ 0.115 &  0.055 $\pm$ 0.078 &  0.227 $\pm$ 0.153 \\
    $\PW\PZ\PZ$                   &  0.000 $\pm$ 0.000 &  0.007 $\pm$ 0.007 &  0.039 $\pm$ 0.018 &  0.044 $\pm$ 0.022 &  0.090 $\pm$ 0.029 \\
    \noalign{\vspace{.3ex}}\hline\noalign{\vspace{.3ex}}
    Total                         &  0.997 $\pm$ 0.072 &  0.805 $\pm$ 0.027 &  2.153 $\pm$ 0.134 &  2.898 $\pm$ 0.122 &  6.853 $\pm$ 0.197 \\
    \bottomrule
  \end{tabular}
  }
\end{table}


\begin{figure}
\subfigure [2016preVFP ] {\includegraphics[width=.25\textwidth]{Figures/VVGammaAnalyzer/2016preVFP/lepCR/SR4P/ZZ_mass_pow.pdf}}%
\subfigure [2016postVFP] {\includegraphics[width=.25\textwidth]{Figures/VVGammaAnalyzer/2016postVFP/lepCR/SR4P/ZZ_mass_pow.pdf}}%
\subfigure [2017       ] {\includegraphics[width=.25\textwidth]{Figures/VVGammaAnalyzer/2017/lepCR/SR4P/ZZ_mass_pow.pdf}}%
\subfigure [2018       ] {\includegraphics[width=.25\textwidth]{Figures/VVGammaAnalyzer/2018/lepCR/SR4P/ZZ_mass_pow.pdf}}
\caption{Invariant mass of the ZZ system, without any requirements on the presence of photons, for each of the data-taking periods of \Run2.}
\label{fig:ZZmass_byyear}
\end{figure}

