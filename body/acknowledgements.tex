%% \subsection{Unacknowlegdements}
%% CMSSW, and especially its ``build system'' \mono{SCRAM}, made me waste months of my PhD trying to solve pointless issues.
%% A so-called build system is supposed to ease the burden of specifying where to find files and their dependencies.
%% SCRAM is somehow able to make it harder, all the while using its own names for things like ``depends on'', ``shared/static library'', ``plugin'', etc.
%% Its errors are as unhelpful as \verb|each dictionary was never generated [...] add (or move) the specification '<class name="whatever"/>' to the appropriate classes_def.xml|.
%% This sort of error messages were ``as useful as a fridge to a penguin'' and actively slowed down my development.

\chapter{Acknowledgements}
There are many people who helped me during my doctoral studies, with whom I collaborated, had fruitful discussions,
or who simply shared with me the lunches, breaks, after-work drinks and countless little moments of everyday life.

First of all I would like to express my gratitude to my supervisor Prof.~Riccardo Bellan, for his guidance and continuous support,
both in my project and in the development of my academic profile.
I already had the opportunity to work with him during my bachelor and master thesis,
profiting from his in-depth scientific knowledge and intuition based on his experience.
Despite his demanding schedule, he always strived to find the time to provide assistance and encouragement,
which significantly contributed to the successful completion of this thesis.

I would like to thank Dr.~Antonio Vagnerini
for his fundamental help in the construction of the analysis strategy and in the statistical interpretation,
as well as introducing and guiding me in the work on the tracker alignment.
With his positive attitude and experience he contributed in making our office a cheerful and productive environment,
kept me from getting sidetracked many times and taught me numerous things, including how to make the most out of my PhD.
We shared countless hours working on the analysis and on the alignment, and many trips to and from Bologna.

I extend my thanks to Cristiano, with whom I collaborated in these three years,
for his splendid work on the EFT study and the suggestions, tips and ornithological insights we shared,
which made our efforts a little more manageable for both of us.
%% \textit{Phasianus colchicus}
I also want to thank Giulio, who shared with me these arduous years of analysis in CMS,
for the inspiring passion in his work and his radiant personality.

I owe an important debt of gratitude to
Alessandra, for her readiness to help me with all kind of doubts about the analysis, the thesis and many other things, and also to
Riccardo, Bilal, Ksenia, Chiara and all the other (ex) Torino people with whom I shared my year at CERN,
who introduced me to the world of the Point~5 operations and
who helped me get the best out of this opportunity, and enjoy it to the fullest.
I would also like to thank my theorists friends from the bunker:
Luca, Antonio, Gloria, Andrea, Giorgio and many others.
We shared plenty of merry memories both inside and outside the physics department,
and the occasional rants about software, politics, energy and just about everything.

%% The numerous specimen of \textit{Columba livia} who live near the window of my office in the physics department.
%% Tally Hall's Marvin's Marvellous Mechanical Museum

An finally I would like to show my gratitude to my family and in particular
my parents Leonardo and Margherita who supported me in this endeavour, especially in the difficult months during the writing of this thesis.
\textit{Dulcis in fundo}, I want to thank
Erika, who was always by my side, despite the physical distance that was between us at times.
