\chapter*{Conclusions and outlook}
\addcontentsline{toc}{chapter}{Conclusions and outlook}

The Standard Model (SM) of particle physics successfully describes the
phenomenology at the Large Hadron Collider, which can be observed by
general purpose detectors, the Compact Muon Solenoid (CMS) and
the Toroidal LHC ApparatuS (ATLAS).

While the discovery of the Higgs Boson in 2012 completed the particle content of the theory,
several of its predictions remain untested to high accuracy.
Accordingly, the physics programme of both general purpose detectors
started to focus increasingly on high precision measurements.
In this context, non-abelian electroweak couplings are prime candidates,
because of their role in preserving the unitarity for certain classes of processes involving the Higgs boson,
and because of the large number of Beyond Standard Model scenarios that predict their enhancement.

The non-abelian triple and quartic gauge couplings are involved in several processes
that can be observed at a particle collider,
including the simultaneous production of three vector bosons.
The search for triboson production is experimentally very challenging
because of the small production cross section predicted by the SM,
which results in a low event yield.

The analysis presented in this thesis searches for
the simultaneous production of three electroweak bosons,
either two \PZ bosons and a photon or one \PZ boson, one \PW boson and a photon,
in three orthogonal channels with four, three and two charged leptons respectively.
This study is conducted using data collected at a centre-of-mass energy of 13\TeV
by the CMS detector in the 2016--2018 period (\Run2),
corresponding to an integrated luminosity of $137.6 \fbinv$.

The cross sections of the inclusive production at a centre-of-mass energy of 13\TeV
of four leptons, either electrons or muons, and a photon,
is also measured in the phase space of interest
$\sigma(\Pp\Pp \to 4\Pl\PGg) = 9.787{}^{+3.329}_{-2.740}$\usep fb.
%% This is $\mu \times \sigma_{\rm sample}^{\PZ\PZ\PGg}$, see Section~\ref{sec:likelihood_scans_inclusive}.
The significance of this process over the background-only hypothesis
is assessed to be $5.47\usep\sigma$.

The cross section for the production of two \PZ bosons and a photon
was measured in fiducial region where the SM contribution from final-state radiation photons are suppressed
The value, extracted with a maximum likelihood fit, is
$\sigma(\Pp\Pp \to \PZ\PZ\PGg \to 4\Pl\PGg) = 5.089{}^{+2.804}_{-2.168}$\usep fb.
The significance of $\PZ\PZ\PGg$ production over the hypothesis of only background
is $3.37\usep\sigma$.

Preliminary results in the three lepton channel for the production of
three leptons, one neutrino and a photon at a centre-of-mass energy of 13\TeV
in proton-proton collisions are presented.
The significance over the background-only hypothesis is $2.87\usep\sigma$,
and the measured cross section is
\todo{$\mu = 1.000{}^{+0.428}_{-0.411}$}.

All of the results in this analysis are limited by the amount of data, and would benefit greatly
from the inclusion of the data from the currently ongoing \Run3, which started in 2022,
and further complemented by the data that will be collected with the upcoming High Luminosity phase of LHC.

My personal contribution covers almost every aspect of the analysis,
starting from the design of the division into three channels.
I studied the simulation of the signal samples for the semileptonic channel,
including their validation, and followed their production.
After implementing the most recent calibrations of the data and simulation,

I included the calibrated photon collection and studied the several
identification criteria and their performance.
I also devised the data-driven estimation strategy for the background
from \nonprompt photons and jets misidentified as photons.
The separation of the triboson component and FSR component,
both in the signal and in the main background,
was also carefully investigated, for the four lepton channel in particular.

I examined all of the sources of systematic uncertainty in the analysis,
assessing their effect on the shape and yield of the signal and main backgrounds,
including their effect in the statistical model used to derive the results.

%% Add some ideas for further work.
%% a) combination of all decay channels ZZg WZg semi and fully-leptonic
%% b) using dedicated BDT to enhance signal purity
%% c) EFT interpretation( ?)
The combination of all three channels for $\PZ\PZ\PGg$ and $\PW\PZ\PGg$, both semi- and fully-leptonic,
will allow an increase in sensitivity, possibly up to the threshold for the observation
of these triboson production processes.
Unlike the four lepton channel, the others have a higher event yield and could benefit
from more effective techniques for background rejection,
such as machine learning methods, to boost the sensitivity to the signal.

An interpretation of this analysis in the context of the Standard Model Effective Field Theory~\todo{citation}
would provide insight and possibly help constraining several bosonic operators,
in connection with the ongoing effort within the CMS Collaboration to perform a global fit~\todo{cite CMS SMEFT combination, if any (?)}.

Within the CMS collaboration, I also took part to several experimental activities.
I participated in data-taking during 2022 as a shifter for the Data Acquisition system,
and also as a ``detector on call''.

I also participated to the tracker alignment activities,
including performing several alignment campaigns,
both for the derivation of the conditions at startup and after environmental changes
and to provide optimal calibrations for the reprocessing campaigns at the end of the year.
I also conducted the validation of many alignment candidates to assess their performance
and evaluate the impact of their application for online data taking.

I also worked on the muon reconstruction, implementing the possibility to redo
the pattern recognition in the muon chambers during the global reconstruction,
starting from seeds produced by the tracks from the inner tracker.
