\label{sec:unblinded_4L_FSRcut}

In the triboson fiducial region of the four lepton channel,
the strategy for which the highest sensitivity is expected is
the one that uses the data-driven wstimate for \nonprompt photons.
However, it is affected by the large uncertainty due to the small number
of events in the fake photon application region CR4P\_1F which,
after the additional cut to suppress FSR, is expected to contain only
$22.4 \pm 0.2$ events.

The second best strategy is the fit on $m_{\PZ\PZ\PGg}$ using the
\texttt{wp80} working point of the MVA based identification for the photon.
The post-fit event yields in the signal region are reported in Table~\ref{tab:yields_postfit_FSRcut_Run2_SR4P}.
The observed (expected) significance of the signal over the background-only hypothesis with this strategy is
$3.09 \usep\sigma$
($3.56 \usep\sigma$).

\begin{figure}
  \renewcommand{\dataMCblind}{}
  \renewcommand{\expobs}{observed}
  \centering
  \includegraphics[height=.33\textheight]{Figures/dataMC/Run2/lepCR/SR4P/SYS_mZZGwp80_central_pow\dataMCblind .pdf}
  \hfill
  \includegraphics[height=.33\textheight]{Figures/combine/FSRcut/scan_\expobs_Run2_SR4P_phoMC_lepCR_mZZGwp80.pdf}
  \caption{\captionScan{mass of the $\PZ\PZ\PGg$ system}{\texttt{wp80}}{MVA ID}{s}{}}
  \label{fig:scan_observed_FSRcut_Run2_SR4P}
\end{figure}

\begin{figure}
  \renewcommand{\dataMCblind}{}
  \renewcommand{\expobs}{observed}
  \centering
  \includegraphics[height=0.33\textheight]{Figures/dataMC_FSRcut/Run2/lepCR/SR4P/SYS_mZZGwp80_central_pow\dataMCblind .pdf}
  \hfill
  \includegraphics[height=0.33\textheight]{Figures/combine/FSRcut/impacts_\expobs_Run2_SR4P_phoMC_lepCR_mZZGwp80.pdf}
  \caption{Observed \captionImpact{mass of the $\PZ\PZ\PGg$ system}{\texttt{wp80}}{MVA ID}{s}{}}
  \label{fig:impacts_observed_FSRcut_Run2_SR4P}
\end{figure}

\begin{table}
  \caption{Post-fit yields in the triboson fiducial region in the four lepton channel.}
  \label{tab:yields_postfit_FSRcut_Run2_SR4P}
  \centering
  \small
  \providecommand{\headcell}[1]{\text{#1}} %{\makecell[c]{\text{#1}}}
  \resizebox{\textwidth}{!}{
  \begin{tabular}{>{$}l<{$} >{$}c<{$} >{$}c<{$} >{$}c<{$} >{$}c<{$} >{$}c<{$} >{$}c<{$}}
    \toprule
    \text{Process} & \headcell{2016postVFP} & \headcell{2016preVFP} & \headcell{2017} & \headcell{2018} & \headcell{\Run2} \\
    \midrule
    \PQt\PAQt\PZ+jets           & 0.01 \pm 0.00 & 0.01 \pm 0.00 & 0.02 \pm 0.01 & 0.01 \pm 0.00 & 0.04 \pm 0.01 \\
    \PW\PZ\PZ                   & 0.01 \pm 0.01 & 0.00 \pm 0.00 & 0.04 \pm 0.01 & 0.03 \pm 0.01 & 0.08 \pm 0.01 \\
    \PZ\PZ\PGg\to4\Pl\PGg       & 0.65 \pm 0.33 & 0.76 \pm 0.38 & 1.60 \pm 0.81 & 2.23 \pm 1.13 & 5.24 \pm 1.48 \\
    \Pq\Pq\to\PZ\PZ\to4\Pl      & 0.07 \pm 0.00 & 0.09 \pm 0.01 & 0.21 \pm 0.01 & 0.29 \pm 0.02 & 0.65 \pm 0.02 \\
    \PZ\PZ\PZ                   & 0.00 \pm 0.00 & 0.00 \pm 0.00 & 0.01 \pm 0.00 & 0.02 \pm 0.01 & 0.04 \pm 0.01 \\
    \Pg\Pg\to\PZ\PZ\to2\Pe2\PGm & 0.01 \pm 0.00 & 0.01 \pm 0.00 & 0.02 \pm 0.00 & 0.02 \pm 0.00 & 0.06 \pm 0.00 \\
    \Pg\Pg\to\PZ\PZ\to4\Pe      & 0.01 \pm 0.00 & 0.01 \pm 0.00 & 0.02 \pm 0.00 & 0.03 \pm 0.00 & 0.06 \pm 0.00 \\
    \Pg\Pg\to\PZ\PZ\to4\PGm     & 0.01 \pm 0.00 & 0.01 \pm 0.00 & 0.03 \pm 0.00 & 0.04 \pm 0.00 & 0.10 \pm 0.00 \\
    \noalign{\vspace{.3ex}}\hline\noalign{\vspace{.3ex}}
    \text{Total signal}         & 0.65 \pm 0.33 & 0.76 \pm 0.38 & 1.60 \pm 0.81 & 2.23 \pm 1.13 & 5.24 \pm 1.48 \\
    \text{Total background}     & 0.11 \pm 0.01 & 0.19 \pm 0.02 & 0.45 \pm 0.04 & 0.51 \pm 0.03 & 1.26 \pm 0.06 \\
    \noalign{\vspace{.3ex}}\hline\noalign{\vspace{.3ex}}
    \text{Total}                & 0.77 \pm 0.33 & 0.95 \pm 0.38 & 2.05 \pm 0.81 & 2.74 \pm 1.13 & 6.51 \pm 1.48 \\
    \text{Data}                 & 1 & 0 & 1 & 5 & 7 \\
    \bottomrule
  \end{tabular}
  }
\end{table}

The maximum likelihood fit in the fiducial triboson region results in a signal strenght of
$0.962{}^{+0.529}_{-0.408}$
($0.962{}^{+0.024}_{-0.015}\lum {}^{+0.010}_{-0.013}\thy {}^{+0.057}_{-0.036}\syst {}^{+0.525}_{-0.406}\stat$),
as displayed in Figure~\ref{fig:scan_observed_FSRcut_Run2_SR4P}.
The detailed breakdown of the sources of systematic uncertainty is reported in
Figure~\ref{fig:impacts_observed_FSRcut_Run2_SR4P}.

As previously discussed, the cross section times branching ratio of the signal sample is
$\sigma^{MC}(\Pp\Pp\to\PZ\PZ\PGg)\times\mathcal{B}(\PZ_0,\PZ_1\to\Pl\Pl) = 9.787\usep\text{fb}$, with $\Pl = \Pe, \PGm$.
Furthermore, the study reported in Section~\ref{sec:signal_genstudy} demonstrates
that the fraction of triboson events for the signal sample in the fiducial region is 52\usep\%.

Using the signal strenght obtained from the fit, the measured cross section
for triboson production of $\PZ\PZ\PGg$ in the final state with four leptons, either electrons or muons, and a photon is
$4.90{}^{+2.69}_{-2.08}\usep\text{fb}$
($4.90 {}^{+2.67}_{-2.07}\stat {}^{+0.05}_{-0.06}\thy {}^{+0.12}_{-0.08}\lum {}^{+0.29}_{-0.18}\syst \usep\text{fb}$).
