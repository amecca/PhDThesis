\label{sec:unblinded_4L_FSRcut}

As for the inclusive region, likelihood scans are performed for each of the candidate strategies,
and a summary is shown in Table~\ref{tab:scanl_SR4P_FSRcut}.
The full results, including the impacts of each systematic uncertainty, are reported in Appendix~\ref{sec:expected_4L_FSRcut}.

\begin{table}
  \caption{
    Summary of the likelihood scans for the four lepton channel
    in the triboson fiducial region
    and contribution from each of the nuisance parameter groups to the total uncertainty on the signal strength.
  }
  \label{tab:scanl_SR4P_FSRcut}
  \small
  \resizebox{\textwidth}{!}{
  \begin{tabular}{lccccccccc}
    \toprule
    Fake photon & data-driven    &    MC          &    MC         &    MC          &    MC         &    MC          &    MC         \\
    Fake lepton & data-driven    & data-driven    & data-driven   & data-driven    & data-driven   & data-driven    & data-driven   \\
    Photon ID   & Loose (cut)    & Loose (cut)    & Loose (cut)   & wp90 (MVA)     & wp90 (MVA)    & wp80 (MVA)     & kinematic     \\
    Variable    &$m_{\PZ\PZ\PGg}$&$m_{\PZ\PZ\PGg}$& $\pt^\PGg$    &$m_{\PZ\PZ\PGg}$& $\pt^\PGg$    &$m_{\PZ\PZ\PGg}$& MVAcut        \\
    \midrule
    $\mu$       & 1.00           & 1.00           & 1.00          & 1.00           & 1.00          & 1.00           & 1.00          \\
    total       & {-}0.43/{+}0.56 & {-}0.46/{+}0.59 & {-}0.41/{+}0.52 & {-}0.45/{+}0.57 & {-}0.42/{+}0.54 & {-}0.42/{+}0.53 & {-}0.43/{+}0.55 \\
    \hline
    statistical & {-}0.42/{+}0.55 & {-}0.45/{+}0.58 & {-}0.40/{+}0.51 & {-}0.44/{+}0.56 & {-}0.41/{+}0.54 & {-}0.41/{+}0.53 & {-}0.42/{+}0.55 \\
    theory      & {-}0.00/{+}0.00 & {-}0.02/{+}0.02 & {-}0.03/{+}0.04 & {-}0.02/{+}0.02 & {-}0.01/{+}0.01 & {-}0.01/{+}0.01 & {-}0.01/{+}0.01 \\
    luminosity  & {-}0.01/{+}0.03 & {-}0.02/{+}0.03 & {-}0.02/{+}0.04 & {-}0.02/{+}0.03 & {-}0.01/{+}0.03 & {-}0.01/{+}0.03 & {-}0.01/{+}0.03 \\
    experimental& {-}0.07/{+}0.10 & {-}0.08/{+}0.09 & {-}0.05/{+}0.08 & {-}0.08/{+}0.09 & {-}0.04/{+}0.07 & {-}0.06/{+}0.08 & {-}0.06/{+}0.08 \\
    \bottomrule
  \end{tabular}
  }
\end{table}

Unlike the inclusive region of the four lepton channel,
the data-driven estimate of \nonprompt photons is affected by the large uncertainty
due to the small number of events in the fake photon application region CR4P\_1F.
As shown in Table~\ref{tab:yield_CR4P_1F_FSRcut},
after the additional cut to suppress FSR, is expected to contain only
$22.4 \pm 0.2$ events.

The second best strategy is the fit on $m_{\PZ\PZ\PGg}$ using the
\texttt{wp80} working point of the MVA based identification for the photon.
The post-fit event yields in the signal region are reported in Table~\ref{tab:yields_postfit_FSRcut_Run2_SR4P}.
The observed (expected) significance of the signal over the background-only hypothesis with this strategy is
$3.00 \usep\sigma$
($3.46 \usep\sigma$),
which is above the threshold for evidence, but insufficient to claim the observation.

\begin{figure}
  \renewcommand{\dataMCblind}{}
  \renewcommand{\expobs}{observed}
  \centering
  \includegraphics[height=.33\textheight]{Figures/dataMC/Run2/lepCR/SR4P/SYS_mZZGwp80_central_pow\dataMCblind .pdf}
  \hfill
  \includegraphics[height=.33\textheight]{Figures/combine/FSRcut/scan_\expobs_Run2_SR4P_phoMC_lepCR_mZZGwp80.pdf}
  \caption{\captionScan{mass of the $\PZ\PZ\PGg$ system}{\texttt{wp80}}{MVA ID}{s}{}}
  \label{fig:scan_observed_FSRcut_Run2_SR4P}
\end{figure}

\begin{figure}
  \renewcommand{\dataMCblind}{}
  \renewcommand{\expobs}{observed}
  \centering
  %% \includegraphics[height=0.33\textheight]{Figures/dataMC_FSRcut/Run2/lepCR/SR4P/SYS_mZZGwp80_central_pow\dataMCblind .pdf}
  %% \hfill
  \includegraphics[height=0.33\textheight]{Figures/combine/FSRcut/impacts_\expobs_Run2_SR4P_phoMC_lepCR_mZZGwp80.pdf}
  \caption{\captionImpact{mass of the $\PZ\PZ\PGg$ system}{\texttt{wp80}}{MVA ID}{s}{}}
  \label{fig:impacts_observed_FSRcut_Run2_SR4P}
\end{figure}

\begin{table}
  \caption{Post-fit yields in the triboson fiducial region in the four lepton channel.}
  \label{tab:yields_postfit_FSRcut_Run2_SR4P}
  \centering
  \small
  \providecommand{\headcell}[1]{\text{#1}} %{\makecell[c]{\text{#1}}}
  \resizebox{\textwidth}{!}{
  \begin{tabular}{>{$}l<{$} >{$}c<{$} >{$}c<{$} >{$}c<{$} >{$}c<{$} >{$}c<{$} >{$}c<{$}}
    \toprule
    \text{Process} & \headcell{2016postVFP} & \headcell{2016preVFP} & \headcell{2017} & \headcell{2018} & \headcell{\Run2} \\
    \midrule
    \PZ\PZ\PGg\to4\Pl\PGg       & 0.65 \pm 0.30 & 0.76 \pm 0.35 & 1.60 \pm 0.73 & 2.23 \pm 1.02 & 5.24 \pm 1.34 \\
    \qqZZnonpro                 & 0.07 \pm 0.02 & 0.09 \pm 0.03 & 0.21 \pm 0.07 & 0.29 \pm 0.09 & 0.66 \pm 0.12 \\
    \Pg\Pg\to\PZ\PZ\to2\Pe2\PGm & 0.01 \pm 0.00 & 0.01 \pm 0.00 & 0.02 \pm 0.01 & 0.02 \pm 0.01 & 0.06 \pm 0.01 \\
    \Pg\Pg\to\PZ\PZ\to4\Pe      & 0.01 \pm 0.00 & 0.01 \pm 0.00 & 0.02 \pm 0.01 & 0.03 \pm 0.01 & 0.06 \pm 0.01 \\
    \Pg\Pg\to\PZ\PZ\to4\PGm     & 0.01 \pm 0.00 & 0.01 \pm 0.00 & 0.03 \pm 0.01 & 0.04 \pm 0.01 & 0.10 \pm 0.02 \\
    \PQt\PAQt\PZ+jets           & 0.01 \pm 0.00 & 0.01 \pm 0.00 & 0.02 \pm 0.01 & 0.01 \pm 0.01 & 0.04 \pm 0.01 \\
    \PZ\PZ\PZ                   & 0.00 \pm 0.00 & 0.00 \pm 0.00 & 0.01 \pm 0.00 & 0.02 \pm 0.00 & 0.04 \pm 0.01 \\
    \PW\PZ\PZ                   & 0.01 \pm 0.00 & 0.00 \pm 0.00 & 0.04 \pm 0.01 & 0.03 \pm 0.01 & 0.08 \pm 0.01 \\
    \noalign{\vspace{.3ex}}\hline\noalign{\vspace{.3ex}}
    \text{Total signal}         & 0.65 \pm 0.30 & 0.76 \pm 0.35 & 1.60 \pm 0.73 & 2.23 \pm 1.02 & 5.24 \pm 1.34 \\
    \text{Total background}     & 0.11 \pm 0.03 & 0.20 \pm 0.04 & 0.45 \pm 0.07 & 0.51 \pm 0.10 & 1.27 \pm 0.13 \\
    \noalign{\vspace{.3ex}}\hline\noalign{\vspace{.3ex}}
    \text{Total}                & 0.76 \pm 0.30 & 0.95 \pm 0.35 & 2.05 \pm 0.73 & 2.74 \pm 1.02 & 6.50 \pm 1.34 \\
    \text{Data}                 & 1 & 0 & 1 & 5 & 7 \\
    \bottomrule
  \end{tabular}
  }
\end{table}

The maximum likelihood fit in the fiducial triboson region results in a signal strength of
$0.96{}^{+0.53}_{-0.41}$
($0.96 {}^{+0.52}_{-0.41}\stat {}^{+0.01}_{-0.01}\thy {}^{+0.03}_{-0.01}\lum {}^{+0.07}_{-0.05}\syst$),
as displayed in Figure~\ref{fig:scan_observed_FSRcut_Run2_SR4P}.
The detailed breakdown of the sources of systematic uncertainty is reported in
Figure~\ref{fig:impacts_observed_FSRcut_Run2_SR4P}.

As previously discussed, the cross section times branching ratio of the signal sample is
$\sigma^{MC}(\Pp\Pp\to\PZ\PZ\PGg)\times\mathcal{B}(\PZ_0,\PZ_1\to\Pl\Pl) = 9.787\usep\text{fb}$, with $\Pl = \Pe, \PGm$.
Furthermore, the study reported in Section~\ref{sec:signal_genstudy} demonstrates
that the fraction of triboson events for the signal sample in the fiducial region is 52\usep\%.

Using the signal strength obtained from the fit, the measured cross section
for triboson production of $\PZ\PZ\PGg$ in the final state with four leptons, either electrons or muons, and a photon is
$4.89{}^{+2.69}_{-2.08}\usep\text{fb}$
($4.89 {}^{+2.67}_{-2.06}\stat {}^{+0.05}_{-0.06}\thy {}^{+0.13}_{-0.07}\lum {}^{+0.35}_{-0.26}\syst \usep\text{fb}$).
