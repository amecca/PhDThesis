\label{gg4l_normgroup}
The normalization uncertainty on the three samples $\Pg\Pg \to 4\Pe$, $\Pg\Pg \to 2\Pe2\PGm$ and $\Pg\Pg \to 4\PGm$
can be applied as a single nuisance parameter common to all of them
or as three separate parameters that are determined separately in the fit.

The difference between the two approaches, which is expected to be really small,
is evaulated by comparing the expected significances, as shown in Table~\ref{tab:grupnorm_cross_check_significance}.
As anticipated, the changes are negligible, usually less than or equal to $0.1\usep\sigma$,
except for the last strategy, where the expected significance changes by $\approx 0.2\usep\sigma$.

\begin{table}
  \caption{Expected significance with the various strategies,
    when the uncertainty on the normalization of the $\Pg\Pg \to 4\Pl$ samples is
    grouped into a single parameter or split into three, one for each final state.
    The data-driven strategy (first row) is evaluated as a sanity check and does not change, as expected.
    }
  \label{tab:grupnorm_cross_check_significance}
  \centering
  \begin{tabular}{l l l c c}
    \toprule
    \multicolumn{3}{c}{Strategy} & \multirow{2}{*}{\shortstack{Significance\\(single)}} & \multirow{2}{*}{\shortstack{Significance\\(split)}}\\
    \noalign{\vspace{.1ex}}\cline{1-3}\noalign{\vspace{.1ex}}
    Photon ID & \nonprompt \PGg & Variable\\
    \midrule
    Cut-based Loose  & data-driven & $m_{\PZ\PZ\PGg}$ & 4.54 & 4.54 \\
    Cut-based Loose  & simulation  & $m_{\PZ\PZ\PGg}$ & 4.60 & 4.65 \\
    Cut-based Loose  & simulation  & $\pt^\PGg$       & 4.53 & 4.57 \\
    MVA ({\tt wp90}) & simulation  & $m_{\PZ\PZ\PGg}$ & 5.06 & 5.13 \\
    MVA ({\tt wp90}) & simulation  & $\pt^\PGg$       & 5.00 & 5.07 \\
    MVA ({\tt wp80}) & simulation  & $m_{\PZ\PZ\PGg}$ & 5.20 & 5.30 \\
    Kinematic        & simulation  & MVA score        & 5.34 & 5.55 \\
    \bottomrule
  \end{tabular}
\end{table}
