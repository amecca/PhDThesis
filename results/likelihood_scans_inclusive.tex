% Likelihood scans with nuisance groups without the FSR cut
\label{sec:likelihood_scans_inclusive}
The extraction of the signal strength modifier $\mu$ proceeds through the maximisation of the likelihood,
as described in Section~\ref{sec:statistical_analysis}.
This procedure can be visualised by scanning the likelihood function for several values of the parameter $\mu$ while profiling the nuisance parameters.
For each value the best fit value of the nuisance parameters is computed,
and the resulting value of the likelihood is stored.
These points are then plotted as a function of $\mu$.

Usually the auxiliary quantity $-2\Delta\text{ln}\Likelihood$ (defined as $t_0$ in Equation~\ref{eq:test_statistic})
is used instead of the likelihood itself.
The width of the its profile is linked to the uncertainty on the estimate of $\mu$ from the fit.
More precisely, the set of values $\{ \mu / -2\Delta\text{ln}\Likelihood(\mu) < 1 (4) \}$ corresponds to the 68\usep\% (95\usep\%) confidence interval.

This procedure can also be performed by fixing the values of one or more nuisances instead of allowing them to be fitted by the algorithm.
The effect of fixing the value of one or more parameters is a reduction in the width of the likelihood shape.
This difference is ascribed to the effect of the frozen parameters.

Four groups of parameters are used in the following results:
\begin{itemize}
\item \textbf{theory:} uncertainties on the QCD scale, proton PDFs and on the value of \alpS;
\item \textbf{data-driven:} uncertainties related to the data-driven estimate of fake lepton or fake photon backgrounds;
\item \textbf{luminosity:} the uncertainty on the integrated luminosity corresponding to the data collected by the CMS experiment;
\item \textbf{others:} remaining experimental uncertainties, such as the lepton or photon efficiency scale factors or the \pileup weight;
\end{itemize}

\begin{figure}
  \centering
  \includegraphics[height=.33\textheight]{Figures/dataMC/Run2/phoCR/SR4P/SYS_mZZGloose_central_pow_\dataMCblind .pdf}
  \hfill
  \includegraphics[height=.33\textheight]{Figures/combine/inclusive/scan_\expobs_Run2_SR4P_phoCR_lepCR_mZZGloose.pdf}
  \caption{\captionScan{mass of the $\PZ\PZ\PGg$ system}{Loose}{cut-based ID}{d}{not }}
  \label{fig:scan_Run2_SR4P_phoCR_lepCR_mZZGloose}
\end{figure}

\begin{figure}
  \centering
  \includegraphics[height=.33\textheight]{Figures/dataMC/Run2/lepCR/SR4P/SYS_loosept_central_pow_\dataMCblind .pdf}
  \hfill
  \includegraphics[height=.33\textheight]{Figures/combine/inclusive/scan_\expobs_Run2_SR4P_phoMC_lepCR_mZZGloose.pdf}
  \caption{\captionScan{mass of the $\PZ\PZ\PGg$ system}{Loose}{cut-based ID}{s}{not }}
  \label{fig:scan_Run2_SR4P_phoMC_lepCR_mZZGloose}
\end{figure}

\begin{figure}
  \centering
  \includegraphics[height=0.33\textheight]{Figures/dataMC/Run2/lepCR/SR4P/SYS_mZZGwp90_central_pow_\dataMCblind .pdf}
  \hfill
  \includegraphics[height=.33\textheight]{Figures/combine/inclusive/scan_\expobs_Run2_SR4P_phoMC_lepCR_mZZGwp90.pdf}
  \caption{\captionScan{mass of the $\PZ\PZ\PGg$ system}{\texttt{wp90}}{MVA ID}{s}{not }}
  \label{fig:scan_Run2_SR4P_phoMC_lepCR_mZZGwp90}
\end{figure}

\begin{figure}
  \includegraphics[height=0.33\textheight]{Figures/dataMC/Run2/lepCR/SR4P/SYS_mZZGwp80_central_pow_\dataMCblind .pdf}
  \hfill
  \centering
  \includegraphics[height=.33\textheight]{Figures/combine/inclusive/scan_\expobs_Run2_SR4P_phoMC_lepCR_mZZGwp80.pdf}
  \caption{\captionScan{mass of the $\PZ\PZ\PGg$ system}{\texttt{wp80}}{MVA ID}{s}{not }}
  \label{fig:scan_Run2_SR4P_phoMC_lepCR_mZZGwp80}
\end{figure}

\begin{figure}
  \centering
  \includegraphics[height=.33\textheight]{Figures/dataMC/Run2/lepCR/SR4P/SYS_wp90pt_central_pow_\dataMCblind .pdf}
  \hfill
  \includegraphics[height=.33\textheight]{Figures/combine/inclusive/scan_\expobs_Run2_SR4P_phoMC_lepCR_wp90pt.pdf}
  \caption{\captionScan{transverse momentum of the photon}{\texttt{wp90}}{MVA ID}{s}{not }}
  \label{fig:scan_Run2_SR4P_phoMC_lepCR_wp90pt}
\end{figure}

\begin{figure}
  \centering
  \includegraphics[height=.33\textheight]{Figures/dataMC/Run2/lepCR/SR4P/SYS_MVAcut_central_pow_\dataMCblind .pdf}
  \hfill
  \includegraphics[height=.33\textheight]{Figures/combine/inclusive/scan_\expobs_Run2_SR4P_phoMC_lepCR_MVAcut.pdf}
  \caption{Likelihood scan for the signal strength parameter
    on the yield in the various bins of the photon MVA ID.
    \descriptionFakePhoton{s}.
    The FSR cut is not applied.
    The effect of groups of nuisance parameters on the uncertainty is assessed by sequentially fixing their value in the fit.
  }
  \label{fig:scan_Run2_SR4P_phoMC_lepCR_MVAcut}
\end{figure}

The uncertainty on the signal strength due to the statistics is between 0.35 and 0.50,
and is by far the largest contribution to the total.
This feature is observed for all of the strategies tested here.
The other groups of systematics have much lower impacts.
The theoretical uncertainties are around 0.03-0.05,
while the effect of the luminosity is around 0.03
and the rest of the experimental uncertainties amount to 0.06-0.08 of the signal strength.
The data-driven estimate, when split of the other experimental uncertainties,
adds an undertainty of 0.05-0.08 on the signal strength when estimating the fake photon background
with the data-driven method (Figure~\ref{fig:scan_Run2_SR4P_phoCR_lepCR_mZZGloose}).

\note{Temp}
The signal sample has a cross section of 22.02\usep{}fb, as reported in Table~\ref{tab:listofsamples}.
Assuming a signal strength of
$1.000^{+0.480}_{-0.406}$,
as extracted from
\todo{best strategy}, % TEMP MVAcut
the measured cross section for the production of $\Pp\Pp \to 4\Pl\PGg$ ($\Pl = \Pe,\,\PGm$) at a centre-of-mass energy of $13\TeV$ is
$22.02^{+10.57}_{-8.94}\usep\text{fb}$.
