% Likelihood scans with nuisance groups without the FSR cut
\label{sec:likelihood_scans_inclusive}

\begin{figure}
  \centering
  \includegraphics[height=.33\textheight]{Figures/dataMC/Run2/phoCR/SR4P/SYS_mZZGloose_central_pow\dataMCblind .pdf}
  \hfill
  \includegraphics[height=.33\textheight]{Figures/combine/inclusive/scan_\expobs_Run2_SR4P_phoCR_lepCR_mZZGloose.pdf}
  \caption{\captionScan{mass of the $\PZ\PZ\PGg$ system}{Loose}{cut-based ID}{d}{not }}
  \label{fig:scan_Run2_SR4P_phoCR_lepCR_mZZGloose}
\end{figure}

\begin{figure}
  \centering
  \includegraphics[height=.33\textheight]{Figures/dataMC/Run2/lepCR/SR4P/SYS_mZZGloose_central_pow\dataMCblind .pdf}
  \hfill
  \includegraphics[height=.33\textheight]{Figures/combine/inclusive/scan_\expobs_Run2_SR4P_phoMC_lepCR_mZZGloose.pdf}
  \caption{\captionScan{mass of the $\PZ\PZ\PGg$ system}{Loose}{cut-based ID}{s}{not }}
  \label{fig:scan_Run2_SR4P_phoMC_lepCR_mZZGloose}
\end{figure}

\begin{figure}
  \centering
  \includegraphics[height=.33\textheight]{Figures/dataMC/Run2/lepCR/SR4P/SYS_loosept_central_pow\dataMCblind .pdf}
  \hfill
  \includegraphics[height=.33\textheight]{Figures/combine/inclusive/scan_\expobs_Run2_SR4P_phoMC_lepCR_loosept.pdf}
  \caption{\captionScan{transverse momentum of the photon}{Loose}{cut-based ID}{s}{not }}
  \label{fig:scan_Run2_SR4P_phoMC_lepCR_loosept}
\end{figure}

\begin{figure}
  \centering
  \includegraphics[height=.33\textheight]{Figures/dataMC/Run2/lepCR/SR4P/SYS_mZZGwp90_central_pow\dataMCblind .pdf}
  \hfill
  \includegraphics[height=.33\textheight]{Figures/combine/inclusive/scan_\expobs_Run2_SR4P_phoMC_lepCR_mZZGwp90.pdf}
  \caption{\captionScan{mass of the $\PZ\PZ\PGg$ system}{\texttt{wp90}}{MVA ID}{s}{not }}
  \label{fig:scan_Run2_SR4P_phoMC_lepCR_mZZGwp90}
\end{figure}

\begin{figure}
  \centering
  \includegraphics[height=.33\textheight]{Figures/dataMC/Run2/lepCR/SR4P/SYS_wp90pt_central_pow\dataMCblind .pdf}
  \hfill
  \includegraphics[height=.33\textheight]{Figures/combine/inclusive/scan_\expobs_Run2_SR4P_phoMC_lepCR_wp90pt.pdf}
  \caption{\captionScan{transverse momentum of the photon}{\texttt{wp90}}{MVA ID}{s}{not }}
  \label{fig:scan_Run2_SR4P_phoMC_lepCR_wp90pt}
\end{figure}

\begin{figure}
  \includegraphics[height=.33\textheight]{Figures/dataMC/Run2/lepCR/SR4P/SYS_mZZGwp80_central_pow\dataMCblind .pdf}
  \hfill
  \centering
  \includegraphics[height=.33\textheight]{Figures/combine/inclusive/scan_\expobs_Run2_SR4P_phoMC_lepCR_mZZGwp80.pdf}
  \caption{\captionScan{mass of the $\PZ\PZ\PGg$ system}{\texttt{wp80}}{MVA ID}{s}{not }}
  \label{fig:scan_Run2_SR4P_phoMC_lepCR_mZZGwp80}
\end{figure}

\begin{figure}
  \centering
  \includegraphics[height=.33\textheight]{Figures/dataMC/Run2/lepCR/SR4P/SYS_MVAcut_central_pow\dataMCblind .pdf}
  \hfill
  \includegraphics[height=.33\textheight]{Figures/combine/inclusive/scan_\expobs_Run2_SR4P_phoMC_lepCR_MVAcut.pdf}
  \caption{Likelihood scan for the signal strength parameter
    on the yield in the various bins of the photon MVA ID.
    \descriptionFakePhoton{s}.
    The FSR cut is not applied.
    The effect of groups of nuisance parameters on the uncertainty is assessed by sequentially fixing their value in the fit.
  }
  \label{fig:scan_Run2_SR4P_phoMC_lepCR_MVAcut}
\end{figure}

The uncertainty on the signal strength due to the limited amount of data ranges between 0.35 and 0.50,
and is by far the largest contribution to the overall uncertainty.
This feature is observed for all of the aforementioned strategies for the statistical inference.
The other groups of systematics have much lower impacts.
The effect of the theoretical uncertainties is around 0.03--0.05,
while the impact of the luminosity is around 0.03
and the rest of the experimental uncertainties amount to 0.06--0.08 of the signal strength.

The data-driven estimate, when separated from the other experimental systematics,
adds an uncertainty of 0.05--0.08 on the signal strength when estimating the fake photon background
with the data-driven method (Figure~\ref{fig:scan_Run2_SR4P_phoCR_lepCR_mZZGloose}).
When estimating only the fake lepton background from data, the impact of its uncertainty on the signal strength is minimal.
These results are summarized in Table~\ref{tab:scanl_SR4P_inclusive}.
