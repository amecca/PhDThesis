\label{sec:unblinded_4L_inclusive}

% Justify the choice of the strategy
In the inclusive four lepton signal region, the data-driven estimate of the fake photon background is the most conservative.
Its main uncertainty is due to the limited number of events in the application region CR4P\_1F.

It is chosen for the unblinding also because of its model-independence.
Indeed, the division of the $\PQq\PAQq \to \PZ\PZ \to 4\Pl$ sample
based on the presence of a generated prompt photon
may change with a different combination of the matrix element generator and the hadronizer or their configuration.

The observed (expected) significance of the signal over the background-only hypothesis with the chosen strategy is
$6.1 \usep\sigma$
($4.5 \usep\sigma$).

\begin{figure}
  \renewcommand{\dataMCblind}{}
  \renewcommand{\expobs}{observed}
  \centering
  \includegraphics[height=.33\textheight]{Figures/dataMC/Run2/phoCR/SR4P/SYS_mZZGloose_central_pow\dataMCblind .pdf}
  \hfill
  \includegraphics[height=.33\textheight]{Figures/combine/inclusive/scan_\expobs_Run2_SR4P_phoCR_lepCR_mZZGloose.pdf}
  \caption{\captionScan{mass of the $\PZ\PZ\PGg$ system}{Loose}{cut-based ID}{s}{not }}
  \label{fig:scan_observed_inclusive_Run2_SR4P}
\end{figure}

\begin{figure}
  \renewcommand{\dataMCblind}{}
  \renewcommand{\expobs}{observed}
  \centering
  \includegraphics[height=0.33\textheight]{Figures/dataMC/Run2/lepCR/SR4P/SYS_mZZGwp90_central_pow\dataMCblind .pdf}
  \hfill
  \includegraphics[height=0.33\textheight]{Figures/combine/inclusive/impacts_\expobs_Run2_SR4P_phoMC_lepCR_mZZGwp90.pdf}
  \caption{Observed \captionImpact{mass of the $\PZ\PZ\PGg$ system}{\texttt{wp90}}{MVA ID}{s}{not }}
  \label{fig:impacts_observed_inclusive_Run2_SR4P}
\end{figure}

The maximum likelihood fit to the distribution of $m_{4\Pl\PGg}$ results in an expected signal strength of
$1.38{}^{+0.55}_{-0.48}$
($1.38{}^{+0.54}_{-0.47}\stat {}^{+0.01}_{-0.01}\thy {}^{+0.05}_{-0.03}\lum {}^{+0.12}_{-0.10}\syst$),
as shown in Figure~\ref{fig:scan_observed_inclusive_Run2_SR4P}.
The detailed breakdown of each source of systematic uncertainty is reported in
Figure~\ref{fig:impacts_observed_inclusive_Run2_SR4P}

The signal sample, which includes tau leptons in the final state,
has a cross section of 22.02\usep fb, as reported in Table~\ref{tab:listofsamples}.
Assuming that $\mathcal{B}(\PZ\to2\Pe) = \mathcal{B}(\PZ\to2\PGm) = \mathcal{B}(\PZ\to2\PGt)$
and that the difference in the phase space due to the different mass of the leptons are negligible,
it follows that $\frac{4}{9}$ of the events in the sample have a final state with four among electrons and muons,
while $\frac{1}{9}$ have four tau leptons and $\frac{4}{9}$ have two tau leptons and two electrons or muons.
Therefore the $\mathcal{B}(4\Pl)\times\sigma$, with $\Pl = \Pe, \PGm$, of the sample is 9.787\usep fb.

Using the signal strength obtained, the measured cross section for the production of
$\Pp\Pp \to 4\Pl\PGg$ ($\Pl = \Pe,\,\PGm$) at a centre-of-mass energy of $13\TeV$ is
$13.52{}^{+5.38}_{-4.66}$\usep fb
($13.52 {}^{+5.24}_{-4.55}\stat {}^{+0.05}_{-0.05}\thy {}^{+0.45}_{-0.31}\lum {}^{+1.15}_{-0.93}\syst$\usep fb).
