The extraction of the signal strength modifier $\mu$ proceeds through the maximization of the likelihood,
as for the four lepton channel.
This procedure is visualized by plotting the scan of the likelihood function as in the four lepton channel.

An estimate of the contribution of the different groups of nuisances is derived
by sequentially freezing them and the signal strength to the best fit value
and profiling the others.

Four groups of parameters are used in the following results:
\begin{itemize}
\item \textbf{theory:} uncertainties on the QCD scale, proton PDFs and on the value of \alpS;
%% \item \textbf{data-driven:} uncertainties related to the data-driven estimate of fake lepton or fake photon backgrounds;
\item \textbf{luminosity:} the uncertainty on the integrated luminosity corresponding to the data collected by the CMS experiment;
\item \textbf{others:} remaining experimental uncertainties, such as the lepton or photon efficiency scale factors or the \pileup{} weight;
\item \textbf{statistical:} the remaining uncertainty after freezing all of the nusiances.
\end{itemize}

For the second strategy, in which both fake leptons and photons are estimated from simulation,
the data-driven contribution is obviously null.

\begin{figure}
  \centering
  \includegraphics[height=.33\textheight]{Figures/dataMC/Run2/phoCR/SR3P/SYS_mWZGloose_central_pow\dataMCblind .pdf}
  \hfill
  \includegraphics[height=.33\textheight]{Figures/combine/inclusive/scan_\expobs_Run2_SR3P_phoCR_lepMC_mWZGloose.pdf}
  \caption{\captionScan{transverse mass of the $\PW\PZ\PGg$ system}{Loose}{cut-based ID}{d}{not }}
  \label{fig:scan_Run2_SR3P_phoCR_lepMC_mWZGloose}
\end{figure}

\begin{figure}
  \centering
  \includegraphics[height=.33\textheight]{Figures/dataMC/Run2/fullMC/SR3P/SYS_mWZGloose_central_pow\dataMCblind .pdf}
  \hfill
  \includegraphics[height=.33\textheight]{Figures/combine/inclusive/scan_\expobs_Run2_SR3P_phoMC_lepMC_mWZGloose.pdf}
  \caption{\captionScan{transverse mass of the $\PW\PZ\PGg$ system}{Loose}{cut-based ID}{s}{not }}
  \label{fig:scan_Run2_SR4P_phoMC_lepMC_mWZGloose}
\end{figure}

The theory group has the highest impact on the total uncertainty on the signal strength,
accounting for ${+}0.20/{-}0.19$ when the data-driven estimate for the \nonprompt photons is used
and ${+}0.37/{-}0.38$ when all the backgrounds are estimated from simulation.
The second most impacting group is the statistics, with an uncertainty of
$\pm 0.30$ and ${+}0.28/{-}0.27$ respectively.
The first strategy has a total uncertainty of ${+}0.40/{-}0.37$,
while for the second it is ${+}0.48/{-}0.47$,
around $20\usep\%$ higher.
