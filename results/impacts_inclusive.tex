\label{sec:impacts_inclusive}


\begin{figure}
  \centering
  \includegraphics[height=0.33\textheight]{Figures/dataMC/Run2/phoCR/SR4P/SYS_mZZGloose_central_pow\dataMCblind .pdf}
  \hfill
  \includegraphics[height=0.33\textheight]{Figures/combine/inclusive/impacts_\expobs_Run2_SR4P_phoCR_lepCR_mZZGloose.pdf}
  \caption{\captionImpact{mass of the $\PZ\PZ\PGg$ system}{Loose}{cut-based ID}{d}{not }}
  \label{fig:inclusive_cutID_phoCR_mZZGloose}
\end{figure}

\begin{figure}
  \centering
  \includegraphics[height=0.33\textheight]{Figures/dataMC/Run2/lepCR/SR4P/SYS_mZZGloose_central_pow\dataMCblind .pdf}
  \hfill
  \includegraphics[height=0.33\textheight]{Figures/combine/inclusive/impacts_\expobs_Run2_SR4P_phoMC_lepCR_mZZGloose.pdf}
  \caption{\captionImpact{mass of the $\PZ\PZ\PGg$ system}{Loose}{cut-based ID}{s}{not }}
  \label{fig:inclusive_cutID_phoMC_mZZGloose}
\end{figure}

\begin{figure}
  \centering
  \includegraphics[height=0.33\textheight]{Figures/dataMC/Run2/lepCR/SR4P/SYS_loosept_central_pow\dataMCblind .pdf}
  \hfill
  \includegraphics[height=0.33\textheight]{Figures/combine/inclusive/impacts_\expobs_Run2_SR4P_phoMC_lepCR_loosept.pdf}
  \caption{\captionImpact{transverse momentum of the photon}{Loose}{cut-based ID}{s}{not }}
  \label{fig:inclusive_cutID_phoMC_loosept}
\end{figure}

\begin{figure}
  \centering
  \includegraphics[height=0.33\textheight]{Figures/dataMC/Run2/lepCR/SR4P/SYS_mZZGwp90_central_pow\dataMCblind .pdf}
  \hfill
  \includegraphics[height=0.33\textheight]{Figures/combine/inclusive/impacts_\expobs_Run2_SR4P_phoMC_lepCR_mZZGwp90.pdf}
  \caption{\captionImpact{mass of the $\PZ\PZ\PGg$ system}{\texttt{wp90}}{MVA ID}{s}{not }}
  \label{fig:inclusive_mvaID_phoMC_mZZGwp90}
\end{figure}

\begin{figure}
  \centering
  \includegraphics[height=0.33\textheight]{Figures/dataMC/Run2/lepCR/SR4P/SYS_wp90pt_central_pow\dataMCblind .pdf}
  \hfill
  \includegraphics[height=0.33\textheight]{Figures/combine/inclusive/impacts_\expobs_Run2_SR4P_phoMC_lepCR_wp90pt.pdf}
  \caption{\captionImpact{transverse momentum of the photon}{\texttt{wp90}}{MVA ID}{s}{not }}
  \label{fig:inclusive_mvaID_phoMC_wp90pt}
\end{figure}

\begin{figure}
  \centering
  \includegraphics[height=0.33\textheight]{Figures/dataMC/Run2/lepCR/SR4P/SYS_mZZGwp80_central_pow\dataMCblind .pdf}
  \hfill
  \includegraphics[height=0.33\textheight]{Figures/combine/inclusive/impacts_\expobs_Run2_SR4P_phoMC_lepCR_mZZGwp80.pdf}
  \caption{\captionImpact{mass of the $\PZ\PZ\PGg$ system}{\texttt{wp80}}{MVA ID}{s}{not }}
  \label{fig:inclusive_mvaID_phoMC_mZZGwp80}
\end{figure}

\begin{figure}
  \centering
  \includegraphics[height=0.33\textheight]{Figures/dataMC/Run2/lepCR/SR4P/SYS_MVAcut_central_pow\dataMCblind .pdf}
  \hfill
  \includegraphics[height=0.33\textheight]{Figures/combine/inclusive/impacts_\expobs_Run2_SR4P_phoMC_lepCR_MVAcut.pdf}
  \caption{Distribution and impacts of the systematic uncertainties on the signal strength fit
    on the yield in the various bins of the photon MVA ID.
    \descriptionFakePhoton{s}.
    The FSR cut is not applied.
  }
  \label{fig:inclusive_kin_phoMC_MVAcut}
\end{figure}

In all cases, the leading contribution from the systematics is the uncertainty on the electron efficiency scale factors,
which result in a shift of approximately $\pm 0.05$ in the signal strength.
When the data-driven estimate of non-prompt photons is used, its effect on $\mu$ is around $\pm 0.02$,
as shown in Figure~\ref{fig:inclusive_cutID_phoCR_mZZGloose}.
The leading theoretical systematic is the QCD scale uncertainty on the cross section of the background samples,
while the experimental uncertainties are lead by the \pileup{} weight and the luminosity.
