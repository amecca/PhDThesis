\label{sec:likelihood_scans_FSRcut}

\begin{figure}
  \centering
  \includegraphics[height=.33\textheight]{Figures/dataMC_FSRcut/Run2/phoCR/SR4P/SYS_mZZGloose_central_pow\dataMCblind .pdf}
  \hfill
  \includegraphics[height=.33\textheight]{Figures/combine/FSRcut/scan_\expobs_Run2_SR4P_phoCR_lepCR_mZZGloose.pdf}
  \caption{\captionScan{mass of the $\PZ\PZ\PGg$ system}{Loose}{cut-based ID}{d}{}}
  \label{fig:scan_FSRcut_Run2_SR4P_phoCR_lepCR_mZZGloose}
\end{figure}

\begin{figure}
  \centering
  \includegraphics[height=.33\textheight]{Figures/dataMC_FSRcut/Run2/lepCR/SR4P/SYS_mZZGloose_central_pow\dataMCblind .pdf}
  \hfill
  \includegraphics[height=.33\textheight]{Figures/combine/FSRcut/scan_\expobs_Run2_SR4P_phoMC_lepCR_mZZGloose.pdf}
  \caption{\captionScan{mass of the $\PZ\PZ\PGg$ system}{Loose}{cut-based ID}{s}{}}
  \label{fig:scan_FSRcut_Run2_SR4P_phoMC_lepCR_mZZGloose}
\end{figure}

\begin{figure}
  \centering
  \includegraphics[height=.33\textheight]{Figures/dataMC_FSRcut/Run2/lepCR/SR4P/SYS_loosept_central_pow\dataMCblind .pdf}
  \hfill
  \includegraphics[height=.33\textheight]{Figures/combine/FSRcut/scan_\expobs_Run2_SR4P_phoMC_lepCR_loosept.pdf}
  \caption{\captionScan{transverse momentum of the photon}{Loose}{cut-based ID}{s}{}}
  \label{fig:scan_FSRcut_Run2_SR4P_phoMC_lepCR_loosept}
\end{figure}

\begin{figure}
  \centering
  \includegraphics[height=.33\textheight]{Figures/dataMC_FSRcut/Run2/lepCR/SR4P/SYS_mZZGwp90_central_pow\dataMCblind .pdf}
  \hfill
  \includegraphics[height=.33\textheight]{Figures/combine/FSRcut/scan_\expobs_Run2_SR4P_phoMC_lepCR_mZZGwp90.pdf}
  \caption{\captionScan{mass of the $\PZ\PZ\PGg$ system}{\texttt{wp90}}{MVA ID}{s}{}}
  \label{fig:scan_FSRcut_Run2_SR4P_phoMC_lepCR_mZZGwp90}
\end{figure}

\begin{figure}
  \centering
  \includegraphics[height=.33\textheight]{Figures/dataMC_FSRcut/Run2/lepCR/SR4P/SYS_wp90pt_central_pow\dataMCblind .pdf}
  \hfill
  \includegraphics[height=.33\textheight]{Figures/combine/FSRcut/scan_\expobs_Run2_SR4P_phoMC_lepCR_wp90pt.pdf}
  \caption{\captionScan{transverse momentum of the photon}{\texttt{wp90}}{MVA ID}{s}{}}
  \label{fig:scan_FSRcut_Run2_SR4P_phoMC_lepCR_wp90pt}
\end{figure}

\begin{figure}
  \includegraphics[height=.33\textheight]{Figures/dataMC_FSRcut/Run2/lepCR/SR4P/SYS_mZZGwp80_central_pow\dataMCblind .pdf}
  \hfill
  \centering
  \includegraphics[height=.33\textheight]{Figures/combine/FSRcut/scan_\expobs_Run2_SR4P_phoMC_lepCR_mZZGwp80.pdf}
  \caption{\captionScan{mass of the $\PZ\PZ\PGg$ system}{\texttt{wp80}}{MVA ID}{s}{}}
  \label{fig:scan_FSRcut_Run2_SR4P_phoMC_lepCR_mZZGwp80}
\end{figure}

\begin{figure}
  \centering
  \includegraphics[height=.33\textheight]{Figures/dataMC_FSRcut/Run2/lepCR/SR4P/SYS_MVAcut_central_pow\dataMCblind .pdf}
  \hfill
  \includegraphics[height=.33\textheight]{Figures/combine/FSRcut/scan_\expobs_Run2_SR4P_phoMC_lepCR_MVAcut.pdf}
  \caption{Likelihood scan for the signal strength parameter
    on the yield in the various bins of the photon MVA ID.
    \descriptionFakePhoton{s}.
    The FSR cut is applied.
    The effect of groups of nuisance parameters on the uncertainty is assessed by sequentially fixing their value in the fit.
  }
  \label{fig:scan_FSRcut_Run2_SR4P_phoMC_lepCR_MVAcut}
\end{figure}

Similarly to the inclusive selection, the sensitivity is dominated by the statistics,
with a symmetric impact on the signal strength ranging from 0.4 to 0.5.
The second most impacting group are the experimental uncertainties
which range from around $-0.05/+0.08$ to $-0.07/0.10$ depending on the strategy.
Theoretical uncertainties range from 0.03 to 0.05 and tend to have a symmetric effect on the signal uncertainty.
The luminosity has a smallest contribution, around $-0.03/+0.04$ for all of the strategies.
When applied, the uncertainty on the data-driven fake photon background shifts the signal strength by $-0.04/+0.08$.

\begin{table}
  \caption{
    Summary of the likelihood scans for the four lepton channel
    in the triboson fiducial region
    and contribution from each of the nuisance parameter groups to the total uncertainty on the signal strength.
  }
  \label{tab:scanl_SR4P_FSRcut}
  \small
  \resizebox{\textwidth}{!}{
  \begin{tabular}{lccccccccc}
    \toprule
    Fake photon & data-driven    &    MC          &    MC         &    MC          &    MC         &    MC          &    MC         \\
    Fake lepton & data-driven    & data-driven    & data-driven   & data-driven    & data-driven   & data-driven    & data-driven   \\
    Photon ID   & Loose (cut)    & Loose (cut)    & Loose (cut)   & wp90 (MVA)     & wp90 (MVA)    & wp80 (MVA)     & kinematic     \\
    Variable    &$m_{\PZ\PZ\PGg}$&$m_{\PZ\PZ\PGg}$& $\pt^\PGg$    &$m_{\PZ\PZ\PGg}$& $\pt^\PGg$    &$m_{\PZ\PZ\PGg}$& MVAcut        \\
    \midrule
    $\mu$       & 1.000000       & 1.000000       & 1.000000      & 1.000000       & 1.000000      & 1.000000       & 1.000000      \\
    total       & -0.424/+0.513  & -0.425/+0.503  & -0.431/+0.514 & -0.406/+0.480 & -0.414/+0.493  & -0.415/+0.496  & -0.404/+0.476 \\
    \hline
    data-driven & -0.037/+0.078  & -0.001/+0.001  & -0.001/+0.001 & -0.001/+0.001  & -0.001/+0.001 & -0.001/+0.001  & -0.002/+0.002 \\
    luminosity  & -0.026/+0.046  & -0.028/+0.043  & -0.028/+0.043 & -0.027/+0.041  & -0.027/+0.042 & -0.026/+0.040  & -0.031/+0.047 \\
    theory      & -0.030/+0.031  & -0.042/+0.040  & -0.040/+0.038 & -0.037/+0.034  & -0.037/+0.035 & -0.034/+0.031  & -0.050/+0.056 \\
    syst        & -0.043/+0.081  & -0.053/+0.080  & -0.052/+0.080 & -0.050/+0.076  & -0.050/+0.076 & -0.048/+0.072  & -0.065/+0.095 \\
    stat        & -0.419/+0.498  & -0.419/+0.493  & -0.425/+0.504 & -0.400/+0.471  & -0.409/+0.484 & -0.410/+0.488  & -0.394/+0.461 \\
    \bottomrule
  \end{tabular}
  }
\end{table}

Choosing the strategy that estimates the fake leptons from data while the fake photons are taken from simulation,
and identifies the photon with the \texttt{wp90} working point of the MVA-based ID,
the fit to the $m_{4\Pl\PGg}$ results in an expected signal strength of
$1.000{}^{+0.551}_{-0.426}$
($1.000{}^{+0.028}_{-0.015}\lum {}^{+0.014}_{-0.015}\thy {}^{+0.063}_{-0.032}\syst {}^{+0.546}_{-0.424}\stat$).

As discussed in Section~\ref{sec:likelihood_scans_inclusive},
the $\mathcal{B}(4\Pl)\times\sigma$, with $\Pl = \Pe, \PGm$, of the signal sample is 9.787\usep fb.
Furthermore, referring to the generator-level study described in Section~\ref{sec:signal_genstudy},
the 52\usep\% of the events passing the full selection and the cut to suppress the FSR contribution
are actually from triboson production $\Pp\Pp \to \PZ\PZ\PGg$.
Thus the cross section from the simulation is 5.089\usep fb.

The measured cross section for the process $\Pp\Pp \to \PZ\PZ\PGg \to 4\Pl\PGg$,
with $\Pl = \Pe, \PGm$ is
$5.089{}^{+2.804}_{-2.168}$\usep fb
($5.089{}^{+0.142}_{-0.076}\lum {}^{+0.071}_{-0.076}\thy {}^{+0.321}_{-0.163}\syst {}^{+2.779}_{-2.158}\stat$\usep fb)
at a centre-of-mass energy of 13\TeV.

\paragraph{Observed\\}
\begin{figure}
  \renewcommand{\dataMCblind}{}
  \renewcommand{\expobs}{observed}
  \centering
  \includegraphics[height=.33\textheight]{Figures/dataMC/Run2/lepCR/SR4P/SYS_mZZGwp80_central_pow\dataMCblind .pdf}
  \hfill
  \includegraphics[height=.33\textheight]{Figures/combine/inclusive/scan_\expobs_Run2_SR4P_phoMC_lepCR_mZZGwp80.pdf}
  \caption{\captionScan{mass of the $\PZ\PZ\PGg$ system}{\texttt{wp80}}{MVA ID}{s}{not }}
  \label{fig:scan_observed_FSRcut_Run2_SR4P}
\end{figure}

\note{
This is just a placeholder to dump the unblinded results for the triboson fiducial region 4L, which will be inserted in the previous paragraph.}
The strategy is: see the unblinded scan in Figure~\ref{fig:scan_observed_FSRcut_Run2_SR4P} and the impacts in Figure~\ref{fig:impacts_observed_FSRcut_Run2_SR4P}.

The signal strength is $0.775{}^{+0.391}_{-0.301}$
\\
($0.775{}^{+0.019}_{-0.010}\lum {}^{+0.012}_{-0.013}\thy {}^{+0.044}_{-0.024}\syst {}^{+0.388}_{-0.300}\stat$).

The measured cross section is
$3.944{}^{+1.988}_{-1.534}$\usep fb
\\
($3.944{}^{+0.099}_{-0.051}\lum {}^{+0.059}_{-0.067}\thy {}^{+0.225}_{-0.122}\syst {}^{+1.972}_{-1.526}\stat$\usep fb).
