\label{sec:likelihood_scans_FSRcut}

\begin{figure}
  \centering
  \includegraphics[height=.33\textheight]{Figures/VVGammaAnalyzer_FSRcut/Run2/phoCR/SR4P/SYS_mZZGloose_central_pow\dataMCblind .pdf}
  \hfill
  \includegraphics[height=.33\textheight]{Figures/combine/FSRcut/scan_\expobs_Run2_SR4P_phoCR_lepCR_mZZGloose.pdf}
  \caption{\captionScan{mass of the $\PZ\PZ\PGg$ system}{Loose}{cut-based ID}{d}{}}
  \label{fig:scan_FSRcut_Run2_SR4P_phoCR_lepCR_mZZGloose}
\end{figure}

\begin{figure}
  \centering
  \includegraphics[height=.33\textheight]{Figures/VVGammaAnalyzer_FSRcut/Run2/lepCR/SR4P/SYS_mZZGloose_central_pow\dataMCblind .pdf}
  \hfill
  \includegraphics[height=.33\textheight]{Figures/combine/FSRcut/scan_\expobs_Run2_SR4P_phoMC_lepCR_mZZGloose.pdf}
  \caption{\captionScan{mass of the $\PZ\PZ\PGg$ system}{Loose}{cut-based ID}{s}{}}
  \label{fig:scan_FSRcut_Run2_SR4P_phoMC_lepCR_mZZGloose}
\end{figure}

\begin{figure}
  \centering
  \includegraphics[height=.33\textheight]{Figures/VVGammaAnalyzer_FSRcut/Run2/lepCR/SR4P/SYS_loosept_central_pow\dataMCblind .pdf}
  \hfill
  \includegraphics[height=.33\textheight]{Figures/combine/FSRcut/scan_\expobs_Run2_SR4P_phoMC_lepCR_loosept.pdf}
  \caption{\captionScan{transverse momentum of the photon}{Loose}{cut-based ID}{s}{}}
  \label{fig:scan_FSRcut_Run2_SR4P_phoMC_lepCR_loosept}
\end{figure}

\begin{figure}
  \centering
  \includegraphics[height=.33\textheight]{Figures/VVGammaAnalyzer_FSRcut/Run2/lepCR/SR4P/SYS_mZZGwp90_central_pow\dataMCblind .pdf}
  \hfill
  \includegraphics[height=.33\textheight]{Figures/combine/FSRcut/scan_\expobs_Run2_SR4P_phoMC_lepCR_mZZGwp90.pdf}
  \caption{\captionScan{mass of the $\PZ\PZ\PGg$ system}{\texttt{wp90}}{MVA ID}{s}{}}
  \label{fig:scan_FSRcut_Run2_SR4P_phoMC_lepCR_mZZGwp90}
\end{figure}

\begin{figure}
  \centering
  \includegraphics[height=.33\textheight]{Figures/VVGammaAnalyzer_FSRcut/Run2/lepCR/SR4P/SYS_wp90pt_central_pow\dataMCblind .pdf}
  \hfill
  \includegraphics[height=.33\textheight]{Figures/combine/FSRcut/scan_\expobs_Run2_SR4P_phoMC_lepCR_wp90pt.pdf}
  \caption{\captionScan{transverse momentum of the photon}{\texttt{wp90}}{MVA ID}{s}{}}
  \label{fig:scan_FSRcut_Run2_SR4P_phoMC_lepCR_wp90pt}
\end{figure}

\begin{figure}
  \includegraphics[height=.33\textheight]{Figures/VVGammaAnalyzer_FSRcut/Run2/lepCR/SR4P/SYS_mZZGwp80_central_pow\dataMCblind .pdf}
  \hfill
  \centering
  \includegraphics[height=.33\textheight]{Figures/combine/FSRcut/scan_\expobs_Run2_SR4P_phoMC_lepCR_mZZGwp80.pdf}
  \caption{\captionScan{mass of the $\PZ\PZ\PGg$ system}{\texttt{wp80}}{MVA ID}{s}{}}
  \label{fig:scan_FSRcut_Run2_SR4P_phoMC_lepCR_mZZGwp80}
\end{figure}

\begin{figure}
  \centering
  \includegraphics[height=.33\textheight]{Figures/VVGammaAnalyzer_FSRcut/Run2/lepCR/SR4P/SYS_MVAcut_central_pow\dataMCblind .pdf}
  \hfill
  \includegraphics[height=.33\textheight]{Figures/combine/FSRcut/scan_\expobs_Run2_SR4P_phoMC_lepCR_MVAcut.pdf}
  \caption{Likelihood scan for the signal strength parameter
    on the yield in the various bins of the photon MVA ID.
    \descriptionFakePhoton{s}.
    The FSR cut is applied.
    The effect of groups of nuisance parameters on the uncertainty is assessed by sequentially fixing their value in the fit.
  }
  \label{fig:scan_FSRcut_Run2_SR4P_phoMC_lepCR_MVAcut}
\end{figure}

Similarly to the inclusive selection, the measurement is statistically limited,
with a symmetric impact on the signal strength ranging from 0.4 to 0.5.
The second most impacting group are the experimental uncertainties
which range from around $-0.05/+0.08$ to $-0.07/0.10$ depending on the strategy.
The contributions from the theoretical uncertainties range from 0.03 to 0.05 and tend to have a symmetric effect on the signal uncertainty.
The luminosity has a smallest contribution, around $-0.03/+0.04$ for all of the strategies.
When applied, the uncertainty on the data-driven fake photon background shifts the signal strength by $-0.04/+0.08$.

%% Choosing the strategy that estimates the fake leptons from data while the fake photons are taken from simulation,
%% and identifies the photon with the \texttt{wp90} working point of the MVA-based ID,
%% the fit to the $m_{4\Pl\PGg}$ results in an expected signal strength of
%% $1.000{}^{+0.551}_{-0.426}$
%% ($1.000{}^{+0.028}_{-0.015}\lum {}^{+0.014}_{-0.015}\thy {}^{+0.063}_{-0.032}\syst {}^{+0.546}_{-0.424}\stat$).

%% As discussed in Section~\ref{sec:likelihood_scans_inclusive},
%% the $\mathcal{B}(4\Pl)\times\sigma$, with $\Pl = \Pe, \PGm$, of the signal sample is 9.787\usep fb.
%% Furthermore, referring to the generator-level study described in Section~\ref{sec:signal_genstudy},
%% the 52\usep\% of the events passing the full selection and the cut to suppress the FSR contribution
%% are actually from triboson production $\Pp\Pp \to \PZ\PZ\PGg$.
%% Thus the cross section from the simulation is 5.089\usep fb.

%% The measured cross section for the process $\Pp\Pp \to \PZ\PZ\PGg \to 4\Pl\PGg$,
%% with $\Pl = \Pe, \PGm$ is
%% $5.089{}^{+2.804}_{-2.168}$\usep fb
%% ($5.089{}^{+0.142}_{-0.076}\lum {}^{+0.071}_{-0.076}\thy {}^{+0.321}_{-0.163}\syst {}^{+2.779}_{-2.158}\stat$\usep fb)
%% at a centre-of-mass energy of 13\TeV.
