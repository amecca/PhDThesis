\label{sec:statistical_analysis}
A statistical analysis is conducted to assess the presence of the signal process in the observed data,
quantify the significance of the excess over background predictions and measure its cross section.
The statistical inference is performed with the \textsc{Combine Tool} software package~\cite{CMS-NOTE-2011-005,CMS-CAT-23-001}.

\providecommand{\poissonpdf}[2]{\ensuremath{ \frac{{#1}^{#2}}{#2!}e^{-#1} }}

\subsection{Likelihood and nuisance parameters}
The likelihood function is defined as the probability density function for a set of parameters of a model
that quantifies the agreement with a certain set of experimental observables (data).
The model adopted for this analysis defines a signal strength modifier $\mu$,
that multiplies the production cross section of the signal process and leaves all the other processes unchanged.
Each independent source of systematic uncertainty described in Section \ref{sec:systematics} is assigned a nuisance parameter $\theta_i$, and the full set is denoted $\vec\theta$.
They are of no direct interest for this analysis, but must be considered in the fitting procedure to extract correct results.
They enter the model through their probability density function $p_i(\tilde{\theta_i}|\theta_i)$,
which is the probability of measuring a certain value of the parameter given that the true value is $\theta_i$.
Furthermore, the expected yields of background, $b$, and signal, $s$, depend on the value of the nuisance parameters.

The global likelihood function is thus defined as:
\begin{equation}
  \label{eq:likelihood_full}
  \Likelihood(data\, |\, \mu, \vec\theta\,) = \prod_c \Likelihood_c(data\, |\, \mu \cdot s(\vec\theta\,) + b(\vec\theta\,)) \cdot \prod_i p_i(\tilde{\theta_i}\, |\, \theta_i)
\end{equation}
where c runs over all the channels, which are the four data-taking periods (2016preVFP, 2016postVFP, 2017, 2018).
The extraction of the signal strength proceeds through the maximization of the complete likelihood function by varying the parameter of interest $\mu$ and the nuisances.
The $\Likelihood_c$ functions are the PDF of the binned distributions in each channel, and are given by the product of Poisson probabilities for every bin $j$ to observe $n_j$ events:
\begin{equation}
  \label{eq:likelihood_bin}
  \Likelihood_c(data\, |\, \mu \cdot s(\vec\theta\,) + b(\vec\theta\,)) =
                       \prod_j \poissonpdf{\left( \mu \cdot s_j(\vec\theta\,) + b_j(\vec\theta\,) \right)}{n_j}
\end{equation}

\subsection{Treatment of nuisance parameters}
Systematics uncertainties can be categorized into two main classes: the ones that affect only the event yield, and those that have an impact also on the shape of the predicted distributions.
Most of the uncertainties of the first class are parametrized with a log-normal distribution:
\begin{equation}
  \label{eq:lnNdef}
  \Probability(\tilde{\theta}\,|\,\theta) = \frac{1}{\sqrt{2 \pi} \text{ln} k} \cdot \frac{1}{\tilde{\theta}} \cdot \text{exp} \left( -\frac{(\text{ln}(\tilde{\theta}/\theta_m))^2}{2 \text{ln}^2 k} \right)
\end{equation}
which is the distribution of a random variable whose logarithm is normally distributed, with mean $\mu$ = $\text{ln}(\theta_m)$ and standard deviation $\sigma$ = $\text{ln}(k)$.
%% where the parameters $theta_s$ and $k$ can be defined in terms of the mean and standard deviation of a normally distributed variable: $\theta_m = e^{\mu}$ and $k = e^{\sigma}$.
The log-normal is used instead of a Gaussian because it enforces the positive-definite normalization for the nuisance modelled, which is usually multiplying an event yield and thus cannot be negative.

The remaining systematics in the first class are those that represent a background coming from a statistically limited control region, such as the fake leptons and photons.
These are dominated by the statistically uncertainty in the control region and are modelled with a Gamma distribution:
\begin{equation}
  \label{eq:gammadef}
  \Probability(\theta\,|\,N\alpha) = \frac{1}{\Gamma(N) \alpha^N} \theta^{N-1} e^{-\theta/\alpha}
\end{equation}
where $N$ is the number of events in the control region, $\theta$ is the average transfer factor and $\Gamma(x)$ is the Gamma function.

The shape uncertainties of the second class are accounted for by interpolating the event fraction for each bin of three histograms:
the one obtained for the central value, and the two obtained by shifting the nuisance parameter up and down by one standard deviation~\cite{CMS-CAT-23-001}.

\subsection{Quantifying an excess}
To quantify the statistical significance of an excess of events over the background-only hypothesis, the following test statistic is used:
\begin{equation}
  \label{eq:test_statistic}
  t_0 = -2\text{ln} \frac {\Likelihood(data\,|\,0,\widehat{\vec{\theta_0}}\,)} {\Likelihood(data\,|\,\hat\mu,\widehat{\vec\theta}_{\hat\mu})}\,,\quad \text{with}\, \hat\mu \ge 0
\end{equation}

The numerator is evaluated under the background-only hypothesis ($\mu$ = 0), and $\widehat{\vec{\theta_0}}$ is the set of values of nuisance parameters that maximizes it under this null hypothesis.
The denominator is evaluated under the alternative signal + background hypothesis,
and the values $\hat{\mu}$ and $\widehat{\vec{\theta}}_{\hat\mu}$
are those that maximize the likelihood in this hypothesis.
This quantity is positive for a signal-like excess ($\mu$ > 0) and becomes 0 in the absence of an excess ($\mu$ = 0).

The significance of an excess is expressed in terms of the local \textit{p-value}, which is the probability to obtain a value of the test statistic $t_0$ greater than or equal to the one observed in experimental data, under the background-only hypothesis:
\begin{equation}
  \label{eq:pvalue}
  p_0 = \Probability(t_0 \ge t_0^{obs}\, |\, \mu = 0)
\end{equation}
That is, $p_0$ is the probability that a local statistical fluctuation of the background yield
is compatible with the observed data in the background only hypothesis,
at least as much as the signal hypothesis.

The p-value is usually expressed as a \textit{significance} $Z$ using the Gaussian one-sided integral:
\begin{equation}
  \label{eq:significance}
  p_0 = \int_Z^\infty \frac{1}{\sqrt{2\pi}}e^{-x^2/2}dx
\end{equation}

The conventional values of $Z$ = 3$\sigma$ and $Z$ = 5$\sigma$, corresponding to p-values of $1.3 \times 10^{-3}$ and $2.8 \times 10^{-7}$, are used to claim evidence for and the discovery of a new phenomenon respectively.


\subsection{Likelihood scans}
\label{sec:likelihood_scans}
The extraction of the signal strength modifier $\mu$ proceeds through the maximization of the likelihood.
%% as described in Section~\ref{sec:statistical_analysis}.
This procedure can be visualized by scanning the likelihood function for several values of the parameter $\mu$
accounting for the systematic uncertainties affecting the measurement.
For each value the best fit value of the nuisance parameters is computed,
and the resulting value of the likelihood is stored.
The corresponding points are then plotted as a function of $\mu$.

Usually the auxiliary quantity $-2\Delta\ln\Likelihood$ (defined as $t_0$ in Equation~\ref{eq:test_statistic})
is used instead of the likelihood itself.
The width of the its profile is linked to the uncertainty on the estimate of $\mu$ from the fit.
More precisely, the set of values $\{ \mu \mid -2\Delta\ln\Likelihood(\mu) < 1 (4) \}$ corresponds to the 68\usep\% (95\usep\%) confidence interval
in the case of a one-dimensional likelihood fit.

This procedure can also be performed by fixing the values of one or more nuisances instead of allowing them to be fitted by the algorithm.
The effect of fixing the value of one or more parameters is a reduction in the overall width of the likelihood shape.
This difference is ascribed to the effect of the \textit{frozen} parameters.

Nuisance parameters are categorized as follows:
\begin{itemize}
\item \textbf{theory:} uncertainties on the QCD scale, proton PDFs and on the value of \alpS;
%% \item \textbf{data-driven:} uncertainties related to the data-driven estimate of fake lepton or fake photon backgrounds;
\item \textbf{luminosity:} the uncertainty on the integrated luminosity corresponding to the data collected by the CMS experiment;
\item \textbf{experimental:} the other experimental uncertainties, such as the lepton or photon efficiency scale factors or the \pileup{} weight,
      including the uncertainty on the data-driven background prediction;
\item \textbf{statistical:} the remaining uncertainty after freezing all of the nuisances.
\end{itemize}

\subsection{Systematic impacts}
\label{sec:systematic_impacts}
The impact of a given source of systematic is defined as the shift $\Delta\mu$ on the signal strength
induced by fixing the corresponding nuisance parameter $\theta_i$ to its $\pm 1 \sigma$ values,
while profiling the other parameters as normal~\cite{CERN-PH-EP-2014-214}.
This is effectively a measure of the correlation between the $i$-th nuisance and the signal strength,
and is useful for determining which nuisances have the largest effect on the uncertainty.
